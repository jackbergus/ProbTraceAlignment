% This is samplepaper.tex, a sample chapter demonstrating the
% LLNCS macro package for Springer Computer Science proceedings;
% Version 2.20 of 2017/10/04
%
\documentclass[runningheads]{llncs}
%
\usepackage{graphicx}
\usepackage{hyperref}
\usepackage{multicol}
\usepackage[final]{changes}
\usepackage{url}
\usepackage{amsmath}
\usepackage{bm}
\usepackage{multirow}
\usepackage{tcolorbox}
\usepackage{rotating}
\usepackage{latexsym,amssymb,amsmath}
\usepackage{makecell}
\usepackage{xspace}
\usepackage{paralist}
\usepackage{wrapfig}
\usepackage{adjustbox}
\usepackage{cite}
\usepackage{todonotes}
\input{neue_imports}
\input{neue_macros}
\newlist{myalist}{enumerate*}{1}
\setlist[myalist]{label=\textbf{(\arabic*)}}
\newlist{mylist}{enumerate*}{1}
\setlist[mylist]{label=\textit{(\roman*)}}
\newlist{alphalist}{enumerate*}{1}
\setlist[alphalist]{label=\textbf{(\alph*)}}
\renewcommand*{\UrlFont}{\ttfamily\smaller\relax}
\usepackage{etoolbox}

\sloppy

\begin{document}
%
\title{A Tool for Computing\\Probabilistic Trace Alignments}
%
%\titlerunning{Abbreviated paper title}
% If the paper title is too long for the running head, you can set
% an abbreviated paper title here
%
\author{Giacomo Bergami\inst{1}\orcidID{0000-0002-1844-0851} \and
Fabrizio Maria Maggi\inst{1}\orcidID{0000-0002-9089-6896} \and
Marco Montali\inst{1}\orcidID{0000-0002-8021-3430} \and
Rafael Pe\~naloza\inst{2}\orcidID{0000-0002-2693-5790}}
%
\authorrunning{G.~Bergami, F.M.~Maggi, M.~Montali and R.~Pe\~naloza}
% First names are abbreviated in the running head.
% If there are more than two authors, 'et al.' is used.
%
\institute{	Free University of Bozen-Bolzano, Italy \\\email{gibergami@unibz.it,\{maggi,montali\}@inf.unibz.it}
	\and
	University of Milano-Bicocca \\\email{rafael.penaloza@unimib.it}}
%
\maketitle              % typeset the header of the contribution
%
\begin{abstract}
Alignments pinpoint trace deviations in a process model and quantify their severity. However, approaches based on trace alignments use crisp process models and recent probabilistic conformance checking approaches check the degree of conformance of an event log with respect to a stochastic process model instead of finding trace alignments. In this paper, for the first time, we provide a conformance checking approach based on trace alignments using stochastic Workflow nets. Conceptually, this requires to handle the two possibly contrasting forces of the cost of the alignment on the one hand and the likelihood of the model trace with respect to which the alignment is computed on the other.

\keywords{Stochastic Petri nets \and Conformance Checking \and Alignments.}
\end{abstract}

\section{Introduction}


\label{introduction}
%Trace alignment is a well-known technique in conformance checking \cite{DBLP:conf/edoc/AdriansyahDA11} providing both a numerical assessment of the degree of conformance of a log trace with respect to a model, as well as a repair strategy if such trace does not conform to the given model. At the time of the writing,
%
In the existing literature on conformance checking, a common approach is based on trace alignment \cite{DBLP:conf/edoc/AdriansyahDA11}. This approach uses crisp process models as reference models. Yet, recently developed probabilistic conformance checking approaches provide a numerical quantification of the degree of conformance
%the existing approaches are used to check the degree of conformance
of an event log with a stochastic process model by either assessing the distribution discrepancies \cite{DBLP:conf/bpm/LeemansSA19}, or by exploiting entropy-based measures \cite{DBLP:conf/icpm/PolyvyanyyK19,DBLP:journals/tosem/PolyvyanyySWCM20}.
As these strategies are not based on trace alignments, these cannot be directly used to repair a given trace with one of the traces generated by a stochastic process model.
%As traces generated by such models are associated to a probability exhibiting its representativeness and relevance within the model, probabilistic trace alignment techniques should take into account the combined provision of trace probability and alignment cost.
%instead of finding trace alignments.
%
In this paper, we provide for the first time an approach for the probabilistic alignment of a trace and a stochastic reference
model. This approach is not comparable with the existing literature on probabilistic conformance checking as its output is not numeric but consists of a ranked list of alignments.
Providing different alignment options is useful since, conceptually, probabilistic trace alignment requires the analyst to 
%handle the two possibly contrasting forces of the cost of the alignment on the one hand and the likelihood of the model trace with respect to which the alignment is computed.
%We consider the important tradeoff between both
%aspects.
balance between the likelihood of the model trace with respect to which the alignment is computed and the cost of the alignment. 
%(if the cost of the alignment is too high even if the model trace is very likely applying too many changes in the original trace is in turn not very likely).

\begin{figure}[!t]
	\centering
	\includegraphics[width=.49\textwidth]{images/petri_tut.pdf}
	\caption{a simple Stochastic Workflow Net.}\label{fig:petri_tut}
	%\vspace{-1.4cm}
\end{figure}
%For example, the probabilistic alignment of a trace with a stochastic net could be represented by the model trace maximizing the combined provision of minimum trace alignment cost and maximum model trace probability. 
With reference to Figure~\ref{fig:petri_tut}, a user might be interested to align the log trace $\langle \textsf{close order},\,\textsf{archive order}\rangle$ with one of the two possible model traces $\langle\textsf{close order},$ $\textsf{accept order},\,\textsf{pay order},\,\textsf{archive order}\rangle$ or $\langle$\textsf{close order}, \textsf{refuse order}, \textsf{archive order}$\rangle$. While the latter trace provides the least alignment cost though the model trace has a low probability ($0.1$), the former gives a slightly greater alignment cost while providing a higher model trace probability ($0.9$). Since, depending on the context, analysts might prefer either the former or the latter alignment, providing a selection of the best $k$ alignments among all the distinct model traces empowers the analysts to find their own trade-off between alignment cost and model trace probability.
%However, in some cases, the user could prefer to identify an alignment with a lower cost even if based on a less probable model trace, while, in other cases, the user could favor a model trace with a higher probability at the expense of a higher alignment cost. Therefore, to provide users with an instrument that allows them to find their own trade-off between alignment cost and model trace probability, we need to return the best $k$ alignments among all the distinct model traces.


%Since when aligning an event log with a stochastic net distinct model traces have different probabilities, the retrieval of the best model trace maximizing the combined provision of minimum trace alignment cost and maximum model trace probability might not suffice. In some cases, indeed, the user could prefer to identify an alignment with a lower cost even if based on a less probable model trace, while, in other cases, the user could favor a model trace with a higher probability at the expense of a higher alignment cost. We consider, therefore, the important tradeoff between both aspects.
%Therefore, in this paper, we propose trace alignment approaches that return the best

To do this, we frame the probabilistic trace alignment problem into the well-known $k$-Nearest Neighbors ($k$NN) problem \cite{Altman} that refers to finding the $k$ nearest data points to a \textit{query}  from a set  of \textit{data points} via a distance function.
We introduce two ranking strategies. The first one is based on a brute force approach that reuses existing trace aligners such as \cite{DBLP:conf/edoc/AdriansyahDA11,LeoniM17}, where the (optimal) ranking of the top-k alignments is obtained by computing the Levensthein distance between the trace to be aligned and all the model traces and by multiplying each of these distances by the probability of the corresponding model trace. However, even if this approach returns the best trace alignment ranking for a query trace, the alignments must be computed a-new for all the possible traces to be aligned. For models generating a large number of model traces, this would clearly become unfeasible. Therefore, we propose a second strategy that produces an approximate ranking where traces are represented as numerical vectors via an embedding. {Then, by exploiting ad-hoc data structures,
	%such as Vp-Trees \cite{Fu2000}, Kd-Trees \cite{Maneewongvatana99}, and M-Trees \cite{Ciaccia},
	we can retrieve the neighborhood of size $k$ containing the traces similar to the given query  by pre-ordering (\textit{indexing}) the model traces  via the aformentioned distance. 
	%Thus, we do not need to analyze the entire space, but just start the search from the top-$1$ alignment. 
	If the embeddings for our model traces are independent of the query of choice, this would not require to constantly recompute the numeric vector representation for the model traces.
	%	
	
	%%%%% Proposed part as the last part of the introduction:
	%\texttt{\color{red}[TODO]}
	%\todo{this is too specific for an introduction; in particular, too many details on how the experiments are done.}
	We implemented both strategies and perform experiments using a real life event log coming from a hospital system to empirically evaluate the properties of our proposed  strategy. We assessed our proposed as follows:
\begin{mylist}
	\item first, we evaluate the degree of approximation introduced by the approximate-ranking approach if compared with the optimal-ranking. We observe that different embedding strategies provide a trade-off between ranking stability vs. precision (\S\ref{subsec:apprp}).
	\item Last, we evaluate the computational time required to both generate the embeddings and to assess the similarity between the embeddings. We observe that approximate-ranking alignments provide the best trade-off between accuracy and efficiency (\S\ref{subsec:efficio}).
\end{mylist}

\begin{figure*}[!t]
	\begin{minipage}{.49\textwidth}
		\centering
		\includegraphics[width=.7\textwidth]{images/petri.pdf}
		\caption{A sample \uswn $N$. Labels are shown in green, $\tau$ transitions in grey, weights in magenta.}\label{fig:spn}
	\end{minipage}\hfill \begin{minipage}{.49\textwidth}
	\centering
		\includegraphics[width=.7\textwidth]{images/rg.pdf}
		\caption{Reachability graph  of the \uswn $N$. Probabilities are shown in violet.}\label{fig:rg}
	\end{minipage}

	\begin{minipage}{.4\textwidth}
		\centering \includegraphics[width=.7\textwidth]{images/running_example.pdf}
		\caption{Preliminary Transition graph $TG$ encoding the SWN $N$ with no $\tau$-closures.}\label{fig:lmc}\label{fig:orig}
	\end{minipage}\hfill \begin{minipage}{.4\textwidth}
	\centering \includegraphics[width=.7\textwidth]{images/closed_example.pdf}
		\caption{Transition graph $TG$ resulting from $N$ after $\tau$-closure.}\label{fig:closed}
	\end{minipage}
	\vspace{-.6cm}
\end{figure*}
\begin{figure}[!t]
	\centering
	\includegraphics[width=.45\textwidth]{images/petri_tut.pdf}
	\caption{Stochastic Workflow Net modeling our use-case scenario.}\label{fig:petri_tut}
	%\vspace{-0.6cm}
\end{figure}
\begin{figure}[!t]
	\centering\includegraphics[width=.8\textwidth]{images/pipeline}
	\caption{Proposed pipeline to assess the probabilistic trace alignment.}\label{fig:pipe}
\end{figure}
\medskip

\section{Probabilistic Trace Alignment Pipeline}
%We now describe the proposed technique for computing probabilistic trace alignments. 
Our approach (\figurename~\ref{fig:pipe}) takes as input
\begin{inparaenum}[\it (i)]
	\item a reference model represented as an Stochastic Workflow Nets $\net$ or an equivalent Transition Graph $G$,
	\item a minimum, positive probability threshold $\pmin \in (0,1]$
	\item a trace $\trace$ of interest,
\end{inparaenum}
and returns a ranking over all the $\net$-traces having a probability greater than or equal to $\pmin$, based on a combined consideration of their probability values and their distance to $\trace$. First, we discuss input formats for \textit{(i)}.


%\section{Modeling Probabilistic Dynamic Systems}
%\label{sec:models}
%In this section, we introduce the different models and techniques that will constitute the basis for representing and computing probabilistic trace alignments.


%\subsection{Input}
%\subsection{Stochastic Workflow Nets}\label{subsec:spn}

\textbf{{Format: \texttt{Petri\_PNML}}.} Petri Nets and Generalized Stochastic Petri Nets are well-established formalisms \cite{DBLP:journals/tosem/PolyvyanyySWCM20} for modelling processes \cite{RoggeSoltiAW13} represented in the Petri Net Markup Language, supported by our tool. Due to the lack of space, we refer to \cite{spdwe} for the usual notation over Petri Nets. We restrict our interest to an interesting class of $1$-\textit{bounded} stochastic Petri nets with no timed transitions, namely \textsc{untimed Stochastic Workflow Nets} denoted as $N$. We now sketch the properties of the SWN accepted in PNML format: we consider two customary markings: the \emph{input} (resp.~\emph{output}) marking $m_{in}$ (resp.~$m_{out}$) assigning a single token to the input (resp.~output) place, and no token elsewhere. We assume to have a set $\alphabet = \tasks \cup \set{\tau}$ of labels, where labels in $\tasks$ indicate process tasks, whereas $\tau$ indicates an invisible execution step ($\tau$-transition). A \emph{trace} is a finite sequence of labels from $\tasks$.
%\begin{definition} An \emph{untimed Stochastic Workflow Net (\uswn)}
%	is a tuple $\net = (P,T,F,\ell,W)$ where:
%	\begin{inparaenum}[\itshape (i)]
%		%\begin{inparaenum}
%		\item $(P,T,F)$ is a standard \emph{Workflow net} with places $P$, transitions $T$, and flow relation $F$ such that there is exactly one \emph{input place} with no incoming arc, and exactly one \emph{output place} with no outgoing arcs;
%		\item $\ell: T \rightarrow \alphabet$ is a \emph{labeling function} mapping each transition $t \in T$ into a label $\ell(t) \in \alphabet$ - this either indicates the task executed upon firing $t$, or the fact that $t$ is an invisible transition (in the latter case, $\ell(t) = \tau$);
%		\item $W\colon T\to \mathbb{R}^+$ is a \emph{weight function} assigning a positive firing weight to each transition of the net.
%	\end{inparaenum}
%\end{definition}
%Given an \uswn $\net$, we use dot notation to extract its constitutive components (e.g., $\net.P$ denotes its places). \emph{The same dot notation will be used for the other structures introduced in the paper}. We also use $\net.in$ and $\net.out$ to respectively denote the input and output place of $\net$.
The current state of execution is captured by a marking, i.e., a multiset of places $P$ indicating how many tokens populate each place.
%As pointed out above, \emph{we always assume, as customary in BPM, that the input \uswn is \underline{bounded}}, that is, in every state the number of tokens associated to each place cannot exceed a maximum, fixed threshold.
The notions of transition enablement and firing are also the standard ones  \cite{MarsanCB84}: %, which provides the basis for capturing the stochastic behavior of the net. 
%We use the following notation: given a marking $\marking$ over \uswn $\net$,  $\enaset{\marking}{\net}$ is the set of enabled transitions in $\marking$; given transition $t \in \enaset{\marking}{\net}$, we write $\fire{\marking}{t}{\marking'}{\net}$ to capture the fact that, with, firing $t$ in $\marking$ results in the new marking $\marking'$. A \emph{firing sequence starting from marking $\marking_0$} is a sequence $t_1\cdots t_n$ of transitions from $\net.T$ so that, for every $i \in \set{1,\ldots,n}$, we have that $\fire{\marking_{i-1}}{t_i}{\marking_{i}}{\net}$. We say that the firing sequence results in $\marking_{n}$.
%
given a marking $\marking$ %of $N$ 
and an enabled transition $t \in T_e$, the \emph{firing probability} of $t$ in $\marking$ is $\probt{t}{\marking}{\net} = \frac{W(t)}{\sum_{t'\in T_e}W(t')}$. %As required, 
The probabilities associated to all enabled transitions in a marking always add up to 1.
 A \emph{valid sequence} $\seq = t_1\cdots t_n$ is a firing sequence starting from $m_{in}^\net$ and resulting in $m_{out}$. The probability $\prob{\seq}{\net}$ of a valid sequence is the product of the probabilities associated to each transition: $\prob{\seq}{\net} = \prod_{i \in \set{1,\ldots,n}}\prob{t_i}{\marking_{i-1},\net}$.  A sequence of labels $\run = \alpha_1 \cdots \alpha_n$ from $\alphabet$ is a \emph{run} if there exists a valid underlying sequence $\seq = t_1\cdots t_n$  having $\alpha_i$ as a label for each $t_i\in \seq$. Run $\run$ may have different underlying valid sequences in $\seqs{\run}{\net}$. A trace $\trace$ is a \emph{model trace} (or $\net$-trace for short) if there exists an underlying run $\run$ corresponding to $\trace$ once all $\tau$ are removed. There may be multiple runs $\runs{\trace}{\net}$ underlying an $\net$-trace $\trace$.  $\traces{\net}$ is the (possibly infinite) set of $\net$-traces. For a trace $\trace$ of $\net$, its probability $\prob{\trace}{\net}$ is then obtained by collecting all its underlying runs, in turn collecting all their underlying valid sequences, and summing up their respective probabilities: $\prob{\trace}{\net} = \sum_{\run \in \runs{\trace}{\net}} \sum_{\seq \in \seqs{\run}{\net}} \prob{\seq}{\net}$. This corresponds to the intuition that, to observe $\trace$, one can equivalently pick any of its underlying valid sequences. Notably, if a trace is not an $\net$-trace (i.e., it does not conform with $\net$), then its probability is 0.
 
 These structural properties makes them suitable for loading BPMN nets as well: when the \texttt{Petri\_BPMN} is choosen, either the firing weight estimator is constant by default (\texttt{W\_CONSTANT}), or we can choose one from \cite{spdwe}. 
\begin{example} %\small
\label{ex:net}
\figurename~\ref{fig:spn} shows an example of an \uswn with input place $p_1$ and output place $p_7$. One run of the net is $\const{\tau c \tau a a \tau}$, which corresponds to trace $\const{caa}$. Overall, the net supports infinitely many finite traces of the form (represented using regular expressions):
\begin{inparaenum}[\it (i)]
\item $\const{aa^*}$,
\item $\const{cb}$,
\item $\const{caa^*}$.
\end{inparaenum}
\end{example}
%When executing an \uswn, the crucial addition to the standard execution semantics of Workflow nets is that, being the net stochastic, in each marking the set of enabled transitions gets associated to a discrete probability distribution. This is defined as follows: 
%given a marking $\marking$ of $N$ and an enabled transition $t \in \enaset{\marking}{\net}$, the \emph{firing probability} of $t$ in $\marking$ is $\probt{t}{\marking}{\net} = \frac{\net.W(t)}{\sum_{t'\in \enaset{\marking}{\net}}\net.W(t')}$. As required, the probabilities associated to all enabled transitions in a marking always add up to 1.
% %For a run $\run$ of $\net$, its probability $\prob{\seq}{\net}$ is then obtained by summing up the probabilities of all valid sequences corresponding to $\run$: $\prob{\run}{\net} = \sum_{\seq \in \seqs{\run}{\net}} \prob{\seq}{\net}$. Likewise, for a trace $\trace$ of $\net$, its probability is obtained by summing up
% For convenience, when needed, we represent an $\net$-trace as a pair $\tup{\trace,\prob{\trace}{\net}}$, where the probability assigned to $\trace$ by $\net$ is retained.


\textbf{Format: \texttt{StochasticMatrix}.} %\label{subsec:ppn}
The graph and trace embedding techniques %that we will use as the basis for computing probabilistic alignments 
cannot be directly defined over reachability graphs, as they %. In fact, these techniques 
rely on probabilistic \textsc{Transition Graphs} \cite{GartnerFW03}. Such graphs have transition probabilities associated to the edges, while nodes have labels in $\Sigma$, 
and suitably represented via transition matrices. Each node is mapped by a matrix $L$ to a single label, as the same label may be used for multiple nodes, while a matrix $R$ represents a probability distribution over the next nodes to be picked upon executing a transition.
%where edges are only labeled by probabilities, whereas labels are attached to nodes. In addition, towards readily enabling efficient algorithmic techniques, such graphs are compactly defined using transition matrixes. We therefore take inspiration from \cite{GartnerFW03} and introduce the so-called \emph{probabilistic transition graphs}, which we will later use to encode \uswn{s} via their reachability graphs.
Due to the lack of space, we refer to \cite{GartnerFW03} for the standard notation for \textsc{probabilistic Transition Graphs}:
%For a matrix $Q$ with row set $A$ and column set $B$, notation $[Q]_{ab}$ for $a \in A$ and $b \in B$ denotes the corresponding element in the matrix. In addition $\transp{Q}$ denotes the transposed matrix where rows and columns are inverted. We employ the usual sum and product operations over matrixes and arrays, and denote, for a square matrix $Q$, the repeated multiplication of $Q$ with itself $n$ times by $Q^n$.\todo{Rimuovere questo paragrafo se serve spazio,}\todo{NOn si capisce il significato di $\omega$}
%In our technical treatment, we continue to assume the existence of a set $\alphabet$ of labels (including the special label $\tau$).
%\begin{definition} A \emph{(Probabilistic) Transition Graph} is a tuple $(V,s,t,L,R)$ where:
%	\begin{inparaenum}[\itshape (i)]
%		\item $V \subset \mathbb{N}$ is a set of \emph{nodes};
%		\item $s\in V$ is the \emph{initial node};
%		\item $e\in V$ is the \emph{accepting node};
%		\item $L: \alphabet \times V \rightarrow \{0,1\}$ is a \emph{label matrix} associating each node in $V$ to a single label in $\alphabet$, where for label $\alpha \in \alphabet$ and node $\ind{i} \in V$, $[L]_{\alpha\ind{i}}$ gives $1$ if $\ind{i}$ is labeled by $\alpha$, $0$ otherwise;
%		\item $R: V \times V \rightarrow [0,1]$ is a \emph{(probabilistic) transition matrix} indicating, for each pair of nodes, what is the probability that executing a transition from the first node leads to the second node.
%		% \item $\omega \in [0,1]$ is a \emph{graph weight} indicating an overall value associated to the entire graph.
%	\end{inparaenum}
%	$L$ and $R$ satisfy the following well-formedness conditions:
%	\begin{inparaenum}[\itshape (i)]
%		\item for every $i \in V$ there is one and only one label $\alpha \in \alphabet$ so   that $[L]_{\alpha\ind{i}}=1$;
%		\item  for  every $\ind{i} \in V$, we have that $\sum_{\ind{j}\in V}[R]_{\ind{ij}}=1$.
%	\end{inparaenum}
%\end{definition}
%The condition for $L$ indicates that 
matrices $L$ and $R$ can be exploited to determine the probability of reaching a node labeled by $\beta\in\Sigma$ from any node labeled $\alpha\in\Sigma$ in $n$ steps with $[LR^n\transp{L}]_{\alpha\beta}/[L\transp{L}]_{\alpha\alpha}$, that we can shorthand as $[G.\Lambda^n]_{\alpha\beta}$ \cite{GartnerFW03}.
%
A transition graph $\tg$ can be visualized as shown in \figurename~\ref{fig:lmc}. Still, %The various elements have the obvious interpretation %, with the only important consideration 
%that 
an edge from node $\ind{i}$ to node $\ind{j}$ is only shown if the transition probability $[\tg.R]_{\ind{i}\ind{j}}$ is positive.
%There, each node $\ind{i} \in \tg.V$ is  represented as a circle with its identifying number. The initial node is decorated by a small incoming edge, while the final node is double circled. The label of the node is shown close to the circle, in agreement with $\tg.L$. Finally, an edge from $\ind{i} \in \tg.V$ to $\ind{j} \in \tg.V$ is shown if the transition probability $[\tg.R]_{\ind{i}\ind{j}}$ is positive. Each edge is decorated with the positive probability assigned by $\tg.R$.

%\begin{definition}[Path, trace]
%A \emph{path} in a transition graph $\tg$ is a finite sequence of nodes $\ind{i}_1 \cdots \ind{i}_n$ (with $n > 1$) such that, for every $j \in \set{1,\ldots,n-1}$, we have that $[\tg.R]_{\ind{i}_j\ind{i}_{j+1}} > 0$. Such a path is \emph{valid} if it starts from the initial node and ends in the accepting node of $\tg$, that is, $\ind{i}_1 = \tg.s$ and $\ind{i}_n = \tg.e$.
%
%A \emph{trace} is a finite sequence of nodes that can be turned into a valid sequence by introducing in the sequence an arbitrary number of $\tau$ labels (so as to account for hidden transitions in the graph).
%\end{definition}
%From the definition, it is clear that every valid path can be straightforwardly converted into a corresponding trace by removing all $\tau$ labels from the sequence.
%$\npath{\ind{i}}{\ind{j}}$
%By mirroring to definitions of \uswn{s} taking into account that now labels are on nodes, a \emph{valid sequence} of $\tg$ is a sequence $\ind{i}_0\ldots\ind{i}_n$ of nodes in $\tg.V$ that leads from the initial to the accepting node by only traversing transitions with nonzero probability. 
%\begin{inparaenum}[\it (i)]
%	\item $\ind{i}_0 = \tg.s$;
%	\item $\ind{i}_n = \tg.e$;
%	\item if the sequence contains at least two nodes, each two consecutive nodes are connected by a positive transition probability, i.e., for every $j \in \set{1,\ldots,n}$ we have $[R]_{\ind{i}_{j-1}\ind{i}_{j}} > 0$.
%\end{inparaenum}
Valid sequences, runs and model traces as well as their probability are defined similarly from \uswn{s}, and we employ the same notation.

%We close this section by introducing how some  matrix operations defined in the literature \cite{GartnerFW03} are applied to matrixes $L$ and $R$ of $\tg$, towards tackling interesting probability computations. These will be instrumental later on in the paper. Given two nodes $\ind{i},\ind{j} \in \tg.V$, $[R^n]_{\ind{i}\ind{j}}$ returns the probability of having a path in $\tg$ that connects $\ind{i}$ to $\ind{j}$ and has length $n$. Given two labels $\alpha,\beta \in \alphabet$, with $[LR^n\transp{L}]_{\alpha\beta}/[L\transp{L}]_{\alpha\alpha}$, we obtain the probability that, starting in any node labeled by $\alpha$, we reach a node labeled by $\beta$ through  $n$ consecutive steps in $\tg$. As a shortcut notation, we call the result $[\tg.\Lambda^n]_{\alpha\beta}$. Since there may be different nodes labeled by $\alpha$, we need to normalize the resulting probabilities. This is obtained with the division by $L\transp{L}$, which does so by assuming a uniform distribution when picking from which specific $\alpha$-labeled node one wants to start. Notice that these calculations need to be refined so as to consider proper runs and  traces. This will be done in Section~\ref{subsec:as}. %\todo{Rimandare alla sezione giusta}



%
%$\texttt{\color{blue}i}\overset{n}{\rightsquigarrow}\texttt{\color{blue}j}$ of length $n$: therefore, $[\Lambda^n]_{\color{green}\alpha\beta}:=[LR^nL^t]_{\color{green}\alpha\beta}/[LL^t]_{\color{green}\alpha\alpha}$ denotes the probability that, having started at any node labeled $\color{green}\alpha$ and taking $n$ steps, we arrive at any node labeled $\color{green}\beta$ (${\color{green}\alpha}\overset{n}{\rightsquigarrow}{\color{green}\beta}$). We denote as ${\color{green}\alpha}{\rightsquigarrow}{\color{green}\beta}$ an aforementioned path of arbitrary length.
%We can also associate a weight $\omega\in[0,1]\subseteq\mathbb{R}$ to a TG, so to express the probability associated with the TG itself as valid.
%




%\begin{example}
%We can graphically represent such TG as in \cite{Myers1989}.
%Figure \ref{fig:orig} is a  TG $P^*=(\mathtt{\color{blue}1},\mathtt{\color{blue}8},L,R,1)$ where $\omega=1$, where the matrices $L$ and $R$ can be both defined as follows:
%$$L:=\kbordermatrix{
%             & \texttt{\color{blue}1}&\texttt{\color{blue}2}&\texttt{\color{blue}3}&\texttt{\color{blue}4}&\texttt{\color{blue}5}&\texttt{\color{blue}6}&\texttt{\color{blue}7}&\texttt{\color{blue}8}&\texttt{\color{blue}9}&\texttt{\color{blue}10}\\
%\color{green}\varepsilon  & \textbf{1}&0&0&0&0&0&\textbf{1}&\textbf{1}&\textbf{1}&\textbf{1}\\
%\color{green}a            & 0&\textbf{1}&0&\textbf{1}&0&\textbf{1}&0&0&0&0\\
%\color{green}b            & 0&0&0&0&\textbf{1}&0&0&0&0&0\\
%\color{green}c            & 0&0&\textbf{1}&0&0&0&0&0&0&0\\
%}\qquad R:=\kbordermatrix{
%& \texttt{\color{blue}1}&\texttt{\color{blue}2}&\texttt{\color{blue}3}&\texttt{\color{blue}4}&\texttt{\color{blue}5}&\texttt{\color{blue}6}&\texttt{\color{blue}7}&\texttt{\color{blue}8}&\texttt{\color{blue}9}&\texttt{\color{blue}10}\\
%\texttt{\color{blue}1}  & 0&0&{\color{red}p_2}&0&0&0&0&0&{\color{red}p_1}&0\\
%\texttt{\color{blue}2}  & 0&0&0&0&0&{\color{red}p_3}&{\color{red}p_6}&0&0&0\\
%\texttt{\color{blue}3}  & 0&0&0&0&0&0&0&0&0&{\color{red}1}\\
%\texttt{\color{blue}4}  & 0&0&0&0&0&{\color{red}p_3}&{\color{red}p_6}&0&0&0\\
%\texttt{\color{blue}5}  & 0&0&0&0&0&0&0&{\color{red}1}&0&0\\
%\texttt{\color{blue}6}  & 0&0&0&0&0&{\color{red}p_3}&{\color{red}p_6}&0&0&0\\
%\texttt{\color{blue}5}  & 0&0&0&0&0&0&0&{\color{red}1}&0&0\\
%\texttt{\color{blue}8}  & 0&0&0&0&0&0&0&0&0&0\\
%\texttt{\color{blue}9}  & 0&{\color{red}1}&0&0&0&0&0&0&0&0\\
%\texttt{\color{blue}10}  & 0&0&0&{\color{red}p_4}&{\color{red}p_5}&0&0&0&0&0\\
%}$$
%\end{example}

% Given a TG $P=(s,t,L,R,\omega)$, a trace $\tau$ is a tuple in $(\Sigma\backslash\{\varepsilon\})^*$ denoting a path always originating from $s$ and terminating in $t$.
%\subsection{Input Transformation}
\medskip

\textbf{Transforming SWN$\to$TG.} If input is provided as a SWN $N$,  we internally represent all transition firings of an \uswn, together with their probabilities, in a reachability graph $RG(N)$ (\figurename~\ref{fig:rg}) by interpreting concurrency by interleaving. The transition probability function $P$ assigns to each transition an edge probability in $RG(N)$ as customary.  Due to the lack of space, we skip the customary definition of $RG(N)$.
%\begin{definition}
%The \emph{Reachability Graph} $\rg{\net}$ of \uswn \net is a triple $(M,E,P)$ where:
%\begin{compactitem}[$\bullet$]
%\item $M$ is the set of all reachable markings from $\marking_0^\net$ (including $\marking_0^\net$ itself).
%\item $E \subseteq M \times \alphabet \times M$ is a $\alphabet$-\emph{labeled transition relation} induced by $\net$, that is, for $\marking,\marking' \in M$, we have edge $(\marking,a,\marking') \in E$ if and only if there exists transition $t$  with label $\ell(t) = a$ and such that $\fire{\marking}{t}{\marking'}{\net}$.
%\item $P:E \rightarrow [0,1]$ is the \emph{transition probability} function assigning to each transition $(\marking,a,\marking') \in E$ its corresponding probability, obtained from the firing probability of the \uswn transition(s) that lead from $\marking$ to $\marking'$ and are labeled by $a$: $P(\marking,a,\marking') = \sum_{t_i \in \enaset{\marking}{\net} \text{ s.t.~} \net.\ell(t) = a \text{ and } \fire{\marking}{t}{\marking'}{\net}} \prob{t}{\marking}{\net}$.
%\end{compactitem}
%\end{definition}
%We now need to consider that, 
For a given state, distinct net transitions with the same label might produce the same consequent state, thus becoming indistinguishable from the trace perspective as they collaps into a single edge of the reachability graph. So, we accumulate all their firing probabilities into a single edge value. 
%Next, we handle %need to cope
%%Notice that, in the definition, 
%%we have to account for the possible case where, in a given state, 
%equivalently-labeled distinct net transitions from a given state % with the same label 
%producing the same consequent state: %. In this case, 
%as they are indistinguishable at the trace level by % when observing the execution traces of the net, and in fact they 
%collapsing into a single edge of the reachability graph, %. This is why, in this case, 
%we accumulate all their firing probabilities into a single value.
%
%In the remainder of the paper, 
Given an \uswn $\net$, we always assume that, in addition to its boundedness %, it the following structural assumptions that are natural in the BPM setting:
%\begin{compactitem}
%\item $\net$ is \emph{bounded}, that is, every marking in $\rg{\net}$ assigns at most a pre-defined number of tokens to each place;
%\item 
$\rg{\net}$ does not contain loops where all edges are labeled with $\tau$.
%\end{compactitem}
%The first assumption indicates that a case of the process does not generate unboundedly many parallel threads, and guarantees in turn that the reachability graph contains finitely many states. The second assumption naturally corresponds to how $\tau$-transitions are used when modeling business processes, where they are essential in representing gateways (such as exclusive and parallel splits/joins), cascaded gateways without tasks in between, and skippable tasks.  In all these cases, 
Multiple $\tau$-transitions may be used, but never creating completely invisible loops. Under this assumption, % $\net$ enjoys a very interesting property:
 there are only boundedly many valid sequences that can produce a given  trace $\trace$. Hence, the probability of $\trace$ can be computed by:
\begin{inparaenum}[\it (i)]
\item exhaustively enumerating all its valid sequences;
\item calculating the probability of each such sequence;
\item summing up all the so-obtained probabilities.
\end{inparaenum}
%\figurename~\ref{fig:rg} shows an example of a reachability graph.
\begin{example} %\small
  \label{ex:trace}
Consider the \uswn \net of Example~\ref{ex:net}. Considering trace $\const{caa}$, it is easy to see that it has only one underlying run, namely $\const{\tau c \tau a a \tau}$, in turn produced by a single underlying valid sequence, and that
%The firing probability of picking the first $\tau$-transition starting from the input marking is $1$, as there are no alternatives. In the new marking, where only one token is assigned to $p_2$, the firing probability of choosing the $\tau$-transition above is $\rho_{23} = \frac{v_{\tau_2}}{v_{\tau_2}+v_c}$, whereas that of choosing the $c$-transition below is $\rho_{24} = \frac{v_c}{v_{\tau_2}+v_c}$. Upon choosing the transition below, the new marking assigns only to $p_4$ one token, leaving just one choice to continue by moving that token to $p_6$. In that marking, the probability of choosing the $a$-transition above is $\rho_{65} = \frac{v_{a_3}}{v_{a_3}+v_b}$, resulting in the token being moved to $p_5$. In this new marking, the probability of iterating over the $a$-transition above is $\rho_{55} = \frac{v_{\tau_3}}{v_{\tau_3}+v_{a_2}}$, while that of completing in the output marking via the enabled $\tau$-transition is $\rho_{57} = \frac{v_{\tau_3}}{v_{\tau_3}+v_{a_2}}$. Hence, all in all
$\prob{\const{caa}}{\net} = 1 \cdot \rho_{24} \cdot 1 \cdot \rho_{65} \cdot \rho_{55} \cdot \rho_{57}$.
\end{example}



%Technically:
%\begin{compactitem}
%\item $P$ is a finite set of \textit{places}.
%\item $T$ is a finite set of \textit{transitions}, each of which is associate to a label. Each label either denotes a task executed upon transition firing, or indicates an invisible transition; in the latter case, we employ the special label $\varepsilon$.\footnotesize{This corresponds to the standard notion of $\tau$-transitions in Petri nets, but we use $\varepsilon$ since in the remainder of the paper $\tau$ is used to refer to an execution trace.}
%%to which we associate a label $\lambda(t)\in\Sigma$, where $\Sigma$ also includes the empty string\footnote{Given that we are going to denote the traces as $\tau$ and $t$ as the Petri net Transitions, we choose to denote the empty string as such instead of $\tau$ as in current literature from Petri nets.} $\varepsilon$.
%\item $F\subseteq (P\times T)\cup (T\times P)$ is the flow relation, representing arcs linking places to transitions and transitions to places.
%%to which we associate a \textit{firing cost} $\omega\colon F\to\mathbb{N}$.
%\item The initial place $i\in P$ has no ingoing edges ($\not\exists t\in T. (t,i)\in F$).
%\item The final place $f\in P$ has no outgoing edges ($\not\exists t\in T. (f,t)\in F$).
%\item $W\colon T\to \mathbb{R}^+_{>0}$ defines a \textit{firing weight} associated to each transition.
%\end{compactitem}

%A \textit{marking} is an assignment of a given amount of indistinguishable tokens to places described by a vector $M\colon P\to \mathbb{N}$. We say that a given transition $t$ is \textit{enabled} if $M(p)\geq 1$ for each ingoing $p$ to $t$ ($(p,t)\in F$). If such transition is enabled, then it can \textit{fire} a token. The \textit{enabling transitions} $E(M)$ for a given marking $M$ are all the $t$ reachable from $p$ ($(p,t)\in F$) with $M(p)\neq 0$ where $t$ is enabled. When $t$ can fire a token for a marking $M$, we can generate a novel marking $M'$ from $M$ by moving the tokens from the ingoing places towards the outgoing places as follows:
%\[\forall p\in P.\; M'(p)=M(p)-\mathbf{1}_{(p,t)\in F}+\mathbf{1}_{(t,p)\in F}\]
%We denote the transition from marking $M$ to marking $M'$ via an enabling $t$ as a relation $M\overset{t}{\to}M'$. We say that an \uswn with initial marking $M$ is $k$-\textit{bounded} if each of the markings $M'$ reachable from $M$, $M$ included, have $\forall p\in P.\; M(p)\leq k$\\

\begin{figure*}[!t]
	\begin{minipage}{.49\textwidth}
		\centering
		\includegraphics[width=.7\textwidth]{images/petri.pdf}
		\caption{A sample \uswn. Labels are shown in green, $\tau$ transitions in grey, weights in magenta.}\label{fig:spn}
	\end{minipage}\hfill \begin{minipage}{.49\textwidth}\centering
		\includegraphics[width=.8\textwidth]{images/rg.pdf}
		\caption{Reachability graph $RG(N)$ of the \uswn $N$. Probabilities are shown in violet.}\label{fig:rg}
	\end{minipage}
\end{figure*} \begin{figure*}[!t]
	\begin{minipage}{.49\textwidth}\centering \includegraphics[width=.55\textwidth]{images/running_example.pdf}
	\caption{Transition graph $G_{RG(N)}$ encoding the reachability graph $RG(N)$.}\label{fig:lmc}\label{fig:orig}
\end{minipage}\hfill \begin{minipage}{.49\textwidth}\centering \includegraphics[width=.55\textwidth]{images/closed_example.pdf}
	\caption{Transition graph $\closed{G_{RG(N)}}$ resulting from the transition graph in $G_{RG(N)}$ after $\tau$-closure.}\label{fig:closed}
\end{minipage}
\end{figure*}


%Figures \ref{fig:spn} and \ref{fig:rg} respectively show a sample \uswn and its corresponding reachability graph. This net will be our running example throughout the paper.


%\begin{example}
%Figure \ref{fig:spn} provides a sample \uswn defined as such, and \ref{fig:rg} provides its associated Reachability Graph. This representation can be beneficial when such \uswns are inferred and extracted from log files \cite{PPNFromLog} for extracting the set of the probabilistic traces associated to the \uswn.
%\end{example}
%
%
%We use \uswns for modelling business processes: in fact, it can be shown \cite{RaedtsPUWGS07} that it is always possible to convert BPMNs to \uswns. Last, we also assume that a transition is enabled when all of its input places contain at least one token and that, when a transition fires, we remove one token from each of its input places and depose tokens for each of its output places.

We can show that there exists a conversion from $\rg{\net}$ (\figurename~\ref{fig:rg}) into a transition graph $\tg_{\rg{\net}}$ (\figurename~\ref{fig:orig}) preserving model traces as well as their probabilities via well-known techniques used to \emph{shift labels} from automata theory.  Reachability Graphs can be directly provided as an input via \texttt{PetriMatrix} format.


\textbf{$\tau$-closure} The resulting transition graph $\tg_{\rg{\net}}$ is processed applying a \emph{$\hidden$-closure} that compiles away
$\tau$-transitions. This results into a new transition graph $\closed{\tg_{\rg{\net}}}$ (\figurename~\ref{fig:closed}) %that on the one hand only retains
retaining $\hidden$ labels only in the initial and accepting states while, for the rest, it exclusively operates over visible labels in $\tasks$. %, and, on the other, continues to preserve model traces and their probabilities. Also in this case we omit the details, as 
The transformation relies on well-known automata-based techniques for removing $\epsilon$-moves while preserving model traces, as well as their associated probabilities. %The only non-trivial observation is that, even in our case where probabilities are present, 
All $\tau$ transitions can still be removed thanks to the working hypothesis done for \uswn{s}, as no loops can contain only $\tau$ transitions. %The transition graph in \figurename~\ref{fig:orig} results in that of \figurename~\ref{fig:closed}.
\medskip

\textbf{Unfolding.} Next, we efficiently unfold the previously closed graph to collect TG-traces having probability greater or equal than $\pmin$ (\textsf{Minimum trace probability}) via the Eigen3 linear algebra library. Unfolding is efficiently performed via sparse matrices for TGs over the Eigen library. %The $\hidden$-closed transition graph $\closed{\tg_{\rg{\net}}}$ is \unravelled, so as to collect all the model traces that have a probability of at least $\pmin$. To do so, we rely on a key property that $\closed{\tg_{\rg{\net}}}$ inherits from the fact that it results from an \uswn. 
Since an \uswn has a Workflow net as underlying control-flow structure and given that TG inherits the results from SWNs, no loop can be executed without strictly decreasing the resulting probability. So, all valid sequences with a resulting probability of at least $\pmin$ can be enumerated and returned in a set. The so-obtained sequences are then combined by merging those that produce the same trace, summing up their probabilities, thus obtaining the set of all the traces having a probability greater than or equal to $\pmin$, $\ptraces{\closed{\tg_{\rg{\net}}}}{\pmin}$. The closure operation also implies that the notion of model trace collapses with the one of run modulo removing the initial and the final $\tau$ labels attached to the initial and accepting nodes. %In addition to that, traces might be also pre-filtered by \textsf{Maximum complete trace length}. 
\medskip

% !TeX root=../main.tex

%\subsection{Computation pipeline}
\section{Probabilistic Trace Alignment Pipeline}
%We now describe the proposed technique for computing probabilistic trace alignments. 
Our approach takes as input
\begin{inparaenum}[\it (i)]
	\item a reference model represented as Stochastic Workflow Nets $\net$ or an equivalent Transition Graph $G$,
	\item a minimum, positive probability threshold $\pmin \in (0,1]$
	\item a trace $\trace$ of interest,
\end{inparaenum}
and returns a ranking over all the $\net$-traces having a probability greater than or equal to $\pmin$, based on a combined consideration of their probability values and their distance to $\trace$. 

\begin{figure}[!t]
	\centering
	\includegraphics[width=.4\textwidth]{images/besser_rg}
	\caption{Transition Graph for \figurename~\ref{fig:petri_tut}.}
\end{figure}
\medskip
\noindent
\textbf{Transition Graphs} The trace probability can be directly compute by inspecting the reachability graph of the net. Still, graph embedding techniques required to represent traces as data points (e.g., vectors) 
cannot be directly defined over reachability graphs. as they %. In fact, these techniques 
rely on probabilistic \textsc{Transition Graphs} \cite{GartnerFW03}.  Such TGs can be computed by firstly shifting the transition labels over graph nodes, and performing $\tau$-closures, while preserving $\tau$s for both start and completion cases if required to preserve traces' probability. The resulting  graphs are described by transition probabilities in a matrix $R$, while nodes are associated to their labels via a matrix $L$, while the same label may be used for multiple nodes. $R$ is then a probability distribution over the next nodes to be picked upon executing a transition.
A special combination of matrices $L$ and $R$ \cite{GartnerFW03} describes the probability of reaching a node labeled by $\beta$ from any node labeled $\alpha$ in $n$ steps. Model traces  are defined similarly from \uswn{s}. This gives us a direct way to compute the probability of such a model trace over TGs:
\begin{enumerate}
	\item  we fetch all model runs of length at most $n+(n+1)\cdot b$ (which can be easily done with a bounded-depth search strategy over reachable markings);
	\item among all such runs, we keep all and only those that yield the model trace of interest;
	\item we sum up the probabilities of the so-filtered model runs.
\end{enumerate}



\textbf{Alignment Strategy}\label{subsec:as}
%Our technique is realized through the pipeline shown in \figurename~\ref{fig:pipe}, consisting of the following steps.
%
%In step 1, the reachability graph $\rg{\net}$ of $\net$ is constructed.
%
%In step 2,  
%$\closed{\tg_{\rg{\net}}}$ preserves model traces and their probabilities while working only over visible tasks in $\tasks$;
% this transition graph is unraveled c we apply a probabilistic alignment technique that ranks all such model traces according to their probability and their similarity to $\trace$.
%
%The pipeline has the following phases: after representing the USWN as a graph of all the sequentially scheduled transitions
%(\S\ref{sec:seqZ}), we shift the labels from the edges towards the nodes while preserving the set of probabilistic traces
%(\S\ref{sec:LSift}) and minimize the graph representation by removing the $\tau$-labeled nodes while preserving the
%trace probability (\S\ref{sec:clos}).
The last step takes the so-obtained TG-traces and ranks the best $k$ by considering their probabilities and the alignment cost with the trace $\trace$ of interest. %In \figurename~\ref{fig:pipe}, this is shown as a black-box. %As we describe in detail next, we have implemented this last step in two alternative ways: one computationally demanding but guaranteeing to produce an optimal ranking, the other more efficient but providing approximate ranking without optimality guarantees.
%\todo{Ho tagliato la descrizione che c'era dopo. Al limite la incorporerei nella rispettiva sottosezione.}
%We later discuss how to rank traces in the exact and approximated scenarios by reducing the alignment process to a k-nearest
%neighbour problem. While the exact trace alignment requires to perform the alignment process each time a novel trace $\sigma^*$ is
%introduced (\S\ref{subsec:exbkptap}), the approximated alignment can split the alignment into a preliminary loading phase and a
%query phase. In the former, each stochastic trace from the USWN is represented as a vector (\S\ref{subsec:ate}), and in the latter the to-be-aligned trace $\sigma^*$ is first represented as a vector and then compared to all the other vectorial representations.

\input{sections/03_embedding_2}


\bibliographystyle{splncs04}
\bibliography{biblio3}

%%
%%
%%
%\section{First Section}
%\subsection{A Subsection Sample}
%Please note that the first paragraph of a section or subsection is
%not indented. The first paragraph that follows a table, figure,
%equation etc. does not need an indent, either.
%
%Subsequent paragraphs, however, are indented.
%
%\subsubsection{Sample Heading (Third Level)} Only two levels of
%headings should be numbered. Lower level headings remain unnumbered;
%they are formatted as run-in headings.
%
%\paragraph{Sample Heading (Fourth Level)}
%The contribution should contain no more than four levels of
%headings. Table~\ref{tab1} gives a summary of all heading levels.
%
%\begin{table}
%\caption{Table captions should be placed above the
%tables.}\label{tab1}
%\begin{tabular}{|l|l|l|}
%\hline
%Heading level &  Example & Font size and style\\
%\hline
%Title (centered) &  {\Large\bfseries Lecture Notes} & 14 point, bold\\
%1st-level heading &  {\large\bfseries 1 Introduction} & 12 point, bold\\
%2nd-level heading & {\bfseries 2.1 Printing Area} & 10 point, bold\\
%3rd-level heading & {\bfseries Run-in Heading in Bold.} Text follows & 10 point, bold\\
%4th-level heading & {\itshape Lowest Level Heading.} Text follows & 10 point, italic\\
%\hline
%\end{tabular}
%\end{table}
%
%
%\noindent Displayed equations are centered and set on a separate
%line.
%\begin{equation}
%x + y = z
%\end{equation}
%Please try to avoid rasterized images for line-art diagrams and
%schemas. Whenever possible, use vector graphics instead (see
%Fig.~\ref{fig1}).
%
%\begin{figure}
%\includegraphics[width=\textwidth]{fig1.eps}
%\caption{A figure caption is always placed below the illustration.
%Please note that short captions are centered, while long ones are
%justified by the macro package automatically.} \label{fig1}
%\end{figure}
%
%\begin{theorem}
%This is a sample theorem. The run-in heading is set in bold, while
%the following text appears in italics. Definitions, lemmas,
%propositions, and corollaries are styled the same way.
%\end{theorem}
%%
%% the environments 'definition', 'lemma', 'proposition', 'corollary',
%% 'remark', and 'example' are defined in the LLNCS documentclass as well.
%%
%\begin{proof}
%Proofs, examples, and remarks have the initial word in italics,
%while the following text appears in normal font.
%\end{proof}
%For citations of references, we prefer the use of square brackets
%and consecutive numbers. Citations using labels or the author/year
%convention are also acceptable. The following bibliography provides
%a sample reference list with entries for journal
%articles~\cite{ref_article1}, an LNCS chapter~\cite{ref_lncs1}, a
%book~\cite{ref_book1}, proceedings without editors~\cite{ref_proc1},
%and a homepage~\cite{ref_url1}. Multiple citations are grouped
%\cite{ref_article1,ref_lncs1,ref_book1},
%\cite{ref_article1,ref_book1,ref_proc1,ref_url1}.
%%
%% ---- Bibliography ----
%%
%% BibTeX users should specify bibliography style 'splncs04'.
%% References will then be sorted and formatted in the correct style.
%%
%% \bibliographystyle{splncs04}
%% \bibliography{mybibliography}
%%


\end{document}
