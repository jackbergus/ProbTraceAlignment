% This is samplepaper.tex, a sample chapter demonstrating the
% LLNCS macro package for Springer Computer Science proceedings;
% Version 2.20 of 2017/10/04
%
\documentclass[runningheads]{llncs}
%
\usepackage{graphicx}
\usepackage{hyperref}
\usepackage{multicol}
\usepackage[final]{changes}
\usepackage{url}
\usepackage{amsmath}
\usepackage{bm}
\usepackage{multirow}
\usepackage{tcolorbox}
\usepackage{rotating}
\usepackage{latexsym,amssymb,amsmath}
\usepackage{makecell}
\usepackage{xspace}
\usepackage{paralist}
\usepackage{wrapfig}
\usepackage{adjustbox}
\usepackage{cite}
\usepackage{todonotes}
\input{neue_imports}
\input{neue_macros}
\newlist{myalist}{enumerate*}{1}
\setlist[myalist]{label=\textbf{(\arabic*)}}
\newlist{mylist}{enumerate*}{1}
\setlist[mylist]{label=\textit{(\roman*)}}
\newlist{alphalist}{enumerate*}{1}
\setlist[alphalist]{label=\textbf{(\alph*)}}
\renewcommand*{\UrlFont}{\ttfamily\smaller\relax}
\input{macros-common}
\usepackage{etoolbox}
\usepackage{hyperref}
\sloppy

\begin{document}
%
\title{Probabilistic Trace Alignments\\ as a kNN Problem}

\author{Giacomo Bergami\inst{1}\orcidID{0000-0002-1844-0851} \and
%Fabrizio Maria Maggi\inst{1}\orcidID{0000-0002-9089-6896} \and
Marco Montali\inst{1}\orcidID{0000-0002-8021-3430} \and
Rafael Pe\~naloza\inst{2}\orcidID{0000-0002-2693-5790}}
%
\authorrunning{G.~Bergami, M.~Montali and R.~Pe\~naloza}
% First names are abbreviated in the running head.
% If there are more than two authors, 'et al.' is used.
%
\institute{	Free University of Bozen-Bolzano, Italy \\\email{gibergami@unibz.it,\{maggi,montali\}@inf.unibz.it}
	\and
	University of Milano-Bicocca \\\email{rafael.penaloza@unimib.it}}
%
\maketitle              % typeset the header of the contribution
%
\begin{abstract}
Alignments provide sophisticated diagnostics that pinpoint deviations in a trace with respect to a process model and their severity. However, approaches based on trace alignments use crisp process models as reference and recent probabilistic conformance checking approaches check the degree of conformance of an event log with respect to a stochastic process model instead of finding trace alignments. In this paper, for the first time, we provide a conformance checking approach based on trace alignments using stochastic Workflow nets. Conceptually, this requires to handle the two possibly contrasting forces of the cost of the alignment on the one hand and the likelihood of the model trace with respect to which the alignment is computed on the other.

\keywords{Stochastic Petri nets \and Conformance Checking \and Alignments.}
\end{abstract}

\section{Introduction}

\texttt{[TODO]}

In order to assess our proposed approach, we perform the following experiments for empirically evaluating some of the properties of the proposed embedding strategy:
\begin{enumerate}
	\item First, we assess the degree of approximation introduced by the approximate-ranking approach if compared with the optimal-ranking. We observe that different embedding strategies provide a trade-off between ranking stability vs. precision (\S\ref{subsec:apprp}).
	\item Next, (\S\ref{subsec:gen}) \texttt{[Generalization: TODO]}
	\item Last, we evaluate the computational time required to both generate the embeddings and to assess the similarity between the embeddings. We observe that (\S\ref{subsec:efficio}).
\end{enumerate}
\section{Related Works}
%%%
%%%
\paragraph*{Stochastic Conformance Checking.} Earlier works on probabilistic trace alignments \cite{AlizadehLZ14a} extended alignment cost functions by considering probabilities that an activity never eventually occurs when activities on both traces do not correspond. Otherwise, such functions always return zero once activity match. This approach favors traces providing optimal alignment: this approach is not ideal for ranking traces, as the resulting ranking provides no trade-off between trace probability and alignment cost. Furthermore, the computation of the aforementioned probability requires the combined provision of a log file and a non-stochastic Petri Net, our proposed solution estimates trace probability by directly assessing Stochastic Petri Nets. After considering that the formers could be always used to estimate the latter \cite{spdwe}, we can deduce that our proposed solution proves to be more general that this first attempt to probabilistic trace alignments. 

More recent works on stochastic model checking \cite{DBLP:conf/icpm/PolyvyanyyK19,DBLP:journals/tosem/PolyvyanyySWCM20} assess the degree of conformance of the whole stochastic model against one single log trace, thus considering the probability distribution of the whole model. On the other hand, we are interested in determining which is the best model trace providing the trade-off between trace probability and alignment cost with the log trace to be aligned, thus limiting the model probability distribution to the part generating one sole given trace at a time. Therefore, such approaches allow to rank models according to the conformance of a give log trace while, on the other hand, our proposed solutions ranks model traces according to one fixed stochastic model. 

 
%%%
%%%
\paragraph*{Graph Kernels.} Graph kernels express similarity measures \cite{Samatova} involved in both classification \cite{TsudaS10} and clustering algorithms. One of the first approaches required a preliminary embedding definition of topological description vectors extracted from the most frequent subgraphs within a graph database \cite{Sidere}. As a drawback, it required the computation of a subgraph isomorphism problem, which is NP-complete. In fact, the definition of a graph kernel function fully recognizing the structure the graph always boils down to solving such NP-Complete problem \cite{GartnerFW03}, as exact embeddings generable in polynomial can be inferred just for loop-free Direct Acyclic Graphs \cite{BergamiBM20}. Consequently, most recent literature focused on extracting relevant features of such graphs, that are then used to define a graph similarity function. The most common approach adopted in the kernel to extract such features is called \textit{propositionalization}: we might extract all the possible features (e.g., subsequences), and then define a kernel function based on the occurrence and similarity of these features \cite{Gartner03}. For node-labelled graphs, the features come from the node labels and the possible strings that might be generated while traversing the graph (see \cite{Gartner03} and \S\ref{subsec:katk}). 
\section{Preliminary Definitions}
\subsection{Stochastic Petri Nets}\label{subsec:spn}
%As customary in probabilistic conformance checking \cite{DBLP:conf/bpm/LeemansSA19,DBLP:conf/icpm/PolyvyanyyK19,DBLP:journals/tosem/PolyvyanyySWCM20}, we adopt stochastic Petri nets \cite{MarsanCB84,RoggeSoltiAW13} as the underlying formal basis to represent processes. More specifically, we consider an interesting class of stochastic Petri nets with only immediate transitions (i.e., no timed ones).
%We assume to have a set $\alphabet = \tasks \cup \set{\tau}$ of labels, where labels in $\tasks$ indicate process tasks, whereas $\tau$ indicates an invisible execution step ($\tau$-transition). A \emph{trace} is a finite sequence of labels from $\tasks$. An \emph{untimed Stochastic Workflow Net (\uswn)}
%is a tuple $\net = (P,T,F,\ell,W)$ where:
%\textit{(i)} $(P,T,F)$ is a standard \emph{Workflow net} with places $P$, transitions $T$, and flow relation $F$ such that there is exactly one \emph{input place} with no incoming arc, and exactly one \emph{output place} with no outgoing arcs;
%\textit{(ii)} $\ell: T \rightarrow \alphabet$ is a \emph{labeling function} mapping each transition $t \in T$ into a label $\ell(t) \in \alphabet$;
%\textit{(iii)} $W\colon T\to \mathbb{R}^+$ is a \emph{weight function} assigning a positive firing weight to each transition.
%
%\texttt{[TODO: describe the notion of marking]}
%
%By interpreting concurrency by interleaving, we can represent all transition firings of an \uswn, together with their probabilities, in a reachability graph  $(M,E,P)$ where
%\textit{(i)} $M$ is the set of all reachable markings from the initial markings,
%\textit{(ii)} $E \subseteq M \times \alphabet \times M$ is a $\alphabet$-\emph{labeled transition relation} induced by $\net$, that is, for $\marking,\marking' \in M$, we have edge $(\marking,a,\marking') \in E$ if and only if there exists transition $t$ in $\net$ with label $\ell(t) = a$ and such that $\fire{\marking}{t}{\marking'}{\net}$, and
%\textit{(iii)} $P:E \rightarrow [0,1]$ is the \emph{transition probability} function assigning to each transition $(\marking,a,\marking') \in E$ its corresponding probability, obtained from the firing probability of the \uswn transition(s) that lead from $\marking$ to $\marking'$ and are labeled by $a$. 
In this paper, we consider a specific case of Stochastic Petri Nets, such as Workflow Networks over $k$-bounded Place-Transitions Nets \cite{MarsanCB84,Desel1998,RoggeSoltiAW13}. We assume to have a set $\alphabet = \tasks \cup \set{\tau}$ of labels, where labels in $\tasks$ indicate process tasks, whereas $\tau$ indicates an invisible execution step ($\tau$-transition). %Labels are associated to transitions via a labelling function $\lambda$. 
 Those Stochastic Petri Nets can be modelled as a tuple $\net=(P,T,F,W,i,f)$ where:
\begin{mylist}
	\item $P$ is a set of \textit{places} to which we can associate a finite number of indistinguishable tokens;
	\item $T$ is a set of \textit{transitions} $t\in T$, to which we associate a label $\lambda(t)\in\Sigma$;
	\item $F\subseteq (P\times T)\cup (T\times P)$ is a set of arcs; % to which we associate a \textit{firing cost} $\omega\colon F\to\mathbb{N}$;
	\item $W\colon T\to \mathbb{R}$ defines a \textit{firing weight} associated to each transition;
	\item the initial place $i\in P$ has no ingoing edges; %($\not\exists t\in T. (t,i)\in F$);
	\item the final place $f\in P$ has no outgoing edges. %($\not\exists t\in T. (f,t)\in F$).
\end{mylist}
A \textit{marking} is an assignment of a given amount of indistinguishable tokens to places described by a vector $M\colon P\to \mathbb{N}$. We say that a given transition $t$ is \textit{enabled} if $M(p)\geq \omega(p,t)$ for each ingoing $p$ to $t$ ($(p,t)\in F$). If such transition is enabled, then it can \textit{fire} a token. The set of the \textit{enabling transitions} $E(M)$ for a given marking $M$ are all the $t$ reachable from $p$ ($(p,t)\in F$) with $M(p)\neq 0$ where $t$ is enabled. When $t$ can fire a token for a marking $M$, we can generate a novel marking $M'$ from $M$ by moving the tokens from the ingoing places towards the outgoing places as 
$\forall p\in P.\; M'(p)=M(p)-[\omega(p,t)]+[\omega(t,p)]$. 
We denote the transition from marking $M$ to marking $M'$ via an enabling $t$ as a relation $M\overset{t}{\to}M'$. Given an initial marking $M$ for a Stochastic Petri Net $SPN$,  the \textit{Reachability Graph} for $SPN$ is a graph $(\mathcal{M},\mathcal{E})$ where the nodes  $\mathcal{M}$ are composed of of all the reachable markings from $M$, $M$ included, and all the edges $\mathcal{E}$ are induced by the aforementioned relation $M\overset{t}{\to}M'$ among the reachability graph's nodes. To each of such edges $M\overset{t}{\to}M'$, we associate a transition probability $\mathbb{P}\left(M\overset{t}{\to}M'\right)=\frac{W(t)}{\sum_{t'\in E(M)}W(t')}$ \cite{spdwe}. We say that a SPN with initial marking $M$ is $k$-\textit{bounded} if each of the nodes $M$ in the reachability graph have $\forall p\in P.\; M(p)\leq k$. 


A \emph{trace} is a finite sequence of labels from $\tasks$. 
%\begin{definition} An \emph{untimed Stochastic Workflow Net (\uswn)}
%	is a tuple $\net = (P,T,F,\ell,W)$ where:
%	\begin{inparaenum}[\itshape (i)]
%		%\begin{inparaenum}
%		\item $(P,T,F)$ is a standard \emph{Workflow net} with places $P$, transitions $T$, and flow relation $F$ such that there is exactly one \emph{input place} with no incoming arc, and exactly one \emph{output place} with no outgoing arcs;
%		\item $\ell: T \rightarrow \alphabet$ is a \emph{labeling function} mapping each transition $t \in T$ into a label $\ell(t) \in \alphabet$ - this either indicates the task executed upon firing $t$, or the fact that $t$ is an invisible transition (in the latter case, $\ell(t) = \tau$);
%		\item $W\colon T\to \mathbb{R}^+$ is a \emph{weight function} assigning a positive firing weight to each transition of the net.
%	\end{inparaenum}
%\end{definition}
%Given an \uswn $\net$, we use dot notation to extract its constitutive components (e.g., $\net.P$ denotes its places). \emph{The same dot notation will be used for the other structures introduced in the paper}. We also use $\net.in$ and $\net.out$ to respectively denote the input and output place of $\net$.
The current state of execution is captured by a marking, i.e., a multiset of places $P$ indicating how many tokens populate each place.
%As pointed out above, \emph{we always assume, as customary in BPM, that the input \uswn is \underline{bounded}}, that is, in every state the number of tokens associated to each place cannot exceed a maximum, fixed threshold.
%The notions of transition enablement and firing are also the standard ones  \cite{MarsanCB84}, as well as the ones for \textit{firing probability} \cite{spdwe}. %: %, which provides the basis for capturing the stochastic behavior of the net. 
%We use the following notation: given a marking $\marking$ over \uswn $\net$,  $\enaset{\marking}{\net}$ is the set of enabled transitions in $\marking$; given transition $t \in \enaset{\marking}{\net}$, we write $\fire{\marking}{t}{\marking'}{\net}$ to capture the fact that, with, firing $t$ in $\marking$ results in the new marking $\marking'$. A \emph{firing sequence starting from marking $\marking_0$} is a sequence $t_1\cdots t_n$ of transitions from $\net.T$ so that, for every $i \in \set{1,\ldots,n}$, we have that $\fire{\marking_{i-1}}{t_i}{\marking_{i}}{\net}$. We say that the firing sequence results in $\marking_{n}$.
%
%given a marking $\marking$ %of $N$ 
%and an enabled transition $t \in T_e$, the \emph{firing probability} of $t$ in $\marking$ is $\probt{t}{\marking}{\net} = \frac{W(t)}{\sum_{t'\in T_e}W(t')}$. %As required, 
The probabilities associated to all enabled transitions in a marking always add up to 1.
A \emph{valid sequence} $\seq = t_1\cdots t_n$ is a firing sequence starting from $m_{in}$ and resulting in $m_{out}$. The probability $\prob{\seq}{\net}$ of a valid sequence is the product of the probabilities associated to each transition. %: $\prob{\seq}{\net} = \prod_{i \in \set{1,\ldots,n}}\prob{t_i}{\marking_{i-1},\net}$. % A sequence of labels $\run = \alpha_1 \cdots \alpha_n$ from $\alphabet$ is a \emph{run} if there exists a valid underlying sequence $\seq = t_1\cdots t_n$  having $\alpha_i$ as a label for each $t_i\in \seq$. Run $\run$ may have different underlying valid sequences in $\seqs{\run}{\net}$.
A trace $\nonlogtrace=\const{a}_1\cdots \const{a}_m$ is a \emph{model trace} (or $\net$-trace for short) if there exists a valid sequence $\seq = t_1\cdots t_n$ where the appended labels $\lambda(t_1)\cdots \lambda(t_n)$ is equivalent to $\trace$ once all the $\tau$-s are stripped.
%underlying run $\run$ corresponding to $\trace$ once all $\tau$ are removed. 
There may be multiple valide sequences $\seq\in\seqs{\nonlogtrace}{\net}$ %$\runs{\trace}{\net}$ 
underlying an $\net$-trace $\trace$. 
$\traces{\net}$ is the (possibly infinite) set of $\net$-traces. For a trace $\nonlogtrace$ of $\net$, its probability $\prob{\nonlogtrace}{\net}$ is then obtained by collecting all its underlying valid sequences, %, in turn collecting all their underlying valid sequences, 
and summing up their respective probabilities. %: $\prob{\trace}{\net} = %\sum_{\run \in \runs{\trace}{\net}} 
%\sum_{\seq \in \seqs{\trace}{\net}} \prob{\seq}{\net}$. 
This corresponds to the intuition that, to observe $\nonlogtrace$, one can equivalently pick any of its underlying valid sequences. Notably, if a trace is not an $\nonlogtrace$-trace (i.e., it does not conform with $\net$), then its probability is 0. Given a log trace $\logtrace$ coming from a log file $\mathcal{L}$, we always associate to it a certain probability, i.e., $\prob{\logtrace}{\mathcal{L}}=1$. 

\subsection{Graph and String Kernels}\label{subsec:katk}
 As a foundational basis to compute trace alignments, we adapt similarity measures from the database literature.  Given a set of data examples $\mathcal{X}$, (e.g., strings or traces, transition graphs) a (positive definite) \emph{kernel} function $k\colon \mathcal{X}\times \mathcal{X}\to \mathbb{R}$ denotes the similarity of elements in $\mathcal{X}$. If $\mathcal{X}$ is the $d$-dimensional Euclidean Space $\mathbb{R}^d$, the simplest kernel function is the inner product $\Braket{\mathbf{x},\mathbf{x}'}=\sum_{1\leq i\leq d}\mathbf{x}_i\mathbf{x}'_i$.
A kernel is said to \emph{perform ideally} \cite{Gartner03} when $k(x,x')=1$ whenever $x$ and $x'$ are the same object (\textit{strong equality}) and $k(x,x')=0$ whenever $x$ and $x'$ are distinct objects (\textit{strong dissimilarity}). A kernel is also said to be \emph{appropriate} when similar elements $x,x'\in\mathcal{X}$ are also close in the feature space. Notice that appropriateness can be only assessed  empirically \cite{Gartner03}.
A positive definite kernel induces a distance metric as 
$
d_k(\mathbf{x},\mathbf{x}'):=\sqrt{k(\mathbf{x},\mathbf{x})-2k(\mathbf{x},\mathbf{x}')+k(\mathbf{x}',\mathbf{x}')}
$.
When the kernel of choice is the inner product, the resulting distance is the Euclidean distance $\norm{\mathbf{x}-\mathbf{x}'}{2}$. A normalized vector $\hat{\mathbf{x}}$ is defined as $\mathbf{x}/\norm{\mathbf{x}}{2}$. For a normalized vector we can easily prove that: $\norm{\hat{\mathbf{x}}-\hat{\mathbf{x}}'}{2}^2=2(1-\Braket{\hat{\mathbf{x}},\hat{\mathbf{x}}'})$.
When $\mathcal{X}$ does not represent directly a $d$-dimensional Euclidean space, we can use an \emph{embedding} $\embed\colon\mathcal{X}\to \mathbb{R}^d$ to define a kernel $k_\embed\colon \mathcal{X}\times \mathcal{X}\to\mathbb{R}$ as $k_\embed(x,x'):=\Braket{\embed(x),\embed(x')}$. As a result, $k_\embed(x,x')=k_\embed(x',x)$ for each $x,x'\in\mathcal{X}$.



 The literature also provides a kernel representation for strings \cite{LodhiSSCW02,GartnerFW03}: if we associate each dimension in $\mathbb{R}^d$ to a different sub-string $\alpha\beta$ of size $2$ (i.e., $2$-grams\footnote{\label{fn:caveat}For our experiments, we choose to consider only $2$-grams, but any $p$-grams of arbitrary length $p\geq 2$ might be adopted \cite{Gartner03}. An increased size of $p$ improves precision but also incurs in a worse computational complexity, as it requires to consider all the arbitrary subtraces of length $p$ whose constitutive elements occur at any distance from each other within the trace.}), it should represent how frequently and ``compactly'' this subtrace is embedded in the trace $\trace$ of interest. Therefore, we introduce a \emph{decay factor} $\lambda\in[0,1]\subseteq\mathbb{R}$ that, for all $m$ sub-strings where $\alpha$ and $\beta$ appear in $\trace$ at the same relative distance $l < |\trace|$, weights the resulting embedding as $\lambda^lm$.



\begin{table*}[t!]
	\vspace{+0.7cm}
	\caption{Embedding of traces $\const{caba}$, $\const{caa}$ and $\const{cb}$.}\label{tb:embedding}
	\vspace{-0.4cm}
	\begin{center}
		%\scalebox{0.6}
		{
			\begin{tabularx}{\textwidth}{
					>{\hsize=.1\hsize}X
					>{\hsize=.2\hsize}X
					>{\hsize=.1\hsize}X
					>{\hsize=.1\hsize}X
					>{\hsize=.1\hsize}X
					>{\hsize=.1\hsize}X
					>{\hsize=.1\hsize}X
					>{\hsize=.25\hsize}X
					>{\hsize=.2\hsize}X
					>{\hsize=.1\hsize}X
				}
				\toprule
				& $\const{aa}$    & $\const{ab}$   & $\const{ac}$    & $\const{ba}$   & $\const{bb}$   & $\const{bc}$ & $\const{ca}$ & $\const{cb}$ & $\const{cc}$   \\
				\midrule
				$\const{caba}$ & $\lambda^2$ & $\lambda$ & $0$ & $\lambda$  & $0$  & $0$ & $\lambda+\lambda^3$ & $\lambda^2$ & $0$\\
				%$\const{caaa}$ & $2\lambda+\lambda^2$& $0$ & $0$ & $0$ & $0$ & $0$ & $\lambda+\lambda^2+\lambda^3$ & $0$ & $0$ \\
				$\const{caa}$  & $\lambda$ & $0$ & $0$ & $0$ & $0$ & $0$ & $\lambda+\lambda^2$ & $0$&  $0$\\
				$\const{cb}$   & $0$ & $0$ & $0$ & $0$ & $0$ & $0$ & $0$ & $\lambda$& $0$ \\
				\bottomrule
			\end{tabularx}
		}
		\vspace{-0.3cm}
	\end{center}
\end{table*}
\begin{example}\label{ex:wheredotiszero} %\small
	Consider tasks $\tasks=\Set{a,b,c}$. The possible 2-grams over $\tasks$ are $\tasks^2=\Set{\const{aa},\const{ab},\const{ac},\const{ba},\const{bb},\const{bc},\const{ca},\const{cb},\const{cc}}$. Table~\ref{tb:embedding} shows the embeddings of some traces. Being a 2-gram, trace $\const{cb}$ has only one nonzero component, namely that corresponding to itself, with $\trembed_{\const{cb}}(\const{cb})=\lambda$. Trace $\const{caa}$ has the 2-gram $\const{ca}$ occurring with length $1$ ($\const{\underline{ca}a}$) and $2$ ($\const{\underline{c}a\underline{a}}$), and the 2-gram $\const{aa}$ with occurring length $1$ ($\const{c\underline{aa}}$). Hence: $\trembed_{\const{ca}}(\const{caa})=\lambda+\lambda^2$ and  $\trembed_{\const{aa}}(\const{caa})=\lambda$.  Similar considerations can be carried out for the other traces in the table.
	We now want to compute the similarity between the first trace $\const{caba}$ and the other two traces. To do so, we sum, column by column (that is, 2-gram by 2-gram) the product of the embeddings for each pair of traces. We then get $k_{\trembed}(\const{caba},\const{caa})=\lambda^3+(\lambda+\lambda^3)(\lambda+\lambda^2)$ and $k_{\trembed}(\const{caba},\const{cb})=\lambda^3
	$,
	%{\footnotesize
	%\[
	%k_{\trembed}(\const{caba},\const{caaa})=\lambda(\lambda+\lambda^2+\lambda^3)
	%~~
	%k_{\trembed}(\const{caba},\const{caa})=\lambda(\lambda+\lambda^2)
	%~~
	%k_{\trembed}(\const{caba},\const{cb})=\lambda(\lambda+\lambda^3)
	%\]}
	which induces ranking $
	k_{\trembed}(\const{caba},\const{caa})>
	k_{\trembed}(\const{caba},\const{cb})
	$.
\end{example}

Nevertheless, such string embedding has several shortcomings: \begin{alphalist}
	\item it is not weakly-ideal, so we cannot numerically assess if two embeddings represent equivalent traces 
	(Example \ref{ex:wheredotiszero});
	\item it does not characterize $\tau$-moves, so the probabilities of the initial and final $\tau$-moves are not preserved; and
	\item it is affected by numerical errors from finite arithmetic: longer traces $\nonlogtrace$ generated from skewed probability 
	distributions $G.\Lambda^i$ yield greater truncation errors, as smaller $\lambda^i$ components for bigger 
	$i<|\nonlogtrace|$ are ignored, preventing a complete numerical vector characterization of  $\nonlogtrace$ in practice.
\end{alphalist}

\endinput





\section{Working Assumptions}
Petri Nets and Generalized Stochastic Petri Nets are well-established formalisms \cite{DBLP:journals/tosem/PolyvyanyySWCM20} for modelling processes \cite{RoggeSoltiAW13} represented in the Petri Net Markup Language, supported by our tool. Due to the lack of space, we refer to \cite{spdwe} for the usual notation over Petri Nets. We restrict our interest to an interesting class of $1$-\textit{bounded} stochastic Petri nets with no timed transitions, namely \textsc{untimed Stochastic Workflow Nets} denoted as $N$. We now sketch the properties of the SWN sketched in \cite{Bergami21}. We consider a single \emph{input} (\emph{output}) marking $m_{in}$ ($m_{out}$) assigning a single token to the input (output) place, and no token elsewhere. We assume to have a set $\alphabet = \tasks \cup \set{\tau}$ of labels, where labels in $\tasks$ indicate process tasks, whereas $\tau$ indicates an invisible execution step ($\tau$-transition). Labels are associated to transitions via a labelling function $\lambda$. A \emph{trace} is a finite sequence of labels from $\tasks$.
%\begin{definition} An \emph{untimed Stochastic Workflow Net (\uswn)}
%	is a tuple $\net = (P,T,F,\ell,W)$ where:
%	\begin{inparaenum}[\itshape (i)]
%		%\begin{inparaenum}
%		\item $(P,T,F)$ is a standard \emph{Workflow net} with places $P$, transitions $T$, and flow relation $F$ such that there is exactly one \emph{input place} with no incoming arc, and exactly one \emph{output place} with no outgoing arcs;
%		\item $\ell: T \rightarrow \alphabet$ is a \emph{labeling function} mapping each transition $t \in T$ into a label $\ell(t) \in \alphabet$ - this either indicates the task executed upon firing $t$, or the fact that $t$ is an invisible transition (in the latter case, $\ell(t) = \tau$);
%		\item $W\colon T\to \mathbb{R}^+$ is a \emph{weight function} assigning a positive firing weight to each transition of the net.
%	\end{inparaenum}
%\end{definition}
%Given an \uswn $\net$, we use dot notation to extract its constitutive components (e.g., $\net.P$ denotes its places). \emph{The same dot notation will be used for the other structures introduced in the paper}. We also use $\net.in$ and $\net.out$ to respectively denote the input and output place of $\net$.
The current state of execution is captured by a marking, i.e., a multiset of places $P$ indicating how many tokens populate each place.
%As pointed out above, \emph{we always assume, as customary in BPM, that the input \uswn is \underline{bounded}}, that is, in every state the number of tokens associated to each place cannot exceed a maximum, fixed threshold.
The notions of transition enablement and firing are also the standard ones  \cite{MarsanCB84}, as well as the ones for \textit{firing probability} \cite{spdwe}. %: %, which provides the basis for capturing the stochastic behavior of the net. 
%We use the following notation: given a marking $\marking$ over \uswn $\net$,  $\enaset{\marking}{\net}$ is the set of enabled transitions in $\marking$; given transition $t \in \enaset{\marking}{\net}$, we write $\fire{\marking}{t}{\marking'}{\net}$ to capture the fact that, with, firing $t$ in $\marking$ results in the new marking $\marking'$. A \emph{firing sequence starting from marking $\marking_0$} is a sequence $t_1\cdots t_n$ of transitions from $\net.T$ so that, for every $i \in \set{1,\ldots,n}$, we have that $\fire{\marking_{i-1}}{t_i}{\marking_{i}}{\net}$. We say that the firing sequence results in $\marking_{n}$.
%
%given a marking $\marking$ %of $N$ 
%and an enabled transition $t \in T_e$, the \emph{firing probability} of $t$ in $\marking$ is $\probt{t}{\marking}{\net} = \frac{W(t)}{\sum_{t'\in T_e}W(t')}$. %As required, 
The probabilities associated to all enabled transitions in a marking always add up to 1.
A \emph{valid sequence} $\seq = t_1\cdots t_n$ is a firing sequence starting from $m_{in}$ and resulting in $m_{out}$. The probability $\prob{\seq}{\net}$ of a valid sequence is the product of the probabilities associated to each transition.%: $\prob{\seq}{\net} = \prod_{i \in \set{1,\ldots,n}}\prob{t_i}{\marking_{i-1},\net}$. % A sequence of labels $\run = \alpha_1 \cdots \alpha_n$ from $\alphabet$ is a \emph{run} if there exists a valid underlying sequence $\seq = t_1\cdots t_n$  having $\alpha_i$ as a label for each $t_i\in \seq$. Run $\run$ may have different underlying valid sequences in $\seqs{\run}{\net}$.
A trace $\trace=\const{a}_1\cdots \const{a}_m$ is a \emph{model trace} (or $\net$-trace for short) if there exists a valid sequence $\seq = t_1\cdots t_n$ where the appended labels $\lambda(t_1)\cdots \lambda(t_n)$ is equivalent to $\trace$ once all the $\tau$-s are stripped.
%underlying run $\run$ corresponding to $\trace$ once all $\tau$ are removed. 
There may be multiple valide sequences $\seq\in\seqs{\trace}{\net}$ %$\runs{\trace}{\net}$ 
underlying an $\net$-trace $\trace$. 
$\traces{\net}$ is the (possibly infinite) set of $\net$-traces. For a trace $\trace$ of $\net$, its probability $\prob{\trace}{\net}$ is then obtained by collecting all its underlying runs, in turn collecting all their underlying valid sequences, and summing up their respective probabilities. %: $\prob{\trace}{\net} = %\sum_{\run \in \runs{\trace}{\net}} 
%\sum_{\seq \in \seqs{\trace}{\net}} \prob{\seq}{\net}$. 
This corresponds to the intuition that, to observe $\trace$, one can equivalently pick any of its underlying valid sequences. Notably, if a trace is not an $\net$-trace (i.e., it does not conform with $\net$), then its probability is 0.
These structural properties makes them suitable for converting BPMN models with no swim-lanes as SWN, which firing weight could be estimated from their associated log files \cite{spdwe}. 
\begin{example} %\small
	\label{ex:net}
	\figurename~\ref{fig:spn} shows an example of an \uswn with input place $p_1$ and output place $p_7$. One run of the net is $\const{\tau c \tau a a \tau}$, which corresponds to trace $\const{caa}$. Overall, the net supports infinitely many finite traces of the form (represented using regular expressions):
	\begin{inparaenum}[\it (i)]
		\item $\const{aa^*}$,
		\item $\const{cb}$,
		\item $\const{caa^*}$.
	\end{inparaenum}
\end{example}
%When executing an \uswn, the crucial addition to the standard execution semantics of Workflow nets is that, being the net stochastic, in each marking the set of enabled transitions gets associated to a discrete probability distribution. This is defined as follows: 
%given a marking $\marking$ of $N$ and an enabled transition $t \in \enaset{\marking}{\net}$, the \emph{firing probability} of $t$ in $\marking$ is $\probt{t}{\marking}{\net} = \frac{\net.W(t)}{\sum_{t'\in \enaset{\marking}{\net}}\net.W(t')}$. As required, the probabilities associated to all enabled transitions in a marking always add up to 1.
% %For a run $\run$ of $\net$, its probability $\prob{\seq}{\net}$ is then obtained by summing up the probabilities of all valid sequences corresponding to $\run$: $\prob{\run}{\net} = \sum_{\seq \in \seqs{\run}{\net}} \prob{\seq}{\net}$. Likewise, for a trace $\trace$ of $\net$, its probability is obtained by summing up
% For convenience, when needed, we represent an $\net$-trace as a pair $\tup{\trace,\prob{\trace}{\net}}$, where the probability assigned to $\trace$ by $\net$ is retained.

We need to lift this approach so as to consider all occurrences of subtraces $\alpha\beta$ at every distance between $1$ and $|\trace|-1$. To do so, we proceed in two steps. First, we encode $\trace$ into a ``linear'' transition graph $\tg_\trace$ (\figurename~\ref{fig:taustar}) in the obvious way. %\todo{Tagliare dopo i due punti se necessario.} each node in $G_\sigma.V$  corresponds to an element of the trace labeled correspondingly, and the nodes representing two consecutive elements in the trace are connected with a transition probability of 1 (whereas in all the other cases, the probability is 0).
As a second step, we rely on the matrix operations to calculate a simplified version of the embedding defined in \cite{LodhiSSCW02} as $\trembed_{\alpha\beta}(\trace)=\sum_{1\leq i\leq |\trace|}\lambda^i[(\tg_{\trace}.\Lambda)^i]_{\alpha\beta}$. %\todo{No spazio per spiegare cosa succede...}
%This value can be seen as a reward.
The kernel between two traces corresponds to the sum of the products of such values calculated 2-gram by 2-gram for the two traces.
%, namely it is equal to the \emph{kernel convolution}. %\todo{L'ho provato a scrivere intuitivamente, ma non e' chiaro da dove arrivi questo modo di calcolarlo... deriva dalle formule sopra ma la digressione in mezzo e' lunga. Come possiamo fare per chiarire? L'esempio spiega bene tutto!}
This trace kernel returns strong dissimilarity when the two traces have no shared 2-grams at any arbitrary occurring length, but does not enjoy strong equality (as the similarity of a trace with itself is at least $\lambda^2$ - returned when the trace is a 2-gram).

%
%we can represent it as a TG \cite{Myers1989} $(1,{|\tau|},L_\tau,R_\tau,1)$ having $[L_\tau]_{{\color{green}\alpha}\texttt{\color{blue}i}}=1\Leftrightarrow \tau_{\texttt{\color{blue}i}}={\color{green}\alpha}$ and $[L_\tau]_{{\color{green}\alpha}\texttt{\color{blue}i}}=0$ otherwise, and $\forall i<|\tau|.\; [R_\tau]_{\texttt{\color{blue}i(i+1)}}=1 $ and $[R_\tau]_{\texttt{\color{blue}ij}}=0$ otherwise.
%Exploiting this encoding, we can adopt a simplified version of the embedding defined in \cite{LodhiSSCW02,Raedt} as $\embed_{\mathcal{T}}(\tau)_{{\color{green}\alpha\beta}}=\sum_{1\leq i\leq |\tau|}\lambda^i[(\Lambda_\tau)^i]_{\color{green}\alpha\beta}$.
%Please note that this definition is similar to a transition matrix embedding proposed in \cite{GartnerFW03} via geometric series, that is $\sum_i\lambda^i[R^i]_{\color{green}\alpha\beta}$.

\begin{figure}[!t]
	\centering
	\includegraphics[width=.4\textwidth]{images/taustar.pdf}
	\caption{Graphical representation of the transition graph encoding trace $\const{caba}$.}\label{fig:taustar}
	
\end{figure}
%
%\begin{example}\label{ex:tracembed}
%	{Let us suppose that we want to align a trace $\tau^*$ to one of the traces from a transition graph: in order to carry out an approximate alignment, we need to transform it to a transition graph first.} A trace $\tau^*=\textup{caba}$ can be graphically represented in Figure \ref{fig:taustar}. The associated TG $T=(\mathtt{\color{blue}1},\mathtt{\color{blue}4},L,R,1)$ has matrices $L$ and $R$  defined as follows:
%	$$L:=\kbordermatrix{
%		& \texttt{\color{blue}1}&\texttt{\color{blue}2}&\texttt{\color{blue}3}&\texttt{\color{blue}4}\\
%		\color{green}a            & 0&\textbf{1}&0&\textbf{1}\\
%		\color{green}b            & 0&0&\textbf{1}&0\\
%		\color{green}c            & \textbf{1}&0&0&0\\
%	}\qquad R:=\kbordermatrix{
%		& \texttt{\color{blue}1}&\texttt{\color{blue}2}&\texttt{\color{blue}3}&\texttt{\color{blue}4}\\
%		\texttt{\color{blue}1}  & 0&\color{red}1&0&0\\
%		\texttt{\color{blue}2}  & 0&0&\color{red}1&0\\
%		\texttt{\color{blue}3}  & 0&0&0&\color{red}1\\
%		\texttt{\color{blue}4}  & 0& 0& 0& 0\\
%	}$$
%We can similarly represent all the traces from the USPN.
%\end{example}

%\begin{example}
%The subtrace \textit{\textbf{\uline{hi}}} is represented in \textit{\textbf{\uline{hi}}deous},   \textit{\uline{\textbf{h}}e\uline{{i}}d\textbf{i}}, and \textit{\uline{{\textbf{h}i}}nd\textbf{i}}, but with different frequencies and subtrace distances. We have $\embed_{\mathcal{T}}(\textit{hideous})_{{\color{green}hi}}=\lambda$,  $\embed_{\mathcal{T}}(\textit{heidi})_{{\color{green}hi}}=\lambda^2+\lambda^4$, and $\embed_{\mathcal{T}}(\textit{hindi})_{{\color{green}hi}}=\lambda+\lambda^4$.
%\end{example}


%\begin{figure*}[!t]
	\begin{minipage}{.49\textwidth}
		\centering
		\includegraphics[width=.7\textwidth]{images/petri.pdf}
		\caption{A sample \uswn. Labels are shown in green, $\tau$ transitions in grey, weights in magenta.}\label{fig:spn}
	\end{minipage}\hfill \begin{minipage}{.49\textwidth}\centering
		\includegraphics[width=.8\textwidth]{images/rg.pdf}
		\caption{Reachability graph $RG(N)$ of the \uswn $N$. Probabilities are shown in violet.}\label{fig:rg}
	\end{minipage}
\end{figure*} \begin{figure*}[!t]
	\begin{minipage}{.49\textwidth}\centering \includegraphics[width=.55\textwidth]{images/running_example.pdf}
		\caption{Transition graph $G_{RG(N)}$ encoding the reachability graph $RG(N)$.}\label{fig:lmc}\label{fig:orig}
	\end{minipage}\hfill \begin{minipage}{.49\textwidth}\centering \includegraphics[width=.55\textwidth]{images/closed_example.pdf}
		\caption{Transition graph $\closed{G_{RG(N)}}$ resulting from the transition graph in $G_{RG(N)}$ after $\tau$-closure.}\label{fig:closed}
	\end{minipage}
\end{figure*}
\section{Computation Pipeline}
%\begin{figure}[!t]
%	\centering
%	\includegraphics[width=.49\textwidth]{images/petri_tut.pdf}
%	\caption{Stochastic Workflow Net modeling our use-case scenario.}\label{fig:petri_tut}
%\end{figure}
\begin{figure}[!t]
	\centering\includegraphics[width=\textwidth]{images/pipeline}
	\caption{Proposed pipeline to assess the probabilistic trace alignment.}\label{fig:pipe}
\end{figure}

%We now describe the proposed technique for computing probabilistic trace alignments. 
Our approach (\figurename~\ref{fig:pipe}) takes as input
\begin{inparaenum}[\it (i)]
	\item a reference model represented as Stochastic Workflow Nets $\net$ or an equivalent Transition Graph $G$,
	\item a minimum, positive probability threshold $\pmin \in (0,1]$
	\item a trace $\trace$ of interest,
\end{inparaenum}
and returns a ranking over all the $\net$-traces having a probability greater than or equal to $\pmin$, based on a combined consideration of their probability values and their distance to $\trace$.
\input{sections/Ζ_embedding_proposal}
\input{sections/Η_experiments}
\section{Conclusions and Future Works}

\bibliographystyle{splncs04}
\bibliography{biblio3}

\end{document}
