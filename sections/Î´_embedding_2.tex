% !TeX root=../main.tex
%\vspace{-0.5cm}
%As mentioned in the introduction, 
%When aligning a log trace with an \uswn, the retrieval of the best model trace maximizing the combined provision of minimum trace alignment cost and maximum model trace probability might not suffice. 
%An alignment of a trace with an \uswn could be represented by the best model trace maximizing the combined provision of minimum trace alignment cost and maximum model trace probability. However, in some cases, the user could prefer to identify an alignment with a lower cost even if based on a less probable model trace, while, in other cases, the user could favor a model trace with a higher probability at the expense of a higher alignment cost.
%Therefore, we need to return the best $k$ alignments among all the distinct \unravelled\ model traces in $\ptraces{\net}{\pmin}=\ptraces{\closed{\tg_{\rg{\net}}}}{\pmin}$. 
In the database community, this problem is usually tackled via $k$-Nearest Neighbors ($k$NN) that refers to finding the $k$ nearest data points to a \textit{query} $x$ from a set $\mathcal{X}$ of \textit{data points} via a distance function $d_k$ defined over $\mathcal{X}\cup\{x\}$. {In particular, by exploiting ad-hoc data structures, such as VP-Trees and KD-Trees, %VP-Trees \cite{Fu2000} \cite{Maneewongvatana99}, and M-Trees \cite{Ciaccia},
we can retrieve the neighborhood of $x$ in $\mathcal{X}$ of size $k$ by pre-ordering (\textit{indexing}) $\mathcal{X}$ via $d_k$ and starting the search from the top-$1$ alignment. Our tool benchmarks both strategies and gives results in CSV.

%Usually, $x$ and $\mathcal{X}$ are represented as numerical vectors via an embedding $\phi$. {By exploiting ad-hoc data structures such as Vp-Trees \cite{Fu2000}, Kd-Trees \cite{Maneewongvatana99}, and M-Trees \cite{Ciaccia}, we can retrieve the neighborhood of $x$ in $\mathcal{X}$ of size $k$  by pre-ordering (\textit{indexing}) $\mathcal{X}$  via $d_k$. Thus, we do not need to analyze the entire space, but just start the search from the top-$1$ alignment. We prefer embeddings $\phi$ for $\mathcal{X}$ that are independent of the query of choice $x$, as it would not require to constantly change the numeric vector representation for $\mathcal{X}$.

%The well-known $k$-nearest neighbors ($k$NN) problem \cite{Altman} refers to finding the $k$ nearest data points to a \textit{query} $x$ from a set $\mathcal{X}$ of \textit{data points} via a distance function $d_k$ defined over $\mathcal{X}\cup\{x\}$. Usually, $x$ and $\mathcal{X}$ are represented as numerical vectors via an embedding $\phi$. {By exploiting ad-hoc data structures such as Vp-Trees \cite{Fu2000}, Kd-Trees \cite{Maneewongvatana99}, and M-Trees \cite{Ciaccia}, we can retrieve the neighborhood of $x$ in $\mathcal{X}$ of size $k$  by pre-ordering (\textit{indexing}) $\mathcal{X}$  via $d_k$. Thus, we do not need to analyze the entire space, but just start the search from the top-$1$ alignment. We prefer embeddings $\phi$ for $\mathcal{X}$ that are independent of the query of choice $x$, as it would not require to constantly change the numeric vector representation for $\mathcal{X}$.
%	
%If we want to align a trace $\logtrace$ over the \unravelled\ traces $\mathcal{X}=\WCal{\pmin}{n}$,
%the $k$-Nearest Neighbors describe the best $k$ alignments for $\logtrace$.} We discuss two strategies for obtaining the
%top-$k$ trace alignments.


%\begin{figure*}[!t]
%	\begin{minipage}{.49\textwidth}
%		\includegraphics[width=\textwidth]{images/aligner_exact.pdf}
%		\caption{Optimal-Ranking Trace Aligner.}\label{fig:optimal}
%	\end{minipage}\hfill \begin{minipage}{.49\textwidth}
		
%		\includegraphics[width=\textwidth]{images/aligner.pdf}
%		\caption{Approx.-Ranking Trace Aligner.}\label{fig:approximate}
%	\end{minipage}
%\end{figure*}
%\vspace{-0.6cm}

\noindent
\textbf{1) Optimal-Ranking Trace Aligner.}\label{subsec:eta}
%One way to probabilistically align traces is to 
By reusing existing trace aligners  \cite{DBLP:conf/edoc/AdriansyahDA11,LeoniM17} providing an alignment cost  $d(\logtrace,\nonlogtrace)$ between model and log traces as the Levenshtein distance and
%, i.e., the {minimum} cost for aligning one of the two traces with respect to the other when all the possible moves have cost equal to 1.\footnote{This is slightly different from the Levenshtein distance from the literature since here the replacement is obtained as a concatenation of a deletion and an addition.}
using the decision theory, a we can express the ranking as a preliminary ranked distance $\probskip{\nonlogtrace} d(\logtrace,\nonlogtrace)$, considering both the alignment cost and the model trace probability.
%
We can represent it 
%To represent the same intuition of such a weighted distance 
as a ranking function %, we must transform it into a
%similarity function 
returning $1$ when $\nonlogtrace=\logtrace$ and $\probskip{\logtrace}=1$ hold. We then express $d$ as
%\xout{We can express our probabilistic trace alignment as finding the trace that maximizes both the trace's probability and its similarity with the query trace $\logtrace$. Still, the trace alignments problems are usually expressed via trace alignments cost functions, and not via trace similarities \cite{LeoniM17}. Given a generic trace cost function $d(\trace,\trace')$, it is always possible to convert it into}
a normalized similarity score $s_d(\logtrace,\nonlogtrace):=\frac{1}{d(\logtrace,\nonlogtrace)+1}$. %The maximum similarity is reached when the distance is $0$ and the similarity decreases while the distance increases. %\xout{As a consequence, the exact trace aligner will find the weighted trace $\braket{\trace,\mathbb{P}(\trace)}$ in $\mathcal{W}_{\pmin}(P)$ maximising $s_d(\trace,\logtrace)$, and use $s_d$ as a ranking function.} \ADD
{The golden ranking function (i.e., the one producing the optimal ranking) can therefore be represented as $\goldenrank(\logtrace,\nonlogtrace)=\probskip{\nonlogtrace} \probskip{\logtrace} s_d(\logtrace,\nonlogtrace)$. Such function can be exploited to generate the best optimal-ranking trace alignment for a log trace $\logtrace$, where $\goldenrank$ must be computed a-new for all the possible $\logtrace$.
	
	
\noindent
\textbf{2) Approximate-Ranking Trace Embedder.}\label{subsec:ate}
%Ranking optimality comes at the sub-optimal cost of a brute-force recomputation of $\goldenrank$ for each novel trace $\logtrace$ to align.
%\footnote{By embedding  all the \unravelled\ model traces via $\phi_{\logtrace}(\nonlogtrace)=\Big(\frac{1}{\probskip{\nonlogtrace}\sqrt{\probskip{\nonlogtrace}^2+s_d^2(\logtrace,\nonlogtrace)}},\frac{1}{s_d(\logtrace,\nonlogtrace)\sqrt{\probskip{\nonlogtrace}^2+s_d^2(\logtrace,\nonlogtrace)}}\Big)$ over a specific $\logtrace$, by representing  $\logtrace$ as $(0,0)$, and by using the Euclidean Distance as a distance, we ensure that all the neighbors of $(0,0)$ will represent the top-$k$ best alignments for $\logtrace$.}. %This is not optimal for $k$NN-based techniques, as this entails the  redefinition of the multidimensional search space $\mathcal{X}$ for each new alignment $\logtrace$. The ``recomputation cost'' could be completely avoided by providing an embedding strategy $\gorgembed$ that is independent of the to-be-aligned trace $\logtrace$.
Since each embedding $\phi$ entails an associated similarity metric $k_\phi$, %and hence an associated
%distance $d_{k_\phi}$ (Equation \ref{eq:dofk}), 
we can compute the embeddings for all the \unravelled\ traces
before performing the top-$k$ search ensuring that they are independent of the trace to align, thus avoiding the brute-force cost. This computational gain comes with a loss in precision %, as the generation of precise embeddings for graph data with loops is an NP-Complete (thus intractable) problem
 \cite{GartnerFW03} and, in its approximated version, is not able to accurately represent the data using low-dimensional vectors \cite{Seshadhri5631}. %So, our proposed embedding (that we indicate with $\gorgembed$) is weakly-ideal.
Differently from some graph embedding literature, we are interested in neither embedding one single node of the graph, nor in providing a vector representation of the whole graph. While these are common strategies for graph embeddings, we want to represent TG traces,  which might be composed by different valid sequences (paths), as vectors.  We are also interested in the intersection of embedding strategies for whole node-labeled graphs with string embedding traces, as we use vectors to compare TG traces with log traces. 
{To obtain our proposed embedding $\gorgembed$, we hence adapt the  embedding strategy} $\trembed$ from \cite{LodhiSSCW02} {by addressing some short\-comings: %that this strategy presents:}
%\begin{alphalist}
% \item it does not perform weakly-ideally, so we cannot numerically assess if two embeddings represent equivalent traces;
% \item it does not characterize $\tau$-moves, so the probabilities of the initial and final $\tau$-moves are not preserved;
% \item it is affected by numerical errors from finite arithmetics: longer traces $\nonlogtrace$ generated from skewed probability distributions $\expN.\Lambda^i$ suffer from greater truncation errors, as smaller $\lambda^i$ components for bigger $i<|\nonlogtrace|$ will be ignored, preventing a complete numerical vector characterization of  $\nonlogtrace$ in practice.
%\end{alphalist}
%
%{To overcome these shortcomings we} 
we
\begin{alphalist} \item propose a weakly-ideal embedding \cite{Gartner03} which, differently from current literature, \item exploits an
$\omega$ factor for preserving probabilities from and to $\tau$ transitions. We also \item mitigate the
numerical truncation errors induced by trace length and probability distribution skewness through two sub-embedding strategies,
$\epsilon^x$ and $\nu^x$, where the former descends from $\trembed$ and the latter approximates the trace similarity via label frequencies similarity.
\end{alphalist}
Since a trace embedding for $\nonlogtrace\in\WCal{\pmin}{n}$ adequately representing the transitions in our implementation requires an intermediate $G$ representation, we map each $\nonlogtrace$ to a pair $(\tg_\nonlogtrace,\omega)$}, where \begin{inparaenum}[\it (i)]
\item $\tg_\nonlogtrace$ is a transition graph containing all nodes and edges leading to the valid sequences in $\expN$ forming a trace $\nonlogtrace$, where all $\tau$-labelled nodes are removed, while
\item the graph weight $\omega$ preserves the weight of the possible initial and final edges that were removed due to the former requirement.
\end{inparaenum}

%The formal definition of such operation is omitted due to the lack of space.
%Please observe that the first requirement is mandatory, as present string embedding strategy do not cope with silent characters $\tau$.
%
%
%
%
%	\begin{definition}[$\closed{G}$ projection over traces]
%		Given a minimum probability thre\-shold $\pmin$ and a $\tau$-closed transition graph $\closed{G}=(V,s,t,L,R)$, for each trace $\nonlogtrace\in\ptraces{\closed{G}}{\pmin}$, the $\closed{G}$ projection over $\nonlogtrace$ is a pair ($\closed{G}_\nonlogtrace,\omega)$, where $\closed{G}_\nonlogtrace$ is a transition graph such that
%\begin{inparaenum}[\it (i)]
%	\item $\closed{G}_\nonlogtrace.V$ contains all distinct nodes generating $\nonlogtrace$ from $\closed{G}$	(i.e., $\bigcup_{\xi\in\runs{\nonlogtrace}{\closed{G}}}\seqs{\xi}{ \closed{G}}$);
%	\item $\closed{G}_\nonlogtrace.s = s$;
%	\item  $\closed{G}_\nonlogtrace.t = t$;
%	\item $\closed{G}_\nonlogtrace.L$ (and $\closed{G}_\nonlogtrace.R$) is the submatrix of $L$ (and $R$) over the columns (and rows) in $\closed{G}_\nonlogtrace.V$, and $\omega \in [0,1]$ is a \emph{graph weight} preserving the transition probabilities from and to $\tau$ nodes and computed as $1-\prod_{\xi\in\runs{\sigma}{\closed{G}},\eta\in\seqs{\xi}{\closed{G}}}\Big(1-(\textit{ifte}([L]_{\tau\eta_1},[R]_{\eta_1\eta_2})$
%	 $\textit{ifte}([L]_{\tau t_n},[R]_{t_{n-1} t_n})\Big)$,
%	where $\textit{ifte}(x,y):=x(y-1)+1$ returns $y$ if $x=1$ and $1$ otherwise. We denote the set of all the $\closed{G}_\nonlogtrace$ as $\TBf{\pmin}{4}(\closed{G})$.
%\end{inparaenum}
%\dots
%		
%		\dots we generate a set $\TBf{\pmin}{n}(\closed{G})$ of projected TGs $P$ for each trace in $\WCal{\pmin}{n}(P)$ as follows: for each weighted trace $\braket{\nonlogtrace,\probskip{\nonlogtrace}}\in\WCal{\pmin}{n}(P)$ generated from a path $\pi_\nonlogtrace=s\to n_2\rightsquigarrow n_m\to t$ over $R$, we generate a TG $P_\nonlogtrace=(s',t',L_\nonlogtrace,R_\nonlogtrace,\omega')$, where \begin{inparaenum}[\it (i)]
%			\item $s'=s$ if $\textit{label}(s)\neq \tau$ and $t'=n_2$ otherwise,
%			\item $t'=t$ if $\textit{label}(t)\neq \tau$ and $t'=n_m$ otherwise,
%			\item $L_\nonlogtrace$ (and $R_\nonlogtrace$) is the submatrix of $L$ (and $R$) over the non-$\tau$ labeled notes in $\pi_\nonlogtrace$ and the labels from $\nonlogtrace$,
%			\item $\omega'$ is initialized by $\omega$ and then multiplied by $[R]_{s,n_2}$ (and also $[R]_{n_m,t}$) if $\textit{label}(s)=\tau$ (and  $\textit{label}(t)=\tau$).
%		\end{inparaenum}
%\end{definition}
%	
\begin{table}[!t]
	\setlength{\textfloatsep}{-1pt}
	\caption{Projections of $\expN$ over $\net$-traces of length $4$.}\label{tab:proj}
	\centering
	\resizebox{.3\textwidth}{!}{\begin{tabular}{>{\centering\arraybackslash} m{1cm}| >{\centering\arraybackslash} m{4cm} >{\centering\arraybackslash} m{1cm} >{\centering\arraybackslash} m{1cm} }
			\toprule
			$\nonlogtrace$&$G_\nonlogtrace$&$l$&$\omega$\\
			\midrule
			$\const{a}$ & \includegraphics{images/trace_a} & $1$ & $\color{violet}\pa\pf$\\
			$\const{cb}$ & \includegraphics{images/trace_cb} & $2$ & $\color{violet}\pb$\\
			$\const{aaa}$ & \includegraphics{images/trace_a_loop} & $3$ & $\color{violet}\pa\pf$\\
			$\const{caa}$ & \includegraphics{images/trace_caa} & $3$ & $\color{violet}\pb\pf$\\
			\bottomrule
	\end{tabular}}\qquad \qquad 	
	\resizebox{.3\textwidth}{!}{\begin{tabular}{>{\centering\arraybackslash} m{1cm}| >{\centering\arraybackslash} m{4cm} >{\centering\arraybackslash} m{1cm} >{\centering\arraybackslash} m{1cm} }
	\toprule
	$\nonlogtrace$&$G_\nonlogtrace$&$l$&$\omega$\\
	\midrule
	$\const{aa}$ & \includegraphics{images/trace_aa} & $2$ & $\color{violet}\pa\pf$\\
	$\const{ca}$ & \includegraphics{images/trace_ca} & $2$ & $\color{violet}\pb\pf$\\
	\begin{tabular}{l}aaaa\end{tabular} & \includegraphics{images/trace_a_loop} & $4$ & $\color{violet}\pa\pf$\\
	$\const{caaa}$ & \includegraphics{images/trace_ca_loop} & $4$ & $\color{violet}\pb\pf$\\
	\bottomrule
\end{tabular}}
\end{table}
%The graph weight $\omega$ derives from the outgoing edges of the initial node and the ingoing edges of the accepting node when such nodes are labeled as $\tau$. Given that (i) the embedding strategy from \cite{LodhiSSCW02} allows trace embedding only for visible (i.e., non-$\tau$) transitions, and (ii) the trace extraction process discards the $\tau$ information, we use $\omega$ to preserve such information.
%We call the pair consisting of a transition graph and a graph weight a \emph{weighted transition graph}.
\begin{example}\label{ex:neue}
Given the $\tau$-closed transition graph $\closed{\tg_{\rg{\net}}}$ in \figurename~\ref{fig:closed}, we assign the probability values
$\pa=0.8$, $\pb=0.2$, $\pc=\pf=0.5$, $\pd=0.7$, and $\pe=0.3$. The $\ptraces{\expN}{0}$ with maximum length $4$ are:
%$$\begin{aligned}
%\ptraces{\expN}{0}_{|\nonlogtrace|\leq 4}=
$\{\braket{\const{a},0.4},\braket{\const{aa},0.2}$, $\braket{\const{aaa},0.1}$, $\braket{\const{ca},0.07}$, $\braket{\const{cb},0.06}$,
$\braket{\const{aaaa},0.05},\braket{\const{caa},0.035},\braket{\const{caaa},0.0175}\}$.
%\end{aligned}$$
Table \ref{tab:proj} shows the projected transition graphs associated to such traces, where only the relevant information
for embedding them is displayed (e.g., all the $\tau$-labeled nodes are removed).
\end{example}
	
{Our proposed embedding $\gorgembed$ is computed for each pair $(\tg_\nonlogtrace,\omega)$ generated from each unfolded trace %transition graph generated in the former definition. The goal is to use
%$k_{\gorgembed}$ for ranking all the traces generated by \unravelling\ via such graphs. 
We can now extend the embedding $\trembed$ \cite{LodhiSSCW02} by including the traces associated probability, and making the ranking induced by $k_{\gorgembed}$ the inverse of the ranking induced by the
sum of the following distances: the transition correlations $\epsilon$ and the transition label frequency $\nu$.}
%\xout{Given that we previously observed that a TGs $\closed{\tg_{\rg{\net}}}$ can be fully characterized (read, similarity) by their associated set of traces $\mathcal{W}_p^n(P)$  and that the trace embedding can be described as an embedding over a TG, we can characterize a TG embedding as a transition matrix embedding. In addition to that, when two Workflow Nets share similar node labelings but no ${\color{green}\alpha}\rightsquigarrow{\color{green}\beta}$ paths for any ${\color{green}\alpha}$ and ${\color{green}\beta}$, we should combine the former embedding with an embedding characterizing the frequency on how the nodes' labels appear in the generated traces.} \ADD
{We also require that the desired properties of $\gorgembed$ are independent of the characterization of $\epsilon$ over the $2$-grams in $\mathcal{A}^2$ and $\nu$ over the labels in $\mathcal{A}$, which  provide different embedding strategies. } Therefore, our proposed $\gorgembed$ embedding is defined as follows:


%\begin{table}[!t]
%	\centering
%	\caption{Embedding representation for the TG $P$ in \figurename~\ref{fig:closed} and the trace $\logtrace=\textup{caba}$ after representing it as in \figurename~\ref{fig:sigmastar}. Please note that we restrict $\trace_\tau^2$ to the one from $P$.}\label{tab:emb1}
%		\begin{tabular}{l|l|l|l|l|l|l|}
%	\toprule
%	& a    & b                                                   & c    & aa   & ca   & cb   \\
%	\midrule
%	$\gorgembed(P)$ & $9.94\cdot10^{-25}$ & $1.18\cdot 10^{-26}$ & $1.04\cdot10^{-25}$ & $4.45\cdot 10^{-25}$ & $6.22\cdot10^{-25}$ & $8.29\cdot10^{-26}$\\
%	$\gorgembed(P_{\logtrace})$ & $8.16\cdot10^{-17}$ & $4.08\cdot 10^{-17}$ & $4.08\cdot10^{-17}$ & $4.37\cdot 10^{-17}$ & $1.03\cdot10^{-16}$ & $4.37\cdot10^{-17}$\\
%	\bottomrule
%\end{tabular}
%\end{table}
\begin{definition}[G-Embedding]\label{def:ppne}
%Given a finite set of non-empty labels $\trace_\tau =\trace\backslash\{\tau\}$, $\trace_\tau^2$ denotes all the possible pair of labels associated to paths ${\color{green}\alpha}\rightsquigarrow{\color{green}\beta}$ and $\trace_\tau$ denotes the set of all the possible non-$\tau$ node labels. Therefore, it is always possible to enumerate $\trace_\tau^2\cup\trace_\tau$ via an enumeration by a bijection $\iota\colon \trace_\tau^2\cup\trace_\tau\to  N$, where $N\subset \mathbb{N}_{\neq 0}$ and $\max N=|N|$.
Given a $\closed{G}$ projection over $\nonlogtrace$ ($\closed{G}_\nonlogtrace,\omega)$ and a tuning parameter $t_f\in[0,1]\subseteq\mathbb{R}^+_{0}$, the \emph{G-Embedding} $\gorgembed$ %over the visible $2$-grams and transition labels, $\mathcal{A}\cup\mathcal{A}^2$, 
exploiting the matrix represantion of\cite{GartnerFW03} in $\nu$ and $\epsilon$ for TGs is defined as
$$\gorgembed_{i}(\closed{G}_\nonlogtrace)=\begin{cases}
	\omega \frac{\epsilon_\const{ab}(\closed{G}_\nonlogtrace)}{\|\epsilon\|_2}\;t_f^{|R>0|}\, & {i}=\const{ab}\\
	\frac{\nu_\const{a}(\closed{G}_\nonlogtrace)}{\|\nu\|_2}\;\;\;\,t_f^{|R>0|}\, & {i}=\const{a}\\
\end{cases}$$
\end{definition}
%
Here, the kernel function associated to such embedding can be exploited to return the best approximated trace alignment for a log trace represented as $G_{\logtrace}$. %\xout{Similarly, we can provide the TG $P\in\mathbf{P}$ providing the best approximated alignment for $P_{\logtrace}$ as $\underset{P}{\max\arg}\underset{ P_\trace\in\mathbf{P}_p^n(P)}{\max} k_{\gorgembed}(P_\trace, P_{\logtrace})$.}¯
%
\begin{table}[!t]
	\caption{Different sub-embedding definitions ($\epsilon^1$, $\epsilon^2$, $\nu^1$, and $\nu^2$) for $\gorgembed$.}\label{tab:embedstrat}
	\centering
	\resizebox{.6\textwidth}{!}{\begin{tabular}{c|c|c}
		\toprule
		& $x=1$ & $x=2$ \\
		\midrule
		$\epsilon^x_\const{ab}(\overline{G}_\nonlogtrace):=$ & $\label{eq:epsilon}
		\sum_{i=1}^l{\lambda^i}\frac{[LR^iL^t]_\const{ab}}{\sum_{\const{a'b'}\in\mathcal{A}^2}R^i_\const{a'b'}}$ & $
		\sum_{i=1}^l\lambda^i[\Lambda^i]_\const{ab}$\\
		$\nu^x_\const{a}(\overline{G}_\nonlogtrace):=$ & $\frac{1}{c}\sum_{\nonlogtrace'\in \ptraces{\closed{G}_\nonlogtrace}{0}}\frac{|\Set{\nonlogtrace_i'\in\nonlogtrace'|\const{a}\in\mathcal{A}\wedge \nonlogtrace'_i=\const{a}}|}{|\nonlogtrace'|}$ & $0$ \\
		\bottomrule
	\end{tabular}}
\end{table}
%
For sub-embeddings $\epsilon$ and $\nu$  we choose two possible interchangeable definitions ($x=1$ and $x=2$) shown in Table \ref{tab:embedstrat}: here, $l$ is the path length (reported in Table \ref{tab:proj}), and $c$ for $\nu^1$ is a normalization factor such that $\sum_{\const{a}\in\mathcal{A}}\nu^1_\const{a}(P)=1$. While $\nu^2$ implies to completely ignore the label frequency contribution, $\epsilon^2$ is the direct implementation of $\trembed$ from \cite{LodhiSSCW02}, and $\epsilon^1$ and $\epsilon^2$ only differ from the normalization perspective. %In Example \ref{ex:cmpexample}, we will see that the given definition of $\epsilon^1$ provides a better approximation than the edge embedding $\epsilon^2$.
%
$t_f\in [0,1]\subseteq\mathbb{R}^+_{0}$ (\textsf{Tuning}) and $\lambda\in [0,1]\subset\mathbb{R}^+_{0}$ (\textsf{Lambda}) are tuning parameters that can be
inferred from the available data \cite{DriessensRG06}. The latter describes the previously mentioned decay factor, while $t_f$
represents the relevance of our embedding representation as the number of edges within $\closed{G}_\nonlogtrace$ increases. In our experiments and
examples, we choose $t_f=0.0001$ and $\lambda=0.07$. Omitted proofs show that our embedding  performs weakly-ideally. 
%This representation is independent of the representation associated with a trace to be aligned. Therefore it does not have to be recomputed at each alignment with a different $\logtrace$.



%When the transition matrix is ergodic \cite{StocasticCC},  the transition matrix embedding converges to $\epsilon(R)_{\color{green}\alpha\beta}=[(\mathbf{I}-\lambda\Lambda)^{-1}]_{\color{green}\alpha\beta}$ \cite{GartnerFW03} for $n\to+\infty$.
%\begin{table}[!t]
%	\centering
%\vspace{+0.2cm}
%	\caption{Embeddings for $\net$-traces of maximum length $4$ and $\logtrace=\const{caba}$.}\label{tab:emb2}\label{tab:embsitar}
%	\resizebox{\textwidth}{!}{\begin{tabular}{lllllllllllll}
%		\toprule
%		& $\const{a}$    & $\const{b}$                                                   & $\const{c}$    & $\const{aa}$  & $\const{ab}$   & $\const{ac}$   & $\const{ba}$  & $\const{bb}$   & $\const{bc}$   & $\const{ca}$   & $\const{cb}$  & $\const{cc}$  \\
%		\midrule		
%		$\gorgembed(\overline{G}_\const{aaaa})$ & $1.00\cdot 10^{-24}$ & $0.00\cdot 10^{0}$   & $0.00\cdot 10^{0}$   & $6.44\cdot 10^{-26}$ & $0.00\cdot 10^{0}$& $0.00\cdot 10^{0}$& $0.00\cdot 10^{0}$& $0.00\cdot 10^{0}$& $0.00\cdot 10^{0}$    &$0.00\cdot 10^{0}$& $0.00\cdot 10^{0}$& $0.00\cdot 10^{0}$ \\
%		$\gorgembed(\overline{G}_\const{aaa})$  & $1.00\cdot 10^{-24}$ & $0.00\cdot 10^{0}$   & $0.00\cdot 10^{0}$   & $1.29\cdot 10^{-25}$ & $0.00\cdot 10^{0}$& $0.00\cdot 10^{0}$& $0.00\cdot 10^{0}$& $0.00\cdot 10^{0}$& $0.00\cdot 10^{0}$    &$0.00\cdot 10^{0}$& $0.00\cdot 10^{0}$& $0.00\cdot 10^{0}$ \\
%		$\gorgembed(\overline{G}_\const{aa})$   & $1.00\cdot 10^{-24}$ & $0.00\cdot 10^{0}$   & $0.00\cdot 10^{0}$   & $2.57\cdot 10^{-25}$ & $0.00\cdot 10^{0}$& $0.00\cdot 10^{0}$& $0.00\cdot 10^{0}$& $0.00\cdot 10^{0}$& $0.00\cdot 10^{0}$    &$0.00\cdot 10^{0}$& $0.00\cdot 10^{0}$& $0.00\cdot 10^{0}$ \\
%		$\gorgembed(\overline{G}_\const{a})$    & $1.00\cdot 10^{-4}$  & $0.00\cdot 10^{0}$   & $0.00\cdot 10^{0}$   & $0.00\cdot 10^{0}$   & $0.00\cdot 10^{0}$& $0.00\cdot 10^{0}$& $0.00\cdot 10^{0}$& $0.00\cdot 10^{0}$& $0.00\cdot 10^{0}$    &$0.00\cdot 10^{0}$& $0.00\cdot 10^{0}$& $0.00\cdot 10^{0}$ \\
%		$\gorgembed(\overline{G}_\const{caa})$  & $7.07\cdot 10^{-25}$ & $0.00\cdot 10^{0}$   & $7.07\cdot 10^{-25}$ & $1.46\cdot 10^{-25}$ & $0.00\cdot 10^{0}$& $0.00\cdot 10^{0}$& $0.00\cdot 10^{0}$& $0.00\cdot 10^{0}$& $0.00\cdot 10^{0}$    &$2.05\cdot 10^{-25}$& $0.00\cdot 10^{0}$& $0.00\cdot 10^{0}$ \\
%		$\gorgembed(\overline{G}_\const{ca})$   & $7.07\cdot 10^{-25}$ & $0.00\cdot 10^{0}$   & $7.07\cdot 10^{-25}$ & $0.00\cdot 10^{0}$   & $0.00\cdot 10^{0}$& $0.00\cdot 10^{0}$& $0.00\cdot 10^{0}$& $0.00\cdot 10^{0}$& $0.00\cdot 10^{0}$    &$1.00\cdot 10^{-8}$& $0.00\cdot 10^{0}$& $0.00\cdot 10^{0}$ \\
%		$\gorgembed(\overline{G}_\const{cb})$   & $0.00\cdot 10^{0}$   & $7.07\cdot 10^{-25}$ & $7.07\cdot 10^{-25}$ & $0.00\cdot 10^{0}$   & $0.00\cdot 10^{0}$& $0.00\cdot 10^{0}$& $0.00\cdot 10^{0}$& $0.00\cdot 10^{0}$& $0.00\cdot 10^{0}$    &$0.00\cdot 10^{0}$ & $4.29\cdot 10^{-9}$& $0.00\cdot 10^{0}$ \\
%		$\gorgembed(\overline{G}_\const{caaa})$ & $7.07\cdot 10^{-25}$ & $0.00\cdot 10^{0}$   & $7.07\cdot 10^{-25}$ & $1.03\cdot 10^{-25}$ & $0.00\cdot 10^{0}$& $0.00\cdot 10^{0}$& $0.00\cdot 10^{0}$& $0.00\cdot 10^{0}$& $0.00\cdot 10^{0}$    &$7.20\cdot 10^{-26}$ & $0.00\cdot 10^{0}$& $0.00\cdot 10^{0}$ \\
%		\bottomrule
%				$\gorgembed(\overline{G}_\const{caba})$ & $8.16\cdot10^{-17}$ & $4.08\cdot 10^{-17}$ & $4.08\cdot10^{-17}$ & $4.37\cdot 10^{-17}$ &  $0.00\cdot 10^{0}$& $0.00\cdot 10^{0}$& $0.00\cdot 10^{0}$ & $0.00\cdot 10^{0}$ & $0.00\cdot 10^{0}$    & $1.03\cdot10^{-16}$ & $4.37\cdot10^{-17}$& $0.00\cdot 10^{0}$ \\
%		\bottomrule
%	\end{tabular}}
%
%\end{table}%
%\xout{After defining the embedding, we can show that this embedding establishes some desired features that are independent of the definition of $\epsilon$ and $\nu$, and that $\epsilon$ and $\nu$ only depend on the characterization of both the labelling $L$ and the transition matrix $R$. We provide a rewriting proposition that is going to be used in the incoming subsection to provide the aforementioned characterizing properties.} \ADD
%
%Given that kernel functions $k_\phi$ are defined as the dot product between the embedding $\phi$ of distinct objects $x$ (\S\ref{subsec:katk}), then we can express the kernel $k_{\gorgembed}$ as such dot product and, after normalizing $\epsilon$ and $\nu$, we can rewrite such kernel
%
%%
%\begin{proposition}\label{lem:rewritinglemma}
%Given $(\overline{G}_\sigma,\omega)$ and $(\overline{G}_{\nonlogtrace},\omega')$ with $\overline{G}_\sigma=(s,t,L,R)$ and $\overline{G}_{\nonlogtrace}=(s',t',L',R')$, the definition for $k_\gorgembed$ is expanded as follows:
%$$
%k_{\gorgembed}(\overline{G}_\sigma,\overline{G}_{\nonlogtrace})=\omega\omega't_f^{|R>0|+|R'>0|}\left(1-\frac{\norm{\hat{\epsilon}-\hat{\epsilon}'}{2}^2}{2}\right)+t_f^{|R>0|+|R'>0|}\left(1-\frac{\norm{\hat{\nu}-\hat{\nu}'}{2}^2}{2}\right)\\$$
%\end{proposition}
%\begin{proof} We can close the goal by definition of $k_{\phi}$ as a vector dot product for any embedding $\phi$ and by  $\|\hat{\mathbf{x}}-\hat{\mathbf{x}}'\|_2^2=(2-1\braket{\hat{\mathbf{x}},\hat{\mathbf{x}}'})$ (\S\ref{subsec:katk}).	
%%	
%%	 for normalized vectors (\S\ref{subsec:katk}), we can expand the former definition as follows:
%%$$\begin{aligned}
%%{k_{\gorgembed}(P,P')}&{=\Braket{\gorgembed(P),\gorgembed(P')}}\\
%%	&{=\sum_{\alpha\beta\in \trace_\tau^2}\frac{\epsilon_{\color{green}\alpha\beta}}{\|\epsilon\|_2}\frac{{\epsilon'}_{\color{green}\alpha\beta}}{\|\epsilon'\|_2}\omega\omega't_f^{|R>0|+|R'>0|}\quad+\quad \sum_{\alpha\in \trace_\tau}\frac{\nu_{\color{green}\alpha}}{\|\nu\|_2}\frac{{\nu'}_{\color{green}\alpha}}{\|\nu'\|_2}t_f^{|R>0|+|R'>0|}}\\
%%	&{=ww'\trace^{|Rb>0|+|R'>0|}\sum_{\alpha\beta\in \trace_\tau^2}\frac{\epsilon_{\color{green}\alpha\beta}}{\|\epsilon\|_2}\frac{{\epsilon'}_{\color{green}\alpha\beta}}{\|\epsilon'\|_2}\quad+\quad t_f^{|R>0|+|R'>0|}\sum_{\alpha\in \trace_\tau}\frac{\nu_{\color{green}\alpha}}{\|\nu\|_2}\frac{{\nu'}_{\color{green}\alpha}}{\|\nu'\|_2}}\\
%%	&{=\omega\omega't_f^{|R>0|+|R'>0|}\Braket{\hat{\epsilon}, \hat{\epsilon}'}+ t_f^{|R>0|+|R'>0|}\Braket{\hat{\nu}, \hat{\nu}'}}\\
%%	&{=\omega\omega't_f^{|R>0|+|R'>0|}\left(1-\frac{\norm{\hat{\epsilon}- \hat{\epsilon}'}{2}^2}{2}\right)+ t_f^{|R>0|+|R'>0|}\left(1-\frac{\norm{\hat{\nu}- \hat{\nu}'}{2}^2}{2}\right)}\\
%%\end{aligned}$$
%\end{proof}
%%

%\xout{Given that we can now follow Definition \ref{def:ppne} for representing a trace $\trace$ as a proper embedding after transforming it as a TG $P_{\logtrace}$ (\S\ref{subsec:katk}), we can find the TG $P$ providing the best approximate match with  a trace $\trace$ as follows:}
%\[\Rcancel{\underset{{P}}{\max\arg}\;k_{\gorgembed}(P,T)}\]
%\xout{Still, this TG matching strategy does not allow to find the trace maximizing such score.} %To assess such problem, the next section is going to determine both an exact (\S\ref{subsec:exbkptap}) and an approximated strategy (\S\ref{subsec:akptap}) for probabilistically matching one single trace from the TG.
%
%\xout{Given the characterization of a TG as in \S\ref{subsec:ppn} and the embedding strategy proposed in Definition \ref{def:ppne}, We can \ADD{now} generate an embedding for each possible weighted trace $\braket{\trace,\probskip{\trace}}\in\mathcal{W}_p^n(P)$ for a given TG $P$ as described in the following definition:}





\begin{table}[!t]
\vspace{+1.3mm}
	\caption{Comparison between the ranking induced by the optimal ranking $\goldenrank$ and the proposed kernel $k_{\gorgembed}$ with embedding strategies $\epsilon^1$ and $\nu^1$: arrows $\boldsymbol{\downarrow}$ remark the column of choice under which we sort the rows.}\label{tab:rank3}
	\centering
	%	\begin{tabular}{l|c|ll}
	%		\toprule
	%		$\trace$ & $k_{\gorgembed}(\trace,\logtrace)$ & \textit{kernel ranking} & expected ranking\\
	%		\midrule
	%		a & $8.16\cdot 10^{-21}$ & \textbf{1} & \textbf{\color{blue}1}\\
	%		ca & $1.89\cdot 10^{-24}$ & \textbf{2} & \textbf{\color{blue}4}\\
	%		cb & $7.64\cdot 10^{-25}$ & \textbf{3} & \textbf{\color{blue}5}\\
	%		caa & $1.14\cdot 10^{-40}$ & \textbf{4} & \textbf{\color{blue}7}\\
	%		caaa & $9.84\cdot 10^{-41}$ & \textbf{5} & \textbf{\color{blue}8}\\
	%		aa & $9.28\cdot 10^{-41}$ & \textbf{6} & \textbf{\color{red}2}\\
	%		aaa & $8.72\cdot 10^{-41}$ & \textbf{7} & \textbf{\color{red}3}\\
	%		aaaa & $8.44\cdot 10^{-41}$ & \textbf{8} & \textbf{\color{red}6}\\
	%		
	%		\bottomrule
	%	\end{tabular}
	
	\resizebox{.6\textwidth}{!}{\begin{tabular}{l|ll|cc}
		\toprule
		
		{$\nonlogtrace$} &
		%\multirow{2}{*}{$d(\trace,\logtrace)$} &
		%\multicolumn{2}{c|}{$\mu_{\logtrace}$} &
		$( \probskip{\nonlogtrace}$ &  $,\,\boldsymbol{\downarrow} s_d(\logtrace,\nonlogtrace)) $ &
		{$=\goldenrank(\logtrace,\nonlogtrace)$} &
		{$k_{\gorgembed}(\closed{G}_\logtrace,\closed{G}_{\nonlogtrace})$} \\
		
		
		\midrule
		$\const{caa}$  & $0.035$ & $\;\; 0.8333$ & $0.0292$ & $1.14\cdot 10^{-40}$\\
		$\const{caaa}$  &  $0.0175$ & $\;\; 0.8333$ & $0.0145$ & $9.84\cdot 10^{-41}$\\
		$\const{a}$  & $0.4$ & $\;\; 0.6250$  & $0.2500$ & $8.16\cdot 10^{-21}$ \\
		$\const{aaaa}$  & $0.05$ & $\;\; 0.6250$ & $0.0357$ & $8.44\cdot 10^{-41}$\\
		$\const{aa}$  & $0.2$ & $\;\; 0.7142$ & $0.1428$ & $9.28\cdot 10^{-41}$ \\
		$\const{aaa}$  & $0.1$ & $\;\; 0.7142$ & $0.0714$ & $8.72\cdot 10^{-41}$\\
		$\const{ca}$  &  $0.07$ & $\;\; 0.7142$ & $0.0500$ & $1.89\cdot 10^{-24}$\\
		$\const{cb}$  &  $0.06$ & $\;\; 0.7142$ & $0.0428$ & $7.64\cdot 10^{-25}$\\
		\bottomrule
	\end{tabular}\quad \begin{tabular}{l|c}
	\toprule
	
	{$\nonlogtrace$} &
	{$\boldsymbol{\downarrow}\goldenrank(\logtrace,\nonlogtrace)$} \\
	
	
	\midrule
	$\const{\color{blue}a}$  &  $0.2500$ \\
	$\const{\color{red}aa}$  &  $0.1428$  \\
	$\const{\color{red}aaa}$  & $0.0714$ \\
	$\const{\color{blue}ca}$  &   $0.0500$\\
	$\const{\color{blue}cb}$  & $0.0428$ \\
	$\const{\color{red}aaaa}$  &  $0.0357$ \\
	$\const{\color{blue}caa}$  &  $0.0292$ \\
	$\const{\color{blue}caaa}$  &   $0.0145$ \\
	\bottomrule
\end{tabular}\quad	\begin{tabular}{l|c}
	\toprule
	
	{$\nonlogtrace$} &
	{$\boldsymbol{\downarrow}k_{\gorgembed}(\closed{G}_\logtrace,\closed{G}_{\nonlogtrace})$} \\

	
	\midrule
	$\const{\color{blue}a}$  & $8.16\cdot 10^{-21}$ \\
	$\const{\color{blue}ca}$  &   $1.89\cdot 10^{-24}$\\
	$\const{\color{blue}cb}$  &   $7.64\cdot 10^{-25}$\\
	$\const{\color{blue}caa}$  &$1.14\cdot 10^{-40}$\\
	$\const{\color{blue}caaa}$  &  $9.84\cdot 10^{-41}$\\
	$\const{\color{red}aa}$  &  $9.28\cdot 10^{-41}$ \\
	$\const{\color{red}aaaa}$  & $8.44\cdot 10^{-41}$\\
	$\const{\color{red}aaa}$  &  $8.72\cdot 10^{-41}$\\
	\bottomrule
\end{tabular}}
\end{table}




\begin{example}%\small \label{ex:11}
	\vspace{0.2cm}
%Table \ref{tab:emb2} shows the embeddings $\gorgembed(\overline{G}_\nonlogtrace)$ generated from Example \ref{ex:neue}, where the $l=|\nonlogtrace|$ for each $\unravelled$ trace $\nonlogtrace$.
%{After representing trace $\logtrace=\const{caba}$ to be aligned as a graph (\figurename~\ref{fig:taustar}), we can represent its}
%%\xout{Similar considerations can be also drawn from the}
%embedding $\gorgembed(\overline{G}_{\logtrace})$ with strategies $\epsilon^1$ and $\nu^1$ %\xout{associated to the trace $\logtrace=\textup{caba}$ (also in Table --):} \ADD
%{as} in Table \ref{tab:embsitar}: $\const{a}$ is clearly the most frequent label while $\const{b}$ and $\const{c}$ are equiprobable. The $2$-gram $\const{ca}$ appears twice in the trace set and then, it is more frequent than the remaining $2$-grams.
	The dot product $k_{\gorgembed}(\logtrace,\nonlogtrace)=\braket{\gorgembed(\overline{G}_\logtrace),\;\gorgembed(\overline{G}_{\nonlogtrace})}$ with sub-embed\-ding $\nu^1$ and $\epsilon^1$, for each trace $\nonlogtrace\in\WCal{0}{4}_{|\sigma'|\leq 4}$, are represented in Table \ref{tab:rank3}. %, where an explicit computation of $\gorgembed(\overline{G}_\logtrace)$ is omitted due to lack of space to the benefit of optimal ranking $\goldenrank$. 
	$k_{\gorgembed}$ approximates the optimal ranking as it tends to rank the transition graphs $\closed{G}_\nonlogtrace$ (generated from $\overline{G}$ via projection) similarly to the $\net$-traces over $\mathcal{R}$. In the same table, the ranking similarities shared between the two different ranking strategies are highlighted in blue, while the most evident ranking discrepancies are marked in red. Such similarities and differences can be assessed via Spearman Correlation Index \cite{BergamiBM20}. Evaluations exploiting such index were removed due to lack of space; our future work will include those as well as temporal benchmarks.
\vspace{0cm}
\end{example}


%\begin{example}\label{ex:moreskew}
%	\xout{Let us suppose to change the probability distribution associated with the $P$'s edges, so that it becomes more skewed and that some traces are relatively more probable than others. Let us set $p_1=p_2=0.5$, $p_3=0.9$, $p_6=0.1$, $p_4=0.3$, and $p_5=0.7$, so that the initial choice is equiprobable but performing a loop is more probable than terminating the path. We keep the other tuning parameters as in Example \ref{ex:withpaths}. In this case, we generate the following set of weighted traces:}
%	$$\begin{aligned}
%	\Rcancel{\mathcal{W}_0^4(P)=\{}&\Rcancel{\braket{cb,0.35},\braket{a,0.05},\braket{aa,0.045},\braket{aaa,0.0405},}\\
%	&\Rcancel{\braket{aaaa,0.03645},\braket{ca,0.015},\braket{caa,0.0135},\braket{caaa,0.01215}\}}\\
%	\end{aligned}$$
%	\xout{Let us also assume that we want to align these traces in a probabilistic way with the query $\logtrace=\textup{caba}$: the distance ($d$) and similarity ($s_d$) scores will be still the same, while the associated probabilities will vary. The expected ranking by multiplying weight with similarity is represented in Table \ref{tab:witherror}.}
%	
%	%\xout{As a consequence of the different probability distribution associated to the edges, a different set of embedding will be generated for each trace of interest while the TG $T$ associated to $\logtrace$ will be kept the same. Table \ref{tab:witherror} represents the ranking induced by the kernel $k_{\gorgembed}$ over this different set of vectors by ranking the traces in descendant order of $k_{\gorgembed}$. As we might notice, the more skewed edge probability distribution introduced more errors in the ranking result: while the largest ranking subsequence (marked in blue) always starts from the best-expected trace \textit{cb}, this element now appears in the third position, and the position of traces \textit{caa} and \textit{aaaa} is swapped.}
%	
%\end{example}
%\begin{table}[!t]
%	\centering
%	\caption{Expected ranking of the paths from Example \ref{ex:moreskew} with the trace $\logtrace=\textup{caba}$. The cost function is the one from \cite{LeoniM17} and its normalized similarity score has $c=5$. Traces are ranked by decreasing kernel $k_{\gorgembed}$ value: slight changes in the expected expected order are circled, the others are marked in red.}\label{tab:witherror}
%	\begin{tabular}{lc|ll|cc|l}
%		\toprule
%		
%		\multirow{2}{*}{$\trace$} &
%		\multirow{2}{*}{$d(\trace,\logtrace)$} &
%		\multicolumn{2}{c|}{$\mu_{\logtrace}$} &
%		\multirow{2}{*}{$\approx s_d(\trace,\logtrace)\cdot \probskip{\trace}$} &
%		\multirow{2}{*}{$k_{\gorgembed}(\trace,\logtrace)$}&
%		\multirow{2}{*}{\textit{expected ranking}}\\
%		
%		\cline{3-4} &&  $\langle \probskip{\trace}$ &  $,\,s_d(\trace,\logtrace)\rangle $ && \\
%		
%		\midrule
%		{a}  & $3$ & $0.05$ & $\;\; 0.6250$  & $0.03125$ & $8.16497\cdot 10^{-16}$ & \textbf{\color{red}3}\\
%		{ca}  & $2$ & $0.015$ & $\;\; 0.7142$ & $0.01071$ & $1.30623\cdot 10^{-18}$ & \textbf{\color{red}7}\\
%		{cb}  & $2$ & $0.35$ & $\;\; 0.7142$ & $0.25000$ & $1.01399\cdot10^{-18}$ & \textbf{\color{blue}1}\\
%		{aa}  & $2$ & $0.045$ & $\;\; 0.7142$ & $0.03214$ & $1.01894\cdot10^{-30}$ & \textbf{\color{blue}2}\\
%		{aaa}  & $2$ & $0.0405$ & $\;\; 0.7142$ & $0.02893$ & $9.98696\cdot10^{-31}$ & \textbf{\color{blue}4}\\
%		{caa}  & $1$ & $0.0135$ & $\;\; 0.8333$ & $0.01125$ & $9.96052\cdot10^{-31}$ & \textbf{\color{blue}\ding{177}}\\
%		{aaaa}  & $3$ & $0.03645$ & $\;\; 0.7142$ & $0.02603$ & $9.80476\cdot10^{-31}$ & \textbf{\color{blue}\ding{176}}\\
%		{caaa}  & $1$  & $0.01215$ & $\;\; 0.8333$ & $0.01012$ & $9.52398\cdot 10^{-31}$ & \textbf{\color{blue}8}\\
%		\bottomrule
%	\end{tabular}
%\end{table}
%\begin{table}[!t]
%	\caption{Comparison between the ranking induced by the expected ranking $\goldenrank$ and the proposed kernel $k_{\gorgembed}$ with embedding strategies $\epsilon^2$ and $\nu^1$: arrows $\boldsymbol{\downarrow}$ remark the column of choice under which we sort the rows (i.e., ranking).}\label{tab:compLit}
%	\centering
%	\resizebox{.9\textwidth}{!}{\begin{tabular}{l|ll|cc}
%			\toprule
%			
%			{$\trace$} &
%			%\multirow{2}{*}{$d(\trace,\logtrace)$} &
%			%\multicolumn{2}{c|}{$\mu_{\logtrace}$} &
%			$( \probskip{\trace}$ &  $,\,\boldsymbol{\downarrow} s_d(\trace,\logtrace)) $ &
%			{$=\goldenrank(\trace,\logtrace)$} &
%			{$k_{\gorgembed}(P_\trace,P_{\logtrace})$} \\
%			
%			
%			\midrule
%			$\const{caa}$  & $0.035$ & $\;\; 0.8333$ & $0.0292$ & $1.03498\cdot10^{-40}$\\
%			$\const{caaa}$  &  $0.0175$ & $\;\; 0.8333$ & $0.0145$ & $8.94997\cdot10^{-41}$ \\
%			$\const{a}$  & $0.4$ & $\;\; 0.6250$  & $0.2500$ & $8.16497\cdot10^{-21}$\\
%			$\const{aaaa}$ & $0.05$ & $\;\; 0.6250$ & $0.0357$ & $8.20640\cdot10^{-41}$\\
%			$\const{aa}$  & $0.2$ & $\;\; 0.7142$ & $0.1428$ & $9.96007\cdot10^{-41}$ \\
%			$\const{aaa}$  & $0.1$ & $\;\; 0.7142$ & $0.0714$ & $8.41263\cdot10^{-41}$\\
%			$\const{ca}$  &  $0.07$ & $\;\; 0.7142$ & $0.0500$ & $1.45079\cdot10^{-24}$\\
%			$\const{cb}$  &  $0.06$ & $\;\; 0.7142$ & $0.0428$ & $8.52070\cdot10^{-25}$\\
%			\bottomrule
%		\end{tabular}\quad \begin{tabular}{l|c}
%			\toprule
%			
%			{$\trace$} &
%			{$\boldsymbol{\downarrow}\goldenrank(\trace,\logtrace)$} \\
%			
%			
%			\midrule
%			$\const{a}$  &  $0.2500$ \\
%			$\const{aa}$  &  $0.1428$  \\
%			$\const{aaa}$  & $0.0714$ \\
%			$\const{ca}$  &   $0.0500$\\
%			$\const{cb}$  & $0.0428$ \\
%			$\const{aaaa}$  &  $0.0357$ \\
%			$\const{caa}$  &  $0.0292$ \\
%			$\const{caaa}$  &   $0.0145$ \\
%			\bottomrule
%		\end{tabular}\quad	\begin{tabular}{l|c}
%			\toprule
%			
%			{$\trace$} &
%			{$\boldsymbol{\downarrow}k_{\gorgembed}(P_\trace,P_{\logtrace})$} \\
%			
%			
%			\midrule
%			{a}  & $8.16497\cdot10^{-21}$ \\
%			{ca}  &   $1.45079\cdot10^{-24}$\\
%			{cb}  &   $8.52070\cdot10^{-25}$\\
%			{caa}  & $1.03498\cdot10^{-40}$\\
%			{aa}  &  $9.96007\cdot10^{-41}$ \\
%			{caaa}  &  $8.94997\cdot10^{-41}$ \\
%			{aaa}  &  $8.41263\cdot10^{-41}$\\
%			{aaaa}  & $8.20640\cdot10^{-41}$\\
%			\bottomrule
%	\end{tabular}}
%	
%	\centering
%	\begin{tabular}{lc|l}
%		\toprule
%		%\multicolumn{3}{c||}{Example \ref{ex:withpaths}} %&
%		%\multicolumn{3}{c}{Example \ref{ex:moreskew}}\\
%		%\hline
%		$\trace$ &  $k_{\gorgembed}(P_\trace,P_{\logtrace})$ & $\goldenrank(\trace,\logtrace)$\\ %&
%		%$\trace$ &  $k_{\gorgembed}(\trace,\logtrace)$ & \textit{exp. ranking}\\
%		\midrule
%		
%		a & $\;8.16497\cdot10^{-21}$ & \textbf{\color{blue}1} \\%& a & $\;8.16497\cdot 10^{-21}$ & \textbf{\color{red}3} \\
%		ca & $\;1.45079\cdot10^{-24}$ & \textbf{\color{blue}4} \\%& ca &  $\;1.45079\cdot 10^{-24}$ & \textbf{\color{red}7}\\
%		cb & $\;8.52070\cdot10^{-25}$ & \textbf{\color{blue}5} \\%& cb & $\;8.52070\cdot10^{-25}$& \textbf{\color{blue}1}\\
%		caa & $\;1.03498\cdot10^{-40}$ & \textbf{\color{blue}7} \\%& aa & $\;9.29342\cdot10^{-41}$ & \textbf{\color{blue}2}\\
%		aa & $\;9.96007\cdot10^{-41}$ & \textbf{\color{red}2} \\%& caa & $\;9.18112\cdot10^{-41}$ & \textbf{\color{blue}6}\\
%		caaa & $\;8.94997\cdot10^{-41}$ & \textbf{\color{red}8} \\%& caaa & $\;8.71867\cdot10^{-41}$ & \textbf{\color{blue}8}\\
%		aaa & $\;8.41263\cdot10^{-41}$ &  \textbf{\color{red}3} \\%& aaa & $\;8.31269\cdot10^{-41}$ & \textbf{\color{red}4}\\
%		aaaa & $\;8.20640\cdot10^{-41}$ &  \textbf{\color{red}6}\\% & aaaa & $\;8.19352\cdot10^{-41}$ & \textbf{\color{red}5}\\
%		
%		\bottomrule
%	\end{tabular}
%\end{table}

%\begin{example}\label{ex:cmpexample}
%	Let us compare the ranking results by replacing in $\gorgembed$ the edge embedding $\epsilon^1$ with $\epsilon^2$. If we re-run the computations performed in  Example \ref{ex:11}, we obtain   Table \ref{tab:compLit}:  $\epsilon^1$ provides longer approximated subsequences if compared to $\epsilon^2$. Both embedding proposals tend to favor sequences containing one single node or one single subtrace due to the normalization of both  the edge and the nodes' distribution, but $\epsilon^2$ seems to be less influenced than $\epsilon^1$ in the change of the edge distribution. Therefore, $\epsilon^1$ proposal is to be preferred to $\epsilon^2$.
%\end{example}

%\noindent
%\textbf{Properties.}\label{subsub:prop} Due to the lack of space, we omit the proofs for the considerations that follow. The kernel $k_{\gorgembed}$ associated to $\gorgembed$ is {a function of the distance  $\|\hat{\epsilon}-\hat{\epsilon}'\|_2^2$ and $\|\hat{\nu}-\hat{\nu}'\|_2^2$ for traces $\logtrace$ and ${\nonlogtrace}$:}
%When $\hat{\epsilon}$ and $\hat{\epsilon}'$ are affected by numerical cancelation due to truncation error (i.e., $\norm{\hat{\epsilon}-\hat{\epsilon}'}{2}^2\to 0$), the $\nu$ strategy intervenes as a backup ranking strategy. The first term of the sum does not affect the ranking, as it reduces to a constant factor.
%%
%We can prove that when two traces $\logtrace$ and ${\nonlogtrace}$ are equivalent (correspond to the same sequence of labels and have the same probability), the kernel computation reduces to $\omega\omega'$. When both weights are $1$, the kernel returns $1$. We call this condition \textit{weak equality} because we cannot possibly prove that when the kernel is equal to $\omega\omega'$ then equivalent embeddings come from equivalent traces. Traces having neither $2$-grams nor transition labels in common have kernel $0$ and vice versa (\textit{strong dissimilarity}). Since weak equality and strong similarity hold, we perform \textit{weakly-ideally}.

%\begin{figure}[!t]
%	\vspace*{-0.5cm}
%	\centering
%	\includegraphics[scale=0.7]{images/counterexample.pdf}
%	\caption{Two TGs, $Q$ (left) and $R$ (right), having a different set of traces but the same embedding representation.}\label{fig:counterexample}
%\end{figure}
%\begin{example}
%	If we use $\epsilon$ (or $\epsilon^2$) and $\nu$ (or $\nu^2$) for $\gorgembed$, we might have a false positive for ``weak equality'' if $Q=(s,s,L,R,w)$ and $R=(s',s',L,R,w)$ are both cycle graphs with $s\neq s'$, $\textit{label}(s)\neq\textit{label}(s')$, $\textit{label}(s)\neq\tau$, and $\textit{label}(s')\neq\tau$. An intuitive example of such situation is presented in Figure \ref{fig:counterexample}: both graphs will have the same frequency for both subtraces and nodes, and therefore have the same  $\epsilon$ and $\nu$ by construction. By having different initial and accepting node with  different labels, we have $\mathcal{W}_0^{\aleph_0}(Q)=\Set{\textup{a(bca)}^n|n\in\mathbb{N}}$ and $\mathcal{W}_0^{\aleph_0}(R)=\Set{\textup{c(abc)}^n|n\in\mathbb{N}}$, thus implying $\mathcal{W}_0^{\aleph_0}(Q)\neq\mathcal{W}_0^{\aleph_0}(R)$ but $k_{\gorgembed}(Q,R)=1$ for $t_f=1$.
%\end{example}
%
%
%\begin{lemma}[Weak Equality]
%	Given two TGs $P=(s,t,L,R,w)$ and $P'=(s',t',L',R',w')$ providing the same set of weighted traces, then $k_{\gorgembed}(P,P')=ww'$ for $t_f=1$.
%\end{lemma}
%%\begin{proof}
%%	\xout{Given Proposition \ref{lem:rewritinglemma} and the positive definition of $\epsilon$ and $\nu$,  we have that $\norm{\hat{\epsilon}-\hat{\epsilon}'}{2}\to 0$ as well as $\norm{\hat{\nu}-\hat{\nu}'}{2}\to 0$, for which we can immediately close the goal.}
%%\end{proof}
%
%%\xout{As per previous observations, we know that}
%Two TGs should have the maximum dissimilarity when all the non $\tau$-nodes have different labels, thus making it impossible to find an alignment, thus implying that they share an utterly dissimilar set of weighted traces:
%
%\begin{lemma}[Strong Dissimilarity]
%	Given two TGs $P=(s,t,L,R,w)$ and $P'=(s',t',L',R',w')$, $k_{\gorgembed}(P,P')=0$ iff. $P$ and $P'$ have a different set of labels with $t_f,w,w'>0$.
%\end{lemma}
%%\begin{proof}
%%	\xout{If we exclude the trivial conditions $t_f=0$, $w=0$ or $w'=0$, the only condition when the kernel returns zero is when  $\Braket{\hat{\epsilon},\hat{\epsilon}'}=0$ and $\Braket{\hat{\nu},\hat{\nu}'}=0$. This implies that, when a component of $\epsilon$ (or $\nu$) is non-zero, the same component of $\epsilon'$ (or $\nu'$) is zero and viceversa. This directly requires that there is a different set of labels associated to the nodes. }
%%\end{proof}
%
%As a corollary of the two lemmas, we have that the proposed embedding performs weakly-ideally as defined in \S\ref{subsec:katk}, as equality condition holds in a relaxed form.
%
%Last, under the assumption that a TG is approximately characterized by $\epsilon$ and $\nu$, we might expect that the TG similarity is characterized by the sum of the distance of both embeddings. Therefore, we show that an increase in both distance embeddings approximately corresponds to a decrease in the kernel output and vice-versa.
%
%\begin{lemma}\label{lem:approxRank}
%	Given two TGs $P=(s,t,L,R,w)$ and $P'=(s',t',L',R',w')$ having respectively the embeddings $(\epsilon,\nu)$ and $(\epsilon',\nu')$, we have that the kernel $k_{\trembed}$ induces an inverse ranking of $\norm{\hat{\epsilon}-\hat{\epsilon}'}{2}+\norm{\hat{\nu}-\hat{\nu}'}{2}$:
%	$$k_{\trembed}(P,P')\appropto 2-(\norm{\hat{\epsilon}-\hat{\epsilon}'}{2}+\norm{\hat{\nu}-\hat{\nu}'}{2})$$
%\end{lemma}
%%\begin{proof}
%%	\xout{Let us use $T=t_f^{|R>0|+|R'>0|}$, $\omega=ww'$, $V=\norm{\hat{\nu}-\hat{\nu}'}{2}$, and $E=\norm{\hat{\epsilon}-\hat{\epsilon}'}{2}$ as shorthands. The goal can be rewritten as $k_{\trembed}(P,P')\appropto 2-(E+V)$. Given that the embeddings $(\epsilon,\nu)$ and $(\epsilon',\nu')$ are normalized kernel function $k_{\gorgembed}$ and that they are always positive definite, then we have that $0\leq E +V\leq 2$, so $0\leq 2-(E+V)\leq 2$. Using Proposition \ref{lem:rewritinglemma}, we can write $k_{\gorgembed}(P,P')$ as follows:}
%%	$$\Rcancel{\left(1-\frac{E}{2}\right)\omega T+\left(1-\frac{V}{2}\right)T=T\left(\omega+1-\frac{E\omega+V}{2}\right)}$$
%%	\xout{Given that the embeddings $(\epsilon,\nu)$ and $(\epsilon',\nu')$ are normalized in $k$ and that they are always positive definite,  we also have that $0\leq E\omega +V\leq 2$ where $0\leq \omega\leq 1$. We can also write  $0\leq \omega+1-\frac{E\omega+V}{2}\leq \frac{2}{T}$. For $\omega,T=1$, we have that $k_{\gorgembed}(P,P')=2-\frac{(E+V)}{2}$. Thus, $0<\omega,T<1$ approximates the expected ranking. }
%%\end{proof}
%
%%\ADD{Such lemma is going to be empirically evaluated in our experiment section.}

\subsubsection*{Acknowledgments}
This research has been partially supported by the project IDEE (FESR1133) funded by the Eur.\ Reg.\ Development Fund (ERDF) Investment for Growth and Jobs Programme 2014-2020.
