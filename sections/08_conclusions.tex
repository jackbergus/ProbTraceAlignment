\section{Conclusions and Future Works}\label{sec:conclusion}
We framed probabilistic trace alignment as a $k$NN problem.
%Conceptually, this requires to handle the two possibly contrasting forces of the cost of the alignment on the one hand and the
%likelihood of the model trace with respect to which the alignment is computed. We consider the important tradeoff between both
%aspects.
Our approach balances  the likelihood of the aligned trace and the cost of the alignment by providing the top-k alignments instead of a single alignment as output. The experimentation shows that the approximated top-k ranking provides a good trade-off between accuracy and efficiency especially when the reference stochastic net generates several model traces.
Future works will investigate probabilistic alignments over fuzzy-labeled nodes and declarative process models. We also aim at improving the efficiency and accuracy of the proposed approach by intervening both on the embedding and the algorithmic strategies.

%\section*{Acknowledgements}
%This research has been partially supported by the project IDEE (FESR1133) funded by the Eur.\ Reg.\ Development Fund (ERDF) Investment for Growth and Jobs Programme 2014-2020. 