\section{Preliminary Definitions}
\subsection{Stochastic Petri Nets}\label{subsec:spn}
%As customary in probabilistic conformance checking \cite{DBLP:conf/bpm/LeemansSA19,DBLP:conf/icpm/PolyvyanyyK19,DBLP:journals/tosem/PolyvyanyySWCM20}, we adopt stochastic Petri nets \cite{MarsanCB84,RoggeSoltiAW13} as the underlying formal basis to represent processes. More specifically, we consider an interesting class of stochastic Petri nets with only immediate transitions (i.e., no timed ones).
%We assume to have a set $\alphabet = \tasks \cup \set{\tau}$ of labels, where labels in $\tasks$ indicate process tasks, whereas $\tau$ indicates an invisible execution step ($\tau$-transition). A \emph{trace} is a finite sequence of labels from $\tasks$. An \emph{untimed Stochastic Workflow Net (\uswn)}
%is a tuple $\net = (P,T,F,\ell,W)$ where:
%\textit{(i)} $(P,T,F)$ is a standard \emph{Workflow net} with places $P$, transitions $T$, and flow relation $F$ such that there is exactly one \emph{input place} with no incoming arc, and exactly one \emph{output place} with no outgoing arcs;
%\textit{(ii)} $\ell: T \rightarrow \alphabet$ is a \emph{labeling function} mapping each transition $t \in T$ into a label $\ell(t) \in \alphabet$;
%\textit{(iii)} $W\colon T\to \mathbb{R}^+$ is a \emph{weight function} assigning a positive firing weight to each transition.
%
%\texttt{[TODO: describe the notion of marking]}
%
%By interpreting concurrency by interleaving, we can represent all transition firings of an \uswn, together with their probabilities, in a reachability graph  $(M,E,P)$ where
%\textit{(i)} $M$ is the set of all reachable markings from the initial markings,
%\textit{(ii)} $E \subseteq M \times \alphabet \times M$ is a $\alphabet$-\emph{labeled transition relation} induced by $\net$, that is, for $\marking,\marking' \in M$, we have edge $(\marking,a,\marking') \in E$ if and only if there exists transition $t$ in $\net$ with label $\ell(t) = a$ and such that $\fire{\marking}{t}{\marking'}{\net}$, and
%\textit{(iii)} $P:E \rightarrow [0,1]$ is the \emph{transition probability} function assigning to each transition $(\marking,a,\marking') \in E$ its corresponding probability, obtained from the firing probability of the \uswn transition(s) that lead from $\marking$ to $\marking'$ and are labeled by $a$. 
In this paper, we consider a specific case of Stochastic Petri Nets, such as Workflow Networks over $k$-bounded Place-Transitions Nets \cite{MarsanCB84,Desel1998,RoggeSoltiAW13}. We assume to have a set $\alphabet = \tasks \cup \set{\tau}$ of labels, where labels in $\tasks$ indicate process tasks, whereas $\tau$ indicates an invisible execution step ($\tau$-transition). %Labels are associated to transitions via a labelling function $\lambda$. 
 Those Stochastic Petri Nets can be modelled as a tuple $\net=(P,T,F,W,i,f)$ where:
\begin{mylist}
	\item $P$ is a set of \textit{places} to which we can associate a finite number of indistinguishable tokens;
	\item $T$ is a set of \textit{transitions} $t\in T$, to which we associate a label $\lambda(t)\in\Sigma$;
	\item $F\subseteq (P\times T)\cup (T\times P)$ is a set of arcs; % to which we associate a \textit{firing cost} $\omega\colon F\to\mathbb{N}$;
	\item $W\colon T\to \mathbb{R}$ defines a \textit{firing weight} associated to each transition;
	\item the initial place $i\in P$ has no ingoing edges; %($\not\exists t\in T. (t,i)\in F$);
	\item the final place $f\in P$ has no outgoing edges. %($\not\exists t\in T. (f,t)\in F$).
\end{mylist}
A \textit{marking} is an assignment of a given amount of indistinguishable tokens to places described by a vector $M\colon P\to \mathbb{N}$. We say that a given transition $t$ is \textit{enabled} if $M(p)\geq \omega(p,t)$ for each ingoing $p$ to $t$ ($(p,t)\in F$). If such transition is enabled, then it can \textit{fire} a token. The set of the \textit{enabling transitions} $E(M)$ for a given marking $M$ are all the $t$ reachable from $p$ ($(p,t)\in F$) with $M(p)\neq 0$ where $t$ is enabled. When $t$ can fire a token for a marking $M$, we can generate a novel marking $M'$ from $M$ by moving the tokens from the ingoing places towards the outgoing places as 
$\forall p\in P.\; M'(p)=M(p)-[\omega(p,t)]+[\omega(t,p)]$. 
We denote the transition from marking $M$ to marking $M'$ via an enabling $t$ as a relation $M\overset{t}{\to}M'$. Given an initial marking $M$ for a Stochastic Petri Net $SPN$,  the \textit{Reachability Graph} for $SPN$ is a graph $(\mathcal{M},\mathcal{E})$ where the nodes  $\mathcal{M}$ are composed of of all the reachable markings from $M$, $M$ included, and all the edges $\mathcal{E}$ are induced by the aforementioned relation $M\overset{t}{\to}M'$ among the reachability graph's nodes. To each of such edges $M\overset{t}{\to}M'$, we associate a transition probability $\mathbb{P}\left(M\overset{t}{\to}M'\right)=\frac{W(t)}{\sum_{t'\in E(M)}W(t')}$ \cite{spdwe}. We say that a SPN with initial marking $M$ is $k$-\textit{bounded} if each of the nodes $M$ in the reachability graph have $\forall p\in P.\; M(p)\leq k$. 


A \emph{trace} is a finite sequence of labels from $\tasks$. 
%\begin{definition} An \emph{untimed Stochastic Workflow Net (\uswn)}
%	is a tuple $\net = (P,T,F,\ell,W)$ where:
%	\begin{inparaenum}[\itshape (i)]
%		%\begin{inparaenum}
%		\item $(P,T,F)$ is a standard \emph{Workflow net} with places $P$, transitions $T$, and flow relation $F$ such that there is exactly one \emph{input place} with no incoming arc, and exactly one \emph{output place} with no outgoing arcs;
%		\item $\ell: T \rightarrow \alphabet$ is a \emph{labeling function} mapping each transition $t \in T$ into a label $\ell(t) \in \alphabet$ - this either indicates the task executed upon firing $t$, or the fact that $t$ is an invisible transition (in the latter case, $\ell(t) = \tau$);
%		\item $W\colon T\to \mathbb{R}^+$ is a \emph{weight function} assigning a positive firing weight to each transition of the net.
%	\end{inparaenum}
%\end{definition}
%Given an \uswn $\net$, we use dot notation to extract its constitutive components (e.g., $\net.P$ denotes its places). \emph{The same dot notation will be used for the other structures introduced in the paper}. We also use $\net.in$ and $\net.out$ to respectively denote the input and output place of $\net$.
The current state of execution is captured by a marking, i.e., a multiset of places $P$ indicating how many tokens populate each place.
%As pointed out above, \emph{we always assume, as customary in BPM, that the input \uswn is \underline{bounded}}, that is, in every state the number of tokens associated to each place cannot exceed a maximum, fixed threshold.
%The notions of transition enablement and firing are also the standard ones  \cite{MarsanCB84}, as well as the ones for \textit{firing probability} \cite{spdwe}. %: %, which provides the basis for capturing the stochastic behavior of the net. 
%We use the following notation: given a marking $\marking$ over \uswn $\net$,  $\enaset{\marking}{\net}$ is the set of enabled transitions in $\marking$; given transition $t \in \enaset{\marking}{\net}$, we write $\fire{\marking}{t}{\marking'}{\net}$ to capture the fact that, with, firing $t$ in $\marking$ results in the new marking $\marking'$. A \emph{firing sequence starting from marking $\marking_0$} is a sequence $t_1\cdots t_n$ of transitions from $\net.T$ so that, for every $i \in \set{1,\ldots,n}$, we have that $\fire{\marking_{i-1}}{t_i}{\marking_{i}}{\net}$. We say that the firing sequence results in $\marking_{n}$.
%
%given a marking $\marking$ %of $N$ 
%and an enabled transition $t \in T_e$, the \emph{firing probability} of $t$ in $\marking$ is $\probt{t}{\marking}{\net} = \frac{W(t)}{\sum_{t'\in T_e}W(t')}$. %As required, 
The probabilities associated to all enabled transitions in a marking always add up to 1.
A \emph{valid sequence} $\seq = t_1\cdots t_n$ is a firing sequence starting from $m_{in}$ and resulting in $m_{out}$. The probability $\prob{\seq}{\net}$ of a valid sequence is the product of the probabilities associated to each transition. %: $\prob{\seq}{\net} = \prod_{i \in \set{1,\ldots,n}}\prob{t_i}{\marking_{i-1},\net}$. % A sequence of labels $\run = \alpha_1 \cdots \alpha_n$ from $\alphabet$ is a \emph{run} if there exists a valid underlying sequence $\seq = t_1\cdots t_n$  having $\alpha_i$ as a label for each $t_i\in \seq$. Run $\run$ may have different underlying valid sequences in $\seqs{\run}{\net}$.
A trace $\nonlogtrace=\const{a}_1\cdots \const{a}_m$ is a \emph{model trace} (or $\net$-trace for short) if there exists a valid sequence $\seq = t_1\cdots t_n$ where the appended labels $\lambda(t_1)\cdots \lambda(t_n)$ is equivalent to $\trace$ once all the $\tau$-s are stripped.
%underlying run $\run$ corresponding to $\trace$ once all $\tau$ are removed. 
There may be multiple valide sequences $\seq\in\seqs{\nonlogtrace}{\net}$ %$\runs{\trace}{\net}$ 
underlying an $\net$-trace $\trace$. 
$\traces{\net}$ is the (possibly infinite) set of $\net$-traces. For a trace $\nonlogtrace$ of $\net$, its probability $\prob{\nonlogtrace}{\net}$ is then obtained by collecting all its underlying valid sequences, %, in turn collecting all their underlying valid sequences, 
and summing up their respective probabilities. %: $\prob{\trace}{\net} = %\sum_{\run \in \runs{\trace}{\net}} 
%\sum_{\seq \in \seqs{\trace}{\net}} \prob{\seq}{\net}$. 
This corresponds to the intuition that, to observe $\nonlogtrace$, one can equivalently pick any of its underlying valid sequences. Notably, if a trace is not an $\nonlogtrace$-trace (i.e., it does not conform with $\net$), then its probability is 0. Given a log trace $\logtrace$ coming from a log file $\mathcal{L}$, we always associate to it a certain probability, i.e., $\prob{\logtrace}{\mathcal{L}}=1$. 

\subsection{Graph and String Kernels}\label{subsec:katk}
 As a foundational basis to compute trace alignments, we adapt similarity measures from the database literature.  Given a set of data examples $\mathcal{X}$, (e.g., strings or traces, transition graphs) a (positive definite) \emph{kernel} function $k\colon \mathcal{X}\times \mathcal{X}\to \mathbb{R}$ denotes the similarity of elements in $\mathcal{X}$. If $\mathcal{X}$ is the $d$-dimensional Euclidean Space $\mathbb{R}^d$, the simplest kernel function is the inner product $\Braket{\mathbf{x},\mathbf{x}'}=\sum_{1\leq i\leq d}\mathbf{x}_i\mathbf{x}'_i$.
A kernel is said to \emph{perform ideally} \cite{Gartner03} when $k(x,x')=1$ whenever $x$ and $x'$ are the same object (\textit{strong equality}) and $k(x,x')=0$ whenever $x$ and $x'$ are distinct objects (\textit{strong dissimilarity}). A kernel is also said to be \emph{appropriate} when similar elements $x,x'\in\mathcal{X}$ are also close in the feature space. Notice that appropriateness can be only assessed  empirically \cite{Gartner03}.
A positive definite kernel induces a distance metric as 
$
d_k(\mathbf{x},\mathbf{x}'):=\sqrt{k(\mathbf{x},\mathbf{x})-2k(\mathbf{x},\mathbf{x}')+k(\mathbf{x}',\mathbf{x}')}
$.
When the kernel of choice is the inner product, the resulting distance is the Euclidean distance $\norm{\mathbf{x}-\mathbf{x}'}{2}$. A normalized vector $\hat{\mathbf{x}}$ is defined as $\mathbf{x}/\norm{\mathbf{x}}{2}$. For a normalized vector we can easily prove that: $\norm{\hat{\mathbf{x}}-\hat{\mathbf{x}}'}{2}^2=2(1-\Braket{\hat{\mathbf{x}},\hat{\mathbf{x}}'})$.
When $\mathcal{X}$ does not represent directly a $d$-dimensional Euclidean space, we can use an \emph{embedding} $\embed\colon\mathcal{X}\to \mathbb{R}^d$ to define a kernel $k_\embed\colon \mathcal{X}\times \mathcal{X}\to\mathbb{R}$ as $k_\embed(x,x'):=\Braket{\embed(x),\embed(x')}$. As a result, $k_\embed(x,x')=k_\embed(x',x)$ for each $x,x'\in\mathcal{X}$.



 The literature also provides a kernel representation for strings \cite{LodhiSSCW02,GartnerFW03}: if we associate each dimension in $\mathbb{R}^d$ to a different sub-string $\alpha\beta$ of size $2$ (i.e., $2$-grams\footnote{\label{fn:caveat}For our experiments, we choose to consider only $2$-grams, but any $p$-grams of arbitrary length $p\geq 2$ might be adopted \cite{Gartner03}. An increased size of $p$ improves precision but also incurs in a worse computational complexity, as it requires to consider all the arbitrary subtraces of length $p$ whose constitutive elements occur at any distance from each other within the trace.}), it should represent how frequently and ``compactly'' this subtrace is embedded in the trace $\trace$ of interest. Therefore, we introduce a \emph{decay factor} $\lambda\in[0,1]\subseteq\mathbb{R}$ that, for all $m$ sub-strings where $\alpha$ and $\beta$ appear in $\trace$ at the same relative distance $l < |\trace|$, weights the resulting embedding as $\lambda^lm$.



\begin{table}[t!]
	\vspace{+0.7cm}
	\caption{Embedding of traces $\const{caba}$, $\const{caa}$ and $\const{cb}$.}\label{tb:embedding}
	\vspace{-0.4cm}
	\begin{center}
		%\scalebox{0.6}
		{
			\begin{tabularx}{\textwidth}{
					>{\hsize=.1\hsize}X
					>{\hsize=.2\hsize}X
					>{\hsize=.1\hsize}X
					>{\hsize=.1\hsize}X
					>{\hsize=.1\hsize}X
					>{\hsize=.1\hsize}X
					>{\hsize=.1\hsize}X
					>{\hsize=.25\hsize}X
					>{\hsize=.2\hsize}X
					>{\hsize=.1\hsize}X
				}
				\toprule
				& $\const{aa}$    & $\const{ab}$   & $\const{ac}$    & $\const{ba}$   & $\const{bb}$   & $\const{bc}$ & $\const{ca}$ & $\const{cb}$ & $\const{cc}$   \\
				\midrule
				$\const{caba}$ & $\lambda^2$ & $\lambda$ & $0$ & $\lambda$  & $0$  & $0$ & $\lambda+\lambda^3$ & $\lambda^2$ & $0$\\
				%$\const{caaa}$ & $2\lambda+\lambda^2$& $0$ & $0$ & $0$ & $0$ & $0$ & $\lambda+\lambda^2+\lambda^3$ & $0$ & $0$ \\
				$\const{caa}$  & $\lambda$ & $0$ & $0$ & $0$ & $0$ & $0$ & $\lambda+\lambda^2$ & $0$&  $0$\\
				$\const{cb}$   & $0$ & $0$ & $0$ & $0$ & $0$ & $0$ & $0$ & $\lambda$& $0$ \\
				\bottomrule
			\end{tabularx}
		}
		\vspace{-0.3cm}
	\end{center}
\end{table}
\begin{example}\label{ex:wheredotiszero} %\small
	Consider tasks $\tasks=\Set{a,b,c}$. The possible 2-grams over $\tasks$ are $\tasks^2=\Set{\const{aa},\const{ab},\const{ac},\const{ba},\const{bb},\const{bc},\const{ca},\const{cb},\const{cc}}$. Table~\ref{tb:embedding} shows the embeddings of some traces. Being a 2-gram, trace $\const{cb}$ has only one nonzero component, namely that corresponding to itself, with $\trembed_{\const{cb}}(\const{cb})=\lambda$. Trace $\const{caa}$ has the 2-gram $\const{ca}$ occurring with length $1$ ($\const{\underline{ca}a}$) and $2$ ($\const{\underline{c}a\underline{a}}$), and the 2-gram $\const{aa}$ with occurring length $1$ ($\const{c\underline{aa}}$). Hence: $\trembed_{\const{ca}}(\const{caa})=\lambda+\lambda^2$ and  $\trembed_{\const{aa}}(\const{caa})=\lambda$.  Similar considerations can be carried out for the other traces in the table.
	We now want to compute the similarity between the first trace $\const{caba}$ and the other two traces. To do so, we sum, column by column (that is, 2-gram by 2-gram) the product of the embeddings for each pair of traces. We then get $k_{\trembed}(\const{caba},\const{caa})=\lambda^3+(\lambda+\lambda^3)(\lambda+\lambda^2)$ and $k_{\trembed}(\const{caba},\const{cb})=\lambda^3
	$,
	%{\footnotesize
	%\[
	%k_{\trembed}(\const{caba},\const{caaa})=\lambda(\lambda+\lambda^2+\lambda^3)
	%~~
	%k_{\trembed}(\const{caba},\const{caa})=\lambda(\lambda+\lambda^2)
	%~~
	%k_{\trembed}(\const{caba},\const{cb})=\lambda(\lambda+\lambda^3)
	%\]}
	which induces ranking $
	k_{\trembed}(\const{caba},\const{caa})>
	k_{\trembed}(\const{caba},\const{cb})
	$.
\end{example}

Nevertheless, such string embedding has several shortcomings: \begin{alphalist}
	\item it is not weakly-ideal, so we cannot numerically assess if two embeddings represent equivalent traces 
	(Example \ref{ex:wheredotiszero});
	\item it does not characterize $\tau$-moves, so the probabilities of the initial and final $\tau$-moves are not preserved; and
	\item it is affected by numerical errors from finite arithmetic: longer traces $\nonlogtrace$ generated from skewed probability 
	distributions $G.\Lambda^i$ yield greater truncation errors, as smaller $\lambda^i$ components for bigger 
	$i<|\nonlogtrace|$ are ignored, preventing a complete numerical vector characterization of  $\nonlogtrace$ in practice.
\end{alphalist}

\endinput




