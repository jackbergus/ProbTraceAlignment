\section{Alignment Strategy}\label{subsec:as}

When aligning a log trace with a TG, retrieving the model trace maximizing the combined provision of minimum trace alignment cost 
and maximum model trace probability does not suffice.
Hence, we find the best $k$ alignments among all  model traces in $\ptraces{\net}{\pmin}$. This reduces to the
$k$-nearest neighbors ($k$NN) problem by finding the $k$ nearest data points to a \textit{query} $x$ from a set 
$\mathcal{X}$ of \textit{data points} w.r.t.\ a given distance function $d_k$. Through ad-hoc data structures, such as VP-Trees 
and KD-Trees, %VP-Trees \cite{Fu2000} \cite{Maneewongvatana99}, and M-Trees \cite{Ciaccia},
we can retrieve the $k$-neighborhood of $x$ in $\mathcal{X}$ by pre-ordering (\textit{indexing}) $\mathcal{X}$ with respect to $d_k$. 
%and searching from the top-$1$ alignment.
%
%To align a trace $\logtrace$ over the \unravelled\ traces $\ptraces{\tg}{\pmin}$,
%the $k$-Nearest Neighbors describe the best $k$ alignments for $\logtrace$. 
%We discuss two strategies to obtain these
%alignments.

\smallskip
\noindent
\textbf{Optimal-Ranking Trace Aligner.}
One approach is to reuse existing trace aligners and compute the alignment cost for each model trace $\nonlogtrace$ to be aligned with a model trace at a time $\logtrace$. Customary alignments \cite{DBLP:conf/edoc/AdriansyahDA11,LeoniM17} can be efficiently computed via string Levenshtein distance $d(\logtrace,\nonlogtrace)$: therefore, we will consider traces as strings with associated probability values. Given that similarity is the inverse function of distance and that, concerning the alignment task, we want to find the model trace maximizing both probability and similarity with the log trace, the problem boils down to maximizing the product $p\cdot s$, where $p=\prob{\logtrace}{\mathcal{L}}\prob{\nonlogtrace}{\net}=\prob{\nonlogtrace}{\net}$ and $s=\frac{1}{\frac{1}{c}d(\logtrace, \nonlogtrace)+1}$, where $c\in\mathbb{N}_{\neq 0}$ is a constant. We refer to $p\cdot s$ as the golden ranking function denoted as $\mathcal{R}(\logtrace,\nonlogtrace)$.


%Using decision theory \cite{dectheor}, we express the ranking score as the product 
%$\probskip{\nonlogtrace} d(\logtrace,\nonlogtrace)$, taking into account the cost of the alignment (the distance between 
%the model trace and the trace to be aligned) and the probability of the model trace.
%
%To represent the same intuition of a weighted distance as a ranking function, we transform it into a
%similarity function returning $1$ when $\nonlogtrace=\logtrace$ and $\probskip{\logtrace}=1$ hold. We express $d$ as
%a normalized similarity score $s_d(\logtrace,\nonlogtrace):=\frac{1}{\frac{1}{c}d(\logtrace,\nonlogtrace)+1}$, where  
%$c\in\mathbb{N}_{\neq0}$ is a constant. The maximum similarity is reached when the distance is $0$ and the similarity decreases 
%as the distance increases. 
%The golden ranking function producing the optimal ranking is 
%$\goldenrank(\logtrace,\nonlogtrace)=\probskip{\nonlogtrace} \probskip{\logtrace} s_d(\logtrace,\nonlogtrace)$;
%${\max\arg}_{\nonlogtrace\in \WCal{\pmin}{n}} \goldenrank(\logtrace,{\nonlogtrace})$ yields the best optimal-ranking trace 
%alignment for a log trace $\logtrace$, where $\goldenrank$ must be computed anew for each possible $\logtrace$. 
%{We can represent each trace as a point  $(\mathbb{P}_N(\sigma)\mathbb{P}_N(\sigma'),\; s_d(\sigma,\sigma'))$ in the 
%	2-dimensional similarity/probability space of coordinates $(p,s)$, reducing the trace finding problem to maximising the product 
%	$p\cdot s\equiv \mathbb{P}_N(\sigma)\mathbb{P}_N(\sigma')\cdot s_d(\sigma,\sigma')$ (Figure \ref{fig:sps}).}
%
%\begin{figure}[!t]
%	\centering
%	\vspace{-4mm}
%	\subfloat{\label{fig:spp}\includegraphics[width=.5\textwidth]{images/original_space.pdf}}
%	\subfloat{\label{fig:knnspace}\includegraphics[width=.5\textwidth]{images/transformed_space.pdf}}\\
%	\caption{Two characterizations of probabilistic trace alignment; the similarity/probability space left, and the transformed
%		$k$NN space right. The best possible match is shown in red in both spaces.}\label{fig:sps}
%\end{figure}


\begin{table}[!t]
	\vspace{5mm}
	\centering
	\caption{Golden ranking of model traces with maximum length $4$, where $\logtrace=\const{caba}$ and $c=5$.}\label{tab:expected}
	\resizebox{.45\textwidth}{!}{\begin{tabular}{lc|ll|c}
			\toprule
			
			{$\nonlogtrace$} &
			{$d(\sigma',\sigma)$} &
			$( \mathbb{P}_N(\sigma')$ &  $,\,s_d(\sigma',\sigma)) $ &
			{$\approx s_d(\sigma,\sigma^*)\cdot w_\sigma$} \\
			
			
			\midrule
			$\const{a}$  & $3$ & $0.4$ & $\;\; 0.6250$  & $0.2500$\\
			$\const{aa}$  & $2$ & $0.2$ & $\;\; 0.7142$ & $0.1428$\\
			$\const{aaa}$  & $2$ & $0.1$ & $\;\; 0.7142$ & $0.0714$ \\
			$\const{ca}$  & $2$ & $0.07$ & $\;\; 0.7142$ & $0.0500$\\
			$\const{cb}$  & $2$ & $0.06$ & $\;\; 0.7142$ & $0.0428$ \\
			$\const{aaaa}$  & $3$ & $0.05$ & $\;\; 0.7142$ & $0.0357$ \\
			$\const{caa}$  & $1$ & $0.035$ & $\;\; 0.8333$ & $0.0292$ \\
			$\const{caaa}$  & $1$  & $0.0175$ & $\;\; 0.8333$ & $0.0145$ \\
			\bottomrule
	\end{tabular}}
\end{table}
\begin{example}\label{ex:rankingTaus}
	Consider the TG $\closed{\tg}$ in \figurename~\ref{fig:lmc} with probabilities
	$\pa=0.8$, $\pb=0.2$, $\pc=\pf=0.5$, $\pd=0.7$, and $\pe=0.3$. The traces with maximum length $4$ are:
	%$$\begin{aligned}
	%\ptraces{\expN}{0}_{|\nonlogtrace|\leq 4}=
	$\{\braket{\const{a},0.4},\braket{\const{aa},0.2}$, $\braket{\const{aaa},0.1}$, $\braket{\const{ca},0.07}$, 
	$\braket{\const{cb},0.06}$,
	$\braket{\const{aaaa},0.05},\braket{\const{caa},0.035},\braket{\const{caaa},0.0175}\}$. 
	Table \ref{tab:expected} represents their alignment raking with  $\nonlogtrace=\textup{caba}$.  Although $\const{caa}$ and 
	$\const{caaa}$ are the most similar to $\const{caba}$, their associated probability  is rather low, so traces with 
	higher probability but lower similarity score are preferred (e.g., $\const{a}$ and $\const{aa}$).
\end{example}
%
{Since users might still prefer the most similar traces to the ones maximizing both probability and similarity, we 
	return the best $k$ solutions.
	We reduce the problem to $k$NN over the Euclidean Space through a transformation $t$ such that the distance of the 
	transformed point $t(p,s)$ from  the origin $\vec{0}$ is $\sfrac{1}{ps}$. This preserves the score from the golden ranking 
	and maps the points maximizing $ps$ close to the origin, but it requires to recompute the transformation for each new log trace $\logtrace$. We choose}
$t(p,s):=\left(\frac{1}{s\sqrt{p^2+s^2}},\; \frac{1}{p\sqrt{p^2+s^2}}\right)$. 
Our search always starts from the origin, and hence finds the best candidates first.
%
%\begin{example}
	{Given a family of hyperbolae $p\cdot s$ with all the alignments having 
		$\mathcal{R}(\sigma,\sigma')=p\cdot s$. 
%		Point $(1,1)$ is the best possible trace match, i.e., a trace 
%		$\nonlogtrace\in\ptraces{G}{0}$ with $\nonlogtrace=\logtrace$ and $\mathbb{P}_G(\nonlogtrace)\mathbb{P}_G(\logtrace)=1$.
%		%		
%		Figure \ref{fig:knnspace} (below) shows that 
%		
	the embedding moves the points of the hyperbola $p\cdot s$ to a circumference $x^2+y^2=\sfrac{1}{(ps)^2}$ describing a locus of the points equidistant as $\sfrac{1}{ps}$ from the origin of the axes $(0,0)$.}
%\end{example}
%
The transformation required for running the $k$NN algorithm preserves the golden ranking.

\begin{lemma}
	\label{lem:transfspace}
	The set of points having the product $ps$ at least $k\in[0,1]$ corresponds to the set of $t$-transformed points with distance 
	at least $1/k$ from the origin.
\end{lemma}

\noindent
\textbf{Approximate-Ranking Trace Embedder.}\label{subsec:ate}
Ranking optimality comes at the cost of a brute-force recomputation of $\goldenrank$ for each trace $\logtrace$ to align. Alternatively, we might avoid the brute-force cost by
%Each embedding $\phi$ entails an associated similarity metric $k_\phi$ (\S\ref{subsec:katk}) and an associated
%distance $d_{k_\phi}$ (Equation \ref{eq:dofk}), hence 
%we can compute 
computing the embeddings for all the \unravelled\ traces
before the top-$k$ search ensuring that they are independent of the trace to align. %, avoiding the brute-force cost. 
%As previously observed, 
This computational gain comes with a loss in precision, that our embedding proposal tries to mitigate by overcoming some of the current literature shortcomings: 
%; the generation of precise embeddings for graph data with loops is 
%NP-complete \cite{GartnerFW03}. %and, in its approximated version, is unable to accurately represent data using low-dimensional 
%%vectors \cite{Seshadhri5631}. 
%Our proposed embedding ($\gorgembed$) is thus weakly-ideal (\S\ref{subsec:katk}).
%%
%$\gorgembed$ is a variant of the embedding $\trembed$ from \cite{LodhiSSCW02}, which addresses some of its shortcomings.
%Indeed, $\trembed$
%\begin{alphalist}
%	\item is not weakly-ideal, so we cannot numerically assess if two embeddings represent equivalent traces 
%	(Example \ref{ex:wheredotiszero});
%	\item does not characterize $\tau$-moves, so the probabilities of the initial and final $\tau$-moves are not preserved; and
%	\item is affected by numerical errors from finite arithmetics: longer traces $\nonlogtrace$ generated from skewed probability 
%	distributions $G.\Lambda^i$ yield greater truncation errors, as smaller $\lambda^i$ components for bigger 
%	$i<|\nonlogtrace|$ are ignored, preventing a complete numerical vector characterization of  $\nonlogtrace$ in practice.
%\end{alphalist}
%%
%To overcome these shortcomings we 
	 propose a weakly-ideal embedding
 preserving probabilities from and to $\tau$ transitions, and 
 mitigating the numerical truncation errors induced by trace length and probability distribution skewness through two 
sub-embedding strategies, $\epsilon$ for transition correlations in $\tasks^2$ and $\nu$ for transition label frequency in $\tasks$\cite{Bergami21}. 
%

In order to meet the goal, we need to first transform a SWN into a TG as sketched in \S\ref{sec:was}: first, for each model trace $\nonlogtrace$ we need to restrict the TG into a \textit{weighted} TG $G_\nonlogtrace=(TG_{\nonlogtrace},\omega_\nonlogtrace)$, where $TG_{\nonlogtrace}$ contains only the nodes generating $\nonlogtrace$, thus restricting the associated matrices $L$ (and $R$) into $L_\nonlogtrace$ (and $R_\nonlogtrace$). $\omega_\nonlogtrace$ is then exploited to preserving probabilities from initial (and to final) $\tau$ transitions. Such $\omega$ can be computed as $\omega := 1-\prod_{\Ind{0}\cdots\Ind{n}\in\seqs{\sigma'}{ \closed{G}}}\Big(1-(\textit{ifte}(\mathbf{1}_{L_{\sigma'}(\Ind{0})=\tau},[R_{\sigma'}]_{\Ind{0}\Ind{1}})\textit{ifte}(\mathbf{1}_{L_{\sigma'}(\Ind{n})=\tau},[R_{\sigma'}]_{\Ind{n-1}\Ind{n}})\Big)$
where $\textit{ifte}(x,y):=x(y-1)+1$ returns $y$ if $x=1$ and $1$ otherwise. %We denote the set of all the $\closed{G}_\nonlogtrace$ as $\TBf{\pmin}{4}(\closed{G})$.
%The graph weight $\omega$ derives from the outgoing edges of the initial node and the ingoing edges of the accepting node; 
%such nodes are labeled as $\tau$. %Since the embedding strategy from \cite{LodhiSSCW02} considers only visible 
%transitions, and the trace extraction process discards the $\tau$ information, we use $\omega$ to preserve such information.
%We call the pair consisting of a transition graph and a graph weight a \emph{weighted transition graph}.
%\begin{example}\label{ex:neue}
%\end{aligned}$$
Table \ref{tab:proj} shows the projected transition graphs associated to  traces from Example \ref{ex:rankingTaus}, where  all the $\tau$-labeled nodes are removed as required.
%\end{example}
%	
%
%
%Since a trace embedding for $\nonlogtrace$ representing the transitions in $\closed{\tg}.\Lambda$ requires an intermediate 
%TG representation, each $\nonlogtrace$ is mapped to a weighted TG $(\tg_\nonlogtrace,\omega)$:
%%
%\begin{definition}[trace projection]
%	Let $\closed{G}=(V,s,t,L,R)$ be a $\tau$-closed TG, and $\pmin\in(0,1]$. The $\closed{G}$ \emph{projection} over the trace 
%	$\nonlogtrace$ is the weighted TG $(\closed{G}_\nonlogtrace,\omega)$, where $\closed{G}_\nonlogtrace$ is the TG 
%	$(V_\nonlogtrace,s,t,L_\nonlogtrace,R_\nonlogtrace)$ where
%	$V_\nonlogtrace$ contains the nodes generating $\nonlogtrace$ from $\closed{G}$	
%	(i.e., $\seqs{\sigma'}{ \closed{G}}$) and
%	$L_\nonlogtrace$ and $R_\nonlogtrace$ are the restrictions of $L$ and $R$ to $V_\nonlogtrace$; and
%	$\omega := 1-\prod_{\Ind{0}\cdots\Ind{n}\in\seqs{\sigma'}{ \closed{G}}}\Big(1-(\textit{ifte}(\mathbf{1}_{L_{\sigma'}(\Ind{0})=\tau},[R_{\sigma'}]_{\Ind{0}\Ind{1}})\textit{ifte}(\mathbf{1}_{L_{\sigma'}(\Ind{n})=\tau},[R_{\sigma'}]_{\Ind{n-1}\Ind{n}})\Big)$
%	where $\textit{ifte}(x,y):=x(y-1)+1$ returns $y$ if $x=1$ and $1$ otherwise. We denote the set of all the $\closed{G}_\nonlogtrace$ as $\TBf{\pmin}{4}(\closed{G})$.%\todo{need to understand this}
%	%\dots
%	%		
%	%		\dots we generate a set $\TBf{\pmin}{n}(\closed{G})$ of projected TGs $P$ for each trace in $\WCal{\pmin}{n}(P)$ as follows: for each weighted trace $\braket{\nonlogtrace,\probskip{\nonlogtrace}}\in\WCal{\pmin}{n}(P)$ generated from a path $\pi_\nonlogtrace=s\to n_2\rightsquigarrow n_m\to t$ over $R$, we generate a TG $P_\nonlogtrace=(s',t',L_\nonlogtrace,R_\nonlogtrace,\omega')$, where \begin{inparaenum}[\it (i)]
%	%			\item $s'=s$ if $\textit{label}(s)\neq \tau$ and $t'=n_2$ otherwise,
%	%			\item $t'=t$ if $\textit{label}(t)\neq \tau$ and $t'=n_m$ otherwise,
%	%			\item $L_\nonlogtrace$ (and $R_\nonlogtrace$) is the submatrix of $L$ (and $R$) over the non-$\tau$ labeled notes in $\pi_\nonlogtrace$ and the labels from $\nonlogtrace$,
%	%			\item $\omega'$ is initialized by $\omega$ and then multiplied by $[R]_{s,n_2}$ (and also $[R]_{n_m,t}$) if $\textit{label}(s)=\tau$ (and  $\textit{label}(t)=\tau$).
%	%		\end{inparaenum}
%\end{definition}
%	
%\begin{table}[!t]
%	\caption{Projections of $\tg$ over $\net$-traces of length $4$.}\label{tab:proj}
%	\centering
%	\resizebox{.3\textwidth}{!}{\begin{tabular}{>{\centering\arraybackslash} m{1cm}| >{\centering\arraybackslash} m{4cm} >{\centering\arraybackslash} m{1cm} >{\centering\arraybackslash} m{1cm} }
%			\toprule
%			$\nonlogtrace$&$G_\nonlogtrace$&$l$&$\omega$\\
%			\midrule
%			$\const{a}$ & \includegraphics{images/trace_a} & $1$ & $\color{violet}\pa\pf$\\
%			$\const{cb}$ & \includegraphics{images/trace_cb} & $2$ & $\color{violet}\pb$\\
%			$\const{aaa}$ & \includegraphics{images/trace_a_loop} & $3$ & $\color{violet}\pa\pf$\\
%			$\const{caa}$ & \includegraphics{images/trace_caa} & $3$ & $\color{violet}\pb\pf$\\
%			%			\bottomrule
%			%	\end{tabular}}\qquad \qquad 	
%			%	\resizebox{.3\textwidth}{!}{\begin{tabular}{>{\centering\arraybackslash} m{1cm}| >{\centering\arraybackslash} m{4cm} >{\centering\arraybackslash} m{1cm} >{\centering\arraybackslash} m{1cm} }
%			%	\toprule
%			%	$\nonlogtrace$&$G_\nonlogtrace$&$l$&$\omega$\\
%			%	\midrule
%			$\const{aa}$ & \includegraphics{images/trace_aa} & $2$ & $\color{violet}\pa\pf$\\
%			$\const{ca}$ & \includegraphics{images/trace_ca} & $2$ & $\color{violet}\pb\pf$\\
%			\begin{tabular}{l}aaaa\end{tabular} & \includegraphics{images/trace_a_loop} & $4$ & $\color{violet}\pa\pf$\\
%			$\const{caaa}$ & \includegraphics{images/trace_ca_loop} & $4$ & $\color{violet}\pb\pf$\\
%			\bottomrule
%	\end{tabular}}
%	%\vspace{-0.2cm}
%\end{table}
\begin{table}[!t]
	\caption{Projections over $\net$-traces of length $4$.}\label{tab:proj}
	\centering
	\resizebox{.3\textwidth}{!}{\begin{tabular}{>{\centering\arraybackslash} m{1cm}| >{\centering\arraybackslash} m{4cm} >{\centering\arraybackslash} m{1cm} >{\centering\arraybackslash} m{1cm} }
			\toprule
			$\nonlogtrace$&$TG_\nonlogtrace$&$l$&$\omega_\nonlogtrace$\\
			\midrule
			$\const{a}$ & \includegraphics{images/trace_a} & $1$ & $\color{violet}\pa\pf$\\
			$\const{cb}$ & \includegraphics{images/trace_cb} & $2$ & $\color{violet}\pb$\\
			$\const{aaa}$ & \includegraphics{images/trace_a_loop} & $3$ & $\color{violet}\pa\pf$\\
			$\const{caa}$ & \includegraphics{images/trace_caa} & $3$ & $\color{violet}\pb\pf$\\
			\bottomrule
	\end{tabular}}\qquad \qquad 	
	\resizebox{.3\textwidth}{!}{\begin{tabular}{>{\centering\arraybackslash} m{1cm}| >{\centering\arraybackslash} m{4cm} >{\centering\arraybackslash} m{1cm} >{\centering\arraybackslash} m{1cm} }
			\toprule
			$\nonlogtrace$&$TG_\nonlogtrace$&$l$&$\omega_\nonlogtrace$\\
			\midrule
			$\const{aa}$ & \includegraphics{images/trace_aa} & $2$ & $\color{violet}\pa\pf$\\
			$\const{ca}$ & \includegraphics{images/trace_ca} & $2$ & $\color{violet}\pb\pf$\\
			\begin{tabular}{l}aaaa\end{tabular} & \includegraphics{images/trace_a_loop} & $4$ & $\color{violet}\pa\pf$\\
			$\const{caaa}$ & \includegraphics{images/trace_ca_loop} & $4$ & $\color{violet}\pb\pf$\\
			\bottomrule
	\end{tabular}}
	\vspace{-0.2cm}
\end{table}
Our proposed embedding $\gorgembed$ \cite{Bergami21} can be now exploited for each  $G_\nonlogtrace$ as follows:
%Our proposed embedding $\gorgembed$ is computed for each $wTG_\nonlogtrace$. The goal is to use
%$k_{\gorgembed}$ for ranking all the traces generated by \unravelling\ via such graphs. We extend $\trembed$ %from 
%%\cite{LodhiSSCW02} including the associated probabilities, and
%by making the ranking induced by $k_{\gorgembed}$ the inverse of 
%that induced by the
%sum of the following distances: the transition correlations $\epsilon$ and the transition label frequency $\nu$.
%We need that the desired properties of $\gorgembed$ are independent of the characterization of $\epsilon$ over the 
%$2$-grams in $\tasks^2$ and $\nu$ over the labels in $\tasks$. %which  provide different embedding strategies. Therefore, our 
%%$\gorgembed$ embedding is defined for any weighted TG.


%\begin{table}[!t]
%	\centering
%	\caption{Embedding representation for the TG $P$ in \figurename~\ref{fig:closed} and the trace $\logtrace=\textup{caba}$ after representing it as in \figurename~\ref{fig:sigmastar}. Please note that we restrict $\trace_\tau^2$ to the one from $P$.}\label{tab:emb1}
%		\begin{tabular}{l|l|l|l|l|l|l|}
%	\toprule
%	& a    & b                                                   & c    & aa   & ca   & cb   \\
%	\midrule
%	$\gorgembed(P)$ & $9.94\cdot10^{-25}$ & $1.18\cdot 10^{-26}$ & $1.04\cdot10^{-25}$ & $4.45\cdot 10^{-25}$ & $6.22\cdot10^{-25}$ & $8.29\cdot10^{-26}$\\
%	$\gorgembed(P_{\logtrace})$ & $8.16\cdot10^{-17}$ & $4.08\cdot 10^{-17}$ & $4.08\cdot10^{-17}$ & $4.37\cdot 10^{-17}$ & $1.03\cdot10^{-16}$ & $4.37\cdot10^{-17}$\\
%	\bottomrule
%\end{tabular}
%\end{table}
\begin{definition}[TG-Embedding]\label{def:ppne}
	%Given a finite set of non-empty labels $\trace_\tau =\trace\backslash\{\tau\}$, $\trace_\tau^2$ denotes all the possible pair of labels associated to paths ${\color{green}\alpha}\rightsquigarrow{\color{green}\beta}$ and $\trace_\tau$ denotes the set of all the possible non-$\tau$ node labels. Therefore, it is always possible to enumerate $\trace_\tau^2\cup\trace_\tau$ via an enumeration by a bijection $\iota\colon \trace_\tau^2\cup\trace_\tau\to  N$, where $N\subset \mathbb{N}_{\neq 0}$ and $\max N=|N|$.
	Given a weighted TG $G_\nonlogtrace=(TG_{\nonlogtrace},\omega_\nonlogtrace)$ and a tuning parameter $t_f\in[0,1]$, the \emph{TG-Embedding} is 
	$$\gorgembed_{i}(G_\nonlogtrace):=\begin{cases}
	\omega_\nonlogtrace \frac{\epsilon_\const{ab}(TG_{\nonlogtrace})}{\|\epsilon\|_2}\;t_f^{|R>0|}\, & {i}=\const{ab}\in\tasks^2\\
	\frac{\nu_\const{a}(TG_{\nonlogtrace})}{\|\nu\|_2}\;\;\;\,t_f^{|R>0|}\, & {i}=\const{a}\in\tasks\\
	\end{cases}$$
	where $\nu$ and $\epsilon$ represent the embeddings associated to $L$ and $L,R$: $\epsilon$  returns 
	$\epsilon_\const{ab}=0$ for the $2$-grams $\const{ab}$ not represented by the TG, and either $\nu$ always returns the empty 
	vector or $\nu_\const{a}(G)=0$ iff the labels $\const{a}\in\tasks$ that are associated to no vertex in $V$.
\end{definition}
%
Here, ${\max\arg}_{\nonlogtrace\in \WCal{\pmin}{n}, G_\nonlogtrace\in\TBf{p}{n}(P)} k_{\gorgembed}(G_\logtrace, G_{\nonlogtrace})$ returns the best approximated trace alignment for a log trace represented as $G_{\logtrace}$. %\xout{Similarly, we can provide the TG $P\in\mathbf{P}$ providing the best approximated alignment for $P_{\logtrace}$ as $\underset{P}{\max\arg}\underset{ P_\trace\in\mathbf{P}_p^n(P)}{\max} k_{\gorgembed}(P_\trace, P_{\logtrace})$.}¯
%
\begin{table}[!t]
	\caption{Different sub-embeddings ($\epsilon^1$, $\epsilon^2$, $\nu^1$, and $\nu^2$) for $\gorgembed$.}\label{tab:embedstrat}
	\centering
	\resizebox{.49\textwidth}{!}
	{\begin{tabular}{c|c|c}
			\toprule
			& $x=1$ & $x=2$ \\
			\midrule
			$\epsilon^x_\const{ab}(G):=$ & $\label{eq:epsilon}
			\sum_{i=1}^l{\lambda^i}\frac{[LR^iL^t]_\const{ab}}{\sum_{\const{a'b'}\in\tasks^2}R^i_\const{a'b'}}$ & $
			\sum_{i=1}^l\lambda^i[\Lambda^i]_\const{ab}$\\
			$\nu^x_\const{a}(G):=$ & $\frac{1}{c}\sum_{\nonlogtrace'\in \ptraces{G}{0}}\frac{|\Set{\nonlogtrace_i'\in\nonlogtrace'|\const{a}\in\tasks\wedge \nonlogtrace'_i=\const{a}}|}{|\nonlogtrace'|}$ & $0$ \\
			\bottomrule
	\end{tabular}}
\end{table}
%
We choose two possible interchangeable definitions for $\nu$ and $\epsilon$ shown in Table \ref{tab:embedstrat}, where $l$ 
is the path length, and $c$ for $\nu^1$ is a normalization factor such that $\sum_{\const{a}\in\tasks}\nu^1_\const{a}(P)=1$; 
$\nu^2$ completely ignores the label frequency contribution; $\epsilon^2$ is the embedding $\trembed$ from \S\ref{subsec:katk};
and $\epsilon^1$ and $\epsilon^2$ only differ from the normalization. $t_f\in [0,1]$ and $\lambda\in (0,1]$ are tuning 
parameters that can be inferred from the available data. %\cite{DriessensRG06}. 
The latter describes the decay factor, while $t_f$
represents the relevance of our embedding representation as the number of edges within $\closed{G}_\nonlogtrace$ increases.
We choose $t_f=0.0001$ and $\lambda=0.07$.
%
This representation is independent of the representation of the trace to be aligned and does not have to be 
recomputed for each alignment.



%When the transition matrix is ergodic \cite{StocasticCC},  the transition matrix embedding converges to $\epsilon(R)_{\color{green}\alpha\beta}=[(\mathbf{I}-\lambda\Lambda)^{-1}]_{\color{green}\alpha\beta}$ \cite{GartnerFW03} for $n\to+\infty$.

\begin{table*}[!t]
	\centering
	\caption{Embeddings for $\net$-traces of maximum length $4$ and $\logtrace=\const{caba}$.}\label{tab:emb2}\label{tab:embsitar}
	\resizebox{\textwidth}{!}{\begin{tabular}{lllllllllllll}
			\toprule
			& $\const{a}$    & $\const{b}$                                                   & $\const{c}$    & $\const{aa}$  & $\const{ab}$   & $\const{ac}$   & $\const{ba}$  & $\const{bb}$   & $\const{bc}$   & $\const{ca}$   & $\const{cb}$  & $\const{cc}$  \\
			\midrule		
			$\gorgembed(TG_\const{aaaa})$ & $1.00\cdot 10^{-24}$ & $0$   & $0$   & $6.44\cdot 10^{-26}$ & $0$& $0$& $0$& $0$& $0$    &$0$& $0$& $0$ \\
			$\gorgembed(\overline{G}_\const{aaa})$  & $1.00\cdot 10^{-24}$ & $0$   & $0$   & $1.29\cdot 10^{-25}$ & $0$& $0$& $0$& $0$& $0$    &$0$& $0$& $0$ \\
			$\gorgembed(\overline{G}_\const{aa})$   & $1.00\cdot 10^{-24}$ & $0$   & $0$   & $2.57\cdot 10^{-25}$ & $0$& $0$& $0$& $0$& $0$    &$0$& $0$& $0$ \\
			$\gorgembed(\overline{G}_\const{a})$    & $1.00\cdot 10^{-4}$  & $0$   & $0$   & $0$   & $0$& $0$& $0$& $0$& $0$    &$0$& $0$& $0$ \\
			$\gorgembed(\overline{G}_\const{caa})$  & $7.07\cdot 10^{-25}$ & $0$   & $7.07\cdot 10^{-25}$ & $1.46\cdot 10^{-25}$ & $0$& $0$& $0$& $0$& $0$    &$2.05\cdot 10^{-25}$& $0$& $0$ \\
			$\gorgembed(\overline{G}_\const{ca})$   & $7.07\cdot 10^{-25}$ & $0$   & $7.07\cdot 10^{-25}$ & $0$   & $0$& $0$& $0$& $0$& $0$    &$1.00\cdot 10^{-8}$& $0$& $0$ \\
			$\gorgembed(\overline{G}_\const{cb})$   & $0$   & $7.07\cdot 10^{-25}$ & $7.07\cdot 10^{-25}$ & $0$   & $0$& $0$& $0$& $0$& $0$    &$0$ & $4.29\cdot 10^{-9}$& $0$ \\
			$\gorgembed(\overline{G}_\const{caaa})$ & $7.07\cdot 10^{-25}$ & $0$   & $7.07\cdot 10^{-25}$ & $1.03\cdot 10^{-25}$ & $0$& $0$& $0$& $0$& $0$    &$7.20\cdot 10^{-26}$ & $0$& $0$ \\
			\bottomrule
			$\gorgembed(\overline{G}_\const{caba})$ & $8.16\cdot10^{-17}$ & $4.08\cdot 10^{-17}$ & $4.08\cdot10^{-17}$ & $4.37\cdot 10^{-17}$ &  $0$& $0$& $0$ & $0$ & $0$    & $1.03\cdot10^{-16}$ & $4.37\cdot10^{-17}$& $0$ \\
			\bottomrule
	\end{tabular}}
	
\end{table*}
\begin{example} 
	%\small \label{ex:withpaths} %After generating all the TGs for the \unravelled\ traces (Example \ref{ex:neue}), we further associate a sub-embedding $\epsilon$ using the $2$-grams included in the model traces.\footnoteref{fn:caveat}
	%Given $\nonlogtrace_\tau=\Set{a,b,c}$, the embedding space is of size $6$: three features (computed using $\nu$) correspond to the labels of the non-$\tau$ vertices in $\nonlogtrace_\tau$, i.e., $\{a,b,c\}$, and other three features (computed using $\epsilon$) correspond to the $2$-grams subsequences that are also traces in $W^4_0(P)$, i.e., $\{aa,ca,cb\}$. Therefore, $\{a,b,c,aa,ca,cb\}\subset \nonlogtrace_\tau^2\cup\nonlogtrace_\tau$ is the whole set of features describing both the label and the transition matrix, so $\gorgembed$ is a vector with 6 dimensions.
	%\xout{The embedding associated to $P$ is described in Table \ref{tab:emb1} as $\gorgembed(P)$: it shows that doing ${\color{green}a}\rightsquigarrow{\color{green}a}$ is more probable than doing  ${\color{green}c}\rightsquigarrow{\color{green}a}$. Also, given that both the probability of performing ${\color{green}c}\overset{1}{\rightsquigarrow}{\color{green}b}$ is relatively low and trace $\color{green}cb$ is relatively infrequent, ${\color{green}c}{\rightsquigarrow}{\color{green}b}$ is less probable than any other subtrace. If we now consider the single nodes, $\color{green}c$ shares a subset of traces with $\color{green}a$ where $\color{green}a$ is more frequent than $\color{green}c$, and therefore the score of the former is higher than the one of the latter. Also, the score associated to the single node $\color{green}b$ is lower than the one for the single node $\color{green}c$ because $\color{green}b$ is less frequent and appears in less probable traces than $\color{green}c$: in particular, $\color{green}c$ appears in \textit{ca}, which is more probable than \textit{cb}.}
	Table \ref{tab:emb2} shows the embeddings $\gorgembed(\overline{G}_\nonlogtrace)$ generated from Table \ref{tab:proj}, where the $l=|\nonlogtrace|$ for each \unravelled trace.
	After representing trace $\logtrace=\const{caba}$ as a sequence graph, we find its
	embedding $\gorgembed(\overline{G}_{\logtrace})$ with strategies $\epsilon^1$ and $\nu^1$ 
	{as} in Table \ref{tab:embsitar}: $\const{a}$ is the most frequent label and $\const{b}$ and $\const{c}$ are equiprobable. 
	The $2$-gram $\const{ca}$ appears twice in the trace set and is more frequent than other $2$-grams.
\end{example}
%
These sub-embeddings satisfy the conditions required by the G-Embedding.
\begin{lemma}
	\label{lem:addedForOurPropos}
	The sub-embeddings in Table~\ref{tab:embedstrat} satisfy the requirements from Definition~\ref{def:ppne}.
\end{lemma}


The kernel $k_{\gorgembed}$ associated to $\gorgembed$ is {a function of the distance  $\|\hat{\epsilon}(G)-\hat{\epsilon}(G')\|_2^2$ and $\|\hat{\nu}(G)-\hat{\nu}(G')\|_2^2$ for traces $\logtrace$ and ${\nonlogtrace}$.}
%

\begin{proposition}
	\label{lem:rewritinglemma}
	Given {two weighted TGs} $(G,\omega)$ and $(G',\omega')$ with $G=(s,t,L,R)$ and $G'=(s',t',L',R')$, the definition of
	$k_\gorgembed$ is expanded to
	$\begin{aligned}
	%k_{\gorgembed}(G,G')=&
	\omega\omega't_f^{|R>0|+|R'>0|}\left(1-\frac{\norm{\hat{\epsilon}(G)-\hat{\epsilon}(G')}{2}^2}{2}\right)+\\t_f^{|R>0|+|R'>0|}\left(1-\frac{\norm{\hat{\nu}(G)-\hat{\nu}(G')}{2}^2}{2}\right)
	\end{aligned}$
\end{proposition}

When $\hat{\epsilon}(G)$ and $\hat{\epsilon}(G')$ are affected by truncation errors 
(i.e., $\norm{\hat{\epsilon}(G)-\hat{\epsilon}(G')}{2}^2\to 0$), the $\nu$ strategy intervenes as a backup ranking. The first 
term of the sum does not affect the ranking, as it reduces to a constant factor.

%\xout{Given that we can now follow Definition \ref{def:ppne} for representing a trace $\trace$ as a proper embedding after transforming it as a TG $P_{\logtrace}$ (\S\ref{subsec:katk}), we can find the TG $P$ providing the best approximate match with  a trace $\trace$ as follows:}
%\[\Rcancel{\underset{{P}}{\max\arg}\;k_{\gorgembed}(P,T)}\]
%\xout{Still, this TG matching strategy does not allow to find the trace maximizing such score.} %To assess such problem, the next section is going to determine both an exact (\S\ref{subsec:exbkptap}) and an approximated strategy (\S\ref{subsec:akptap}) for probabilistically matching one single trace from the TG.
%
%\xout{Given the characterization of a TG as in \S\ref{subsec:ppn} and the embedding strategy proposed in Definition \ref{def:ppne}, We can \ADD{now} generate an embedding for each possible weighted trace $\braket{\trace,\probskip{\trace}}\in\mathcal{W}_p^n(P)$ for a given TG $P$ as described in the following definition:}





\begin{table}[!t]
	\vspace{+0.9mm}
	\caption{Comparison between the optimal ranking $\goldenrank$ and the kernel $k_{\gorgembed}$ with embedding strategies $\epsilon^1$ and $\nu^1$: arrows $\boldsymbol{\downarrow}$ remark the column of choice under which we sort the rows.}\label{tab:rank3}
	\centering
	%	\begin{tabular}{l|c|ll}
	%		\toprule
	%		$\trace$ & $k_{\gorgembed}(\trace,\logtrace)$ & \textit{kernel ranking} & expected ranking\\
	%		\midrule
	%		a & $8.16\cdot 10^{-21}$ & \textbf{1} & \textbf{\color{blue}1}\\
	%		ca & $1.89\cdot 10^{-24}$ & \textbf{2} & \textbf{\color{blue}4}\\
	%		cb & $7.64\cdot 10^{-25}$ & \textbf{3} & \textbf{\color{blue}5}\\
	%		caa & $1.14\cdot 10^{-40}$ & \textbf{4} & \textbf{\color{blue}7}\\
	%		caaa & $9.84\cdot 10^{-41}$ & \textbf{5} & \textbf{\color{blue}8}\\
	%		aa & $9.28\cdot 10^{-41}$ & \textbf{6} & \textbf{\color{red}2}\\
	%		aaa & $8.72\cdot 10^{-41}$ & \textbf{7} & \textbf{\color{red}3}\\
	%		aaaa & $8.44\cdot 10^{-41}$ & \textbf{8} & \textbf{\color{red}6}\\
	%		
	%		\bottomrule
	%	\end{tabular}
	
	\resizebox{\columnwidth}{!}{\begin{tabular}{l|ll|cc}
			\toprule
			
			{$\nonlogtrace$} &
			%\multirow{2}{*}{$d(\trace,\logtrace)$} &
			%\multicolumn{2}{c|}{$\mu_{\logtrace}$} &
			$( \probskip{\nonlogtrace}$ &  $,\,\boldsymbol{\downarrow} s_d(\logtrace,\nonlogtrace)) $ &
			{$=\goldenrank(\logtrace,\nonlogtrace)$} &
			{$k_{\gorgembed}(\closed{G}_\logtrace,\closed{G}_{\nonlogtrace})$} \\
			
			
			\midrule
			$\const{caa}$  & $0.035$ & $\;\; 0.8333$ & $0.0292$ & $1.14\cdot 10^{-40}$\\
			$\const{caaa}$  &  $0.0175$ & $\;\; 0.8333$ & $0.0145$ & $9.84\cdot 10^{-41}$\\
			$\const{a}$  & $0.4$ & $\;\; 0.6250$  & $0.2500$ & $8.16\cdot 10^{-21}$ \\
			$\const{aaaa}$  & $0.05$ & $\;\; 0.6250$ & $0.0357$ & $8.44\cdot 10^{-41}$\\
			$\const{aa}$  & $0.2$ & $\;\; 0.7142$ & $0.1428$ & $9.28\cdot 10^{-41}$ \\
			$\const{aaa}$  & $0.1$ & $\;\; 0.7142$ & $0.0714$ & $8.72\cdot 10^{-41}$\\
			$\const{ca}$  &  $0.07$ & $\;\; 0.7142$ & $0.0500$ & $1.89\cdot 10^{-24}$\\
			$\const{cb}$  &  $0.06$ & $\;\; 0.7142$ & $0.0428$ & $7.64\cdot 10^{-25}$\\
			\bottomrule
		\end{tabular}\ \begin{tabular}{l|c}
			\toprule
			
			{$\nonlogtrace$} &
			{$\boldsymbol{\downarrow}\goldenrank(\logtrace,\nonlogtrace)$} \\
			
			
			\midrule
			$\const{a}$  &  $0.2500$ \\
			$\const{aa}$  &  $0.1428$  \\
			$\const{aaa}$  & $0.0714$ \\
			$\const{ca}$  &   $0.0500$\\
			$\const{cb}$  & $0.0428$ \\
			$\const{aaaa}$  &  $0.0357$ \\
			$\const{caa}$  &  $0.0292$ \\
			$\const{caaa}$  &   $0.0145$ \\
			\bottomrule
		\end{tabular}\	\begin{tabular}{l|c}
			\toprule
			
			{$\nonlogtrace$} &
			{$\boldsymbol{\downarrow}k_{\gorgembed}(\closed{G}_\logtrace,\closed{G}_{\nonlogtrace})$} \\
			
			
			\midrule
			$\const{a}$  & $8.16\cdot 10^{-21}$ \\
			$\const{ca}$  &   $1.89\cdot 10^{-24}$\\
			$\const{cb}$  &   $7.64\cdot 10^{-25}$\\
			$\const{caa}$  &$1.14\cdot 10^{-40}$\\
			$\const{caaa}$  &  $9.84\cdot 10^{-41}$\\
			$\const{aa}$  &  $9.28\cdot 10^{-41}$ \\
			$\const{aaaa}$  & $8.44\cdot 10^{-41}$\\
			$\const{aaa}$  &  $8.72\cdot 10^{-41}$\\
			\bottomrule
	\end{tabular}}
\end{table}




\begin{example}%\small \label{ex:11}
	The products $k_{\gorgembed}(\logtrace,\nonlogtrace)=\braket{\gorgembed(\overline{G}_\logtrace),\;\gorgembed(\overline{G}_{\nonlogtrace})}$ 
	with sub-embedding $\nu^1$ and $\epsilon^1$, for each trace $\nonlogtrace$ appear in 
	Table \ref{tab:rank3} along optimal ranking $\goldenrank$. $k_{\gorgembed}$ approximates the optimal ranking as it tends to rank the transition graphs $\closed{G}_\nonlogtrace$ (generated from $\overline{G}$ via projection) similarly to the traces over $\mathcal{R}$.
\end{example}
%
%\begin{example}\label{ex:moreskew}
%	\xout{Let us suppose to change the probability distribution associated with the $P$'s edges, so that it becomes more skewed and that some traces are relatively more probable than others. Let us set $p_1=p_2=0.5$, $p_3=0.9$, $p_6=0.1$, $p_4=0.3$, and $p_5=0.7$, so that the initial choice is equiprobable but performing a loop is more probable than terminating the path. We keep the other tuning parameters as in Example \ref{ex:withpaths}. In this case, we generate the following set of weighted traces:}
%	$$\begin{aligned}
%	\Rcancel{\mathcal{W}_0^4(P)=\{}&\Rcancel{\braket{cb,0.35},\braket{a,0.05},\braket{aa,0.045},\braket{aaa,0.0405},}\\
%	&\Rcancel{\braket{aaaa,0.03645},\braket{ca,0.015},\braket{caa,0.0135},\braket{caaa,0.01215}\}}\\
%	\end{aligned}$$
%	\xout{Let us also assume that we want to align these traces in a probabilistic way with the query $\logtrace=\textup{caba}$: the distance ($d$) and similarity ($s_d$) scores will be still the same, while the associated probabilities will vary. The expected ranking by multiplying weight with similarity is represented in Table \ref{tab:witherror}.}
%	
%	%\xout{As a consequence of the different probability distribution associated to the edges, a different set of embedding will be generated for each trace of interest while the TG $T$ associated to $\logtrace$ will be kept the same. Table \ref{tab:witherror} represents the ranking induced by the kernel $k_{\gorgembed}$ over this different set of vectors by ranking the traces in descendant order of $k_{\gorgembed}$. As we might notice, the more skewed edge probability distribution introduced more errors in the ranking result: while the largest ranking subsequence (marked in blue) always starts from the best-expected trace \textit{cb}, this element now appears in the third position, and the position of traces \textit{caa} and \textit{aaaa} is swapped.}
%	
%\end{example}
%\begin{table}[!t]
%	\centering
%	\caption{Expected ranking of the paths from Example \ref{ex:moreskew} with the trace $\logtrace=\textup{caba}$. The cost function is the one from \cite{LeoniM17} and its normalized similarity score has $c=5$. Traces are ranked by decreasing kernel $k_{\gorgembed}$ value: slight changes in the expected expected order are circled, the others are marked in red.}\label{tab:witherror}
%	\begin{tabular}{lc|ll|cc|l}
%		\toprule
%		
%		\multirow{2}{*}{$\trace$} &
%		\multirow{2}{*}{$d(\trace,\logtrace)$} &
%		\multicolumn{2}{c|}{$\mu_{\logtrace}$} &
%		\multirow{2}{*}{$\approx s_d(\trace,\logtrace)\cdot \probskip{\trace}$} &
%		\multirow{2}{*}{$k_{\gorgembed}(\trace,\logtrace)$}&
%		\multirow{2}{*}{\textit{expected ranking}}\\
%		
%		\cline{3-4} &&  $\langle \probskip{\trace}$ &  $,\,s_d(\trace,\logtrace)\rangle $ && \\
%		
%		\midrule
%		{a}  & $3$ & $0.05$ & $\;\; 0.6250$  & $0.03125$ & $8.16497\cdot 10^{-16}$ & \textbf{\color{red}3}\\
%		{ca}  & $2$ & $0.015$ & $\;\; 0.7142$ & $0.01071$ & $1.30623\cdot 10^{-18}$ & \textbf{\color{red}7}\\
%		{cb}  & $2$ & $0.35$ & $\;\; 0.7142$ & $0.25000$ & $1.01399\cdot10^{-18}$ & \textbf{\color{blue}1}\\
%		{aa}  & $2$ & $0.045$ & $\;\; 0.7142$ & $0.03214$ & $1.01894\cdot10^{-30}$ & \textbf{\color{blue}2}\\
%		{aaa}  & $2$ & $0.0405$ & $\;\; 0.7142$ & $0.02893$ & $9.98696\cdot10^{-31}$ & \textbf{\color{blue}4}\\
%		{caa}  & $1$ & $0.0135$ & $\;\; 0.8333$ & $0.01125$ & $9.96052\cdot10^{-31}$ & \textbf{\color{blue}\ding{177}}\\
%		{aaaa}  & $3$ & $0.03645$ & $\;\; 0.7142$ & $0.02603$ & $9.80476\cdot10^{-31}$ & \textbf{\color{blue}\ding{176}}\\
%		{caaa}  & $1$  & $0.01215$ & $\;\; 0.8333$ & $0.01012$ & $9.52398\cdot 10^{-31}$ & \textbf{\color{blue}8}\\
%		\bottomrule
%	\end{tabular}
%\end{table}
%\begin{table}[!t]
%	\caption{Comparison between the ranking induced by the expected ranking $\goldenrank$ and the proposed kernel $k_{\gorgembed}$ with embedding strategies $\epsilon^2$ and $\nu^1$: arrows $\boldsymbol{\downarrow}$ remark the column of choice under which we sort the rows (i.e., ranking).}\label{tab:compLit}
%	\centering
%	\resizebox{.9\textwidth}{!}{\begin{tabular}{l|ll|cc}
%			\toprule
%			
%			{$\trace$} &
%			%\multirow{2}{*}{$d(\trace,\logtrace)$} &
%			%\multicolumn{2}{c|}{$\mu_{\logtrace}$} &
%			$( \probskip{\trace}$ &  $,\,\boldsymbol{\downarrow} s_d(\trace,\logtrace)) $ &
%			{$=\goldenrank(\trace,\logtrace)$} &
%			{$k_{\gorgembed}(P_\trace,P_{\logtrace})$} \\
%			
%			
%			\midrule
%			$\const{caa}$  & $0.035$ & $\;\; 0.8333$ & $0.0292$ & $1.03498\cdot10^{-40}$\\
%			$\const{caaa}$  &  $0.0175$ & $\;\; 0.8333$ & $0.0145$ & $8.94997\cdot10^{-41}$ \\
%			$\const{a}$  & $0.4$ & $\;\; 0.6250$  & $0.2500$ & $8.16497\cdot10^{-21}$\\
%			$\const{aaaa}$ & $0.05$ & $\;\; 0.6250$ & $0.0357$ & $8.20640\cdot10^{-41}$\\
%			$\const{aa}$  & $0.2$ & $\;\; 0.7142$ & $0.1428$ & $9.96007\cdot10^{-41}$ \\
%			$\const{aaa}$  & $0.1$ & $\;\; 0.7142$ & $0.0714$ & $8.41263\cdot10^{-41}$\\
%			$\const{ca}$  &  $0.07$ & $\;\; 0.7142$ & $0.0500$ & $1.45079\cdot10^{-24}$\\
%			$\const{cb}$  &  $0.06$ & $\;\; 0.7142$ & $0.0428$ & $8.52070\cdot10^{-25}$\\
%			\bottomrule
%		\end{tabular}\quad \begin{tabular}{l|c}
%			\toprule
%			
%			{$\trace$} &
%			{$\boldsymbol{\downarrow}\goldenrank(\trace,\logtrace)$} \\
%			
%			
%			\midrule
%			$\const{a}$  &  $0.2500$ \\
%			$\const{aa}$  &  $0.1428$  \\
%			$\const{aaa}$  & $0.0714$ \\
%			$\const{ca}$  &   $0.0500$\\
%			$\const{cb}$  & $0.0428$ \\
%			$\const{aaaa}$  &  $0.0357$ \\
%			$\const{caa}$  &  $0.0292$ \\
%			$\const{caaa}$  &   $0.0145$ \\
%			\bottomrule
%		\end{tabular}\quad	\begin{tabular}{l|c}
%			\toprule
%			
%			{$\trace$} &
%			{$\boldsymbol{\downarrow}k_{\gorgembed}(P_\trace,P_{\logtrace})$} \\
%			
%			
%			\midrule
%			{a}  & $8.16497\cdot10^{-21}$ \\
%			{ca}  &   $1.45079\cdot10^{-24}$\\
%			{cb}  &   $8.52070\cdot10^{-25}$\\
%			{caa}  & $1.03498\cdot10^{-40}$\\
%			{aa}  &  $9.96007\cdot10^{-41}$ \\
%			{caaa}  &  $8.94997\cdot10^{-41}$ \\
%			{aaa}  &  $8.41263\cdot10^{-41}$\\
%			{aaaa}  & $8.20640\cdot10^{-41}$\\
%			\bottomrule
%	\end{tabular}}
%	
%	\centering
%	\begin{tabular}{lc|l}
%		\toprule
%		%\multicolumn{3}{c||}{Example \ref{ex:withpaths}} %&
%		%\multicolumn{3}{c}{Example \ref{ex:moreskew}}\\
%		%\hline
%		$\trace$ &  $k_{\gorgembed}(P_\trace,P_{\logtrace})$ & $\goldenrank(\trace,\logtrace)$\\ %&
%		%$\trace$ &  $k_{\gorgembed}(\trace,\logtrace)$ & \textit{exp. ranking}\\
%		\midrule
%		
%		a & $\;8.16497\cdot10^{-21}$ & \textbf{\color{blue}1} \\%& a & $\;8.16497\cdot 10^{-21}$ & \textbf{\color{red}3} \\
%		ca & $\;1.45079\cdot10^{-24}$ & \textbf{\color{blue}4} \\%& ca &  $\;1.45079\cdot 10^{-24}$ & \textbf{\color{red}7}\\
%		cb & $\;8.52070\cdot10^{-25}$ & \textbf{\color{blue}5} \\%& cb & $\;8.52070\cdot10^{-25}$& \textbf{\color{blue}1}\\
%		caa & $\;1.03498\cdot10^{-40}$ & \textbf{\color{blue}7} \\%& aa & $\;9.29342\cdot10^{-41}$ & \textbf{\color{blue}2}\\
%		aa & $\;9.96007\cdot10^{-41}$ & \textbf{\color{red}2} \\%& caa & $\;9.18112\cdot10^{-41}$ & \textbf{\color{blue}6}\\
%		caaa & $\;8.94997\cdot10^{-41}$ & \textbf{\color{red}8} \\%& caaa & $\;8.71867\cdot10^{-41}$ & \textbf{\color{blue}8}\\
%		aaa & $\;8.41263\cdot10^{-41}$ &  \textbf{\color{red}3} \\%& aaa & $\;8.31269\cdot10^{-41}$ & \textbf{\color{red}4}\\
%		aaaa & $\;8.20640\cdot10^{-41}$ &  \textbf{\color{red}6}\\% & aaaa & $\;8.19352\cdot10^{-41}$ & \textbf{\color{red}5}\\
%		
%		\bottomrule
%	\end{tabular}
%\end{table}
%
%\begin{example}\label{ex:cmpexample}
%	Let us compare the ranking results by replacing in $\gorgembed$ the edge embedding $\epsilon^1$ with $\epsilon^2$. If we re-run the computations performed in  Example \ref{ex:11}, we obtain   Table \ref{tab:compLit}:  $\epsilon^1$ provides longer approximated subsequences if compared to $\epsilon^2$. Both embedding proposals tend to favor sequences containing one single node or one single subtrace due to the normalization of both  the edge and the nodes' distribution, but $\epsilon^2$ seems to be less influenced than $\epsilon^1$ in the change of the edge distribution. Therefore, $\epsilon^1$ proposal is to be preferred to $\epsilon^2$.
%\end{example}
%
The kernel $k_{\gorgembed}(G,G')$ can also be expressed as a function of the dot product of the two sub-embeddings because $\tasks^2\neq\tasks$.

\vspace*{-3mm}
\begin{equation}\label{eq:corollLem1}
\begin{array}{l}\omega\omega't_f^{|R>0|+|R'>0|}\Braket{\hat{\epsilon}(G), \hat{\epsilon}(G')}+t_f^{|R>0|+|R'>0|}\Braket{\hat{\nu}(G), \hat{\nu}(G')}
\end{array}  
\end{equation}

\noindent
\textbf{Properties.}\label{subsub:prop}
When two traces $\logtrace$ and ${\nonlogtrace}$ are equivalent (correspond to the same sequence of labels with the 
same probability), the kernel computation reduces to $\omega\omega'$. When both weights are $1$, the kernel returns $1$. 
We call this condition \textit{weak equality} because we cannot prove that when the kernel is equal to $\omega\omega'$ then the two traces are equivalent (there could be equal embeddings coming from non-equivalent traces). \figurename~\ref{fig:counterexample} show in fact two TGs generating a different set of traces but providing the same embedding.
%
Traces having neither $2$-grams nor transition labels in common have kernel $0$ and vice versa (\textit{strong dissimilarity}).
%
Due to weak equality and strong similarity, the embedding is weakly-ideal. {By previous lemmas, all combinations of sub-embeddings from Table~\ref{tab:embedstrat} give weakly-ideal $G$-embeddings.} 
%
Last, we provide the aforementioned lemmas' statements:

\begin{figure}[!t]
	\vspace*{-0.5cm}
	\centering
	\includegraphics[scale=0.7]{images/counterexample.pdf}
	\caption{Two $\tau$-closed TGs, $Q$ (left) and $Q'$ (right), having a different set of traces but the same embedding.}\label{fig:counterexample}
\end{figure}
%\begin{example}
%	Using $\epsilon$ (or $\epsilon^2$) and $\nu$ (or $\nu^2$) for $\gorgembed$ may yield false positives for ``weak equality'' 
%	if $Q=(V,s,s,L,R)$ and $Q'=(V,s',s',L,R)$ are both cycle graphs with $s\neq s'$, $\textit{label}(s)\neq\textit{label}(s')$, and
%	$\textit{label}(s),\textit{label}(s')\neq\tau$. The graphs in Figure \ref{fig:counterexample} have the 
%	same frequency for subtraces and nodes, and thus the same  $\epsilon$ and $\nu$ by construction. Using different 
%	initial and accepting node with  different labels, $\ptraces{Q}{0}=\{\textup{a(bca)}^n\mid n\in\mathbb{N}\}$ and 
%	$\ptraces{Q'}{0}=\{\textup{c(abc)}^n|n\in\mathbb{N}\}$. Thus, implying $\ptraces{Q}{0}\neq\ptraces{Q'}{0}$ but 
%	$k_{\gorgembed}(Q,Q')=1$ for $t_f=1$.
%\end{example}

\begin{lemma}[Weak Equality]
	\label{we}
	If two weighted TGs $(G,\omega)$ and $(G',\omega')$ yield the same set of weighted traces, then 
	$k_{\gorgembed}(G,G')=\omega\omega'$ for $t_f=1$.
\end{lemma}

 

\begin{lemma}[Strong Dissimilarity]
	\label{lem:sdiss}
	Given two weighted TGs $(G,\omega)$, $(G',\omega')$, $k_{\gorgembed}(G,G')=0$ iff $G$ and $G'$ have different set 
	of vertex labels or $2$-grams with $t_f,\omega,\omega'>0$.
\end{lemma}




%Last, under the assumption that a TG is approximately characterized by $\epsilon$ and $\nu$, we might expect that the TG similarity is characterized by the sum of the distance of both embeddings. Therefore, we show that an increase in both distance embeddings approximately corresponds to a decrease in the kernel output and vice-versa.
%
%\begin{lemma}\label{lem:approxRank}
%	Given two TGs $P=(s,t,L,R,w)$ and $P'=(s',t',L',R',w')$ having respectively the embeddings $(\epsilon,\nu)$ and $(\epsilon',\nu')$, we have that the kernel $k_{\trembed}$ induces an inverse ranking of $\norm{\hat{\epsilon}-\hat{\epsilon}'}{2}+\norm{\hat{\nu}-\hat{\nu}'}{2}$:
%	$$k_{\trembed}(P,P')\appropto 2-(\norm{\hat{\epsilon}-\hat{\epsilon}'}{2}+\norm{\hat{\nu}-\hat{\nu}'}{2})$$
%\end{lemma}
%\begin{proof}
%	{Let us use $T=t_f^{|R>0|+|R'>0|}$, $\omega=ww'$, $V=\norm{\hat{\nu}-\hat{\nu}'}{2}$, and $E=\norm{\hat{\epsilon}-\hat{\epsilon}'}{2}$ as shorthands. The goal can be rewritten as $k_{\trembed}(P,P')\appropto 2-(E+V)$. Given that the embeddings $(\epsilon,\nu)$ and $(\epsilon',\nu')$ are normalized kernel function $k_{\gorgembed}$ and that they are always positive definite, then we have that $0\leq E +V\leq 2$, so $0\leq 2-(E+V)\leq 2$. Using Proposition \ref{lem:rewritinglemma}, we can write $k_{\gorgembed}(P,P')$ as follows:}
%	$${\left(1-\frac{E}{2}\right)\omega T+\left(1-\frac{V}{2}\right)T=T\left(\omega+1-\frac{E\omega+V}{2}\right)}$$
%	{Given that the embeddings $(\epsilon,\nu)$ and $(\epsilon',\nu')$ are normalized in $k$ and that they are always positive definite,  we also have that $0\leq E\omega +V\leq 2$ where $0\leq \omega\leq 1$. We can also write  $0\leq \omega+1-\frac{E\omega+V}{2}\leq \frac{2}{T}$. For $\omega,T=1$, we have that $k_{\gorgembed}(P,P')=2-\frac{(E+V)}{2}$. Thus, $0<\omega,T<1$ approximates the expected ranking. }
%\end{proof}

%
%%\xout{As per previous observations, we know that}
%Two TGs should have the maximum dissimilarity when all the non $\tau$-nodes have different labels, thus making it impossible to find an alignment, thus implying that they share an utterly dissimilar set of weighted traces:
%
%%\end{proof}
%
%As a corollary of the two lemmas, we have that the proposed embedding performs weakly-ideally as defined in \S\ref{subsec:katk}, as equality condition holds in a relaxed form.
%

%
%%\ADD{Such lemma is going to be empirically evaluated in our experiment section.}

