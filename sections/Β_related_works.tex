\section{Related Works}

\paragraph*{Graph Kernels.} Graph kernels express similarity measures \cite{Samatova} involved in both classification \cite{TsudaS10} and clustering algorithms. One of the first approaches required a preliminary embedding definition of topological description vectors extracted from the most frequent subgraphs within a graph database \cite{Sidere}. As a drawback, it required the computation of a subgraph isomorphism problem, which is NP-complete. In fact, the definition of a graph kernel function fully recognizing the structure the graph always boils down to solving such NP-Complete problem \cite{GartnerFW03}, as exact embeddings generable in polynomial can be inferred just for loop-free Direct Acyclic Graphs \cite{BergamiBM20}. Consequently, most recent literature focused on extracting relevant features of such graphs, that are then used to define a graph similarity function. The most common approach adopted in the kernel to extract such features is called \textit{propositionalization}: we might extract all the possible features (e.g., subsequences), and then define a kernel function based on the occurrence and similarity of these features \cite{Gartner03}. For node-labelled graphs, the features come from the node labels and the possible strings that might be generated while traversing the graph (see \cite{Gartner03} and \S\ref{subsec:katk}). 